\chapter{Разработка метода прогнозирования временных рядов с использованием нечетких систем на основе правил}\label{ch:ch2}

\section{Задача нечеткого логического вывода}\label{sec:ch2/fuzzy-inference-problem-statement}

Нечеткая система на основе правил представляет собой базу правил вида:
\begin{equation}
	\label{eqn:fuz-problem-1}
	R_k:\ \text{Если}\ x_1\ \text{есть}\ A_{k1}\ \text{и}\ x_2\ \text{есть}\ A_{k2}\ \text{и} \dots \text{и}\ x_n\ \text{есть}\ A_{kn}, \text{то}\ y\ \text{есть}\ B_k,
\end{equation}
где $N$ "--- количество нечетких правил, $A_{ki} \subseteq X_i, i=\overline{1,n}, B_k \subseteq Y$"--- нечеткие множества, которые характеризуются функциями принадлежности $\mu_{A_{ki}}(x_i)$ и $\mu_{B_k}(y)$ соответственно; $x_1, x_2,…,x_n$"--- входные переменные, причем
\[
[x_1, x_2, ..., x_n]^T = \mathbf{x} \in X_1 \times X_2 \times \dots \times X_n.
\]

Символами  $X_i, i=\overline{1,n}$ и $Y$ обозначаются соответственно пространства входных и выходной переменных. Если ввести обозначения $\mathbf{X}=X_1 \times X_2 \times \dots \times X_n$ и $\mathbf{A_k}=A_{k1}\times A_{k2} \times \dots \times A_{kn}$ , причем
\[
\mu_\mathbf{A_k}(\mathbf{x}) = \underset{i=\overline{1,n}}{T_1} \mu_{A_{ki}}(x_i),
\]
где $T_1$ - произвольная $t$-норма, то правило \ref{eqn:fuz-problem-1} представляется в виде нечеткой импликации
\begin{equation}
	\label{eqn:fuz-problem-2}
	R_k: \mathbf{A_k} \to B_k, k=\overline{1,N}.
\end{equation}

Правило $R_k$ можно формализовать как нечеткое отношение, определенное на множестве  $\mathbf{X}\times Y$, т.е. $R_k \subseteq \mathbf{X} \times Y$ - нечеткое множество с функцией принадлежности
\[
\mu_{R_k}(\mathbf{x}, y) = \mu_{\mathbf{A_k} \to B_k} (\mathbf{x}, y).
\]

Модель логического типа определяет задание функции $\mu_{\mathbf{A_k} \to B_k} (\mathbf{x}, y)$ на основе известных функций принадлежности $\mu_{\mathbf{A_k}}(\mathbf{x})$ и $\mu_{B_k}(y)$ с помощью одной из функций нечеткой импликации:
\[
\mu_{\mathbf{A_k} \to B_k} (\mathbf{x}, y) = I(\mu_{\mathbf{A_k}}(\mathbf{x}), \mu_{B_k}(y)),
\]
где $I$"--- некоторая импликация.

Ставится задача определить нечеткий вывод $B'_k \subseteq Y$ для системы, представленной в виде (\ref{eqn:fuz-problem-1}), если на входах - нечеткие множества.
$\mathbf{A'}=A'_1 \times A'_2 \times \dots \times A'_n \subseteq \mathbf{X}$ или $x_1\ \text{есть}\ A'_1\ \text{и}\ x_2\ \text{есть}\ A'_2\ \text{и} \dots \text{и}\ x_n\ \text{есть}\ A'_n$  с соответствующей функцией принадлежности $\mu_{\mathbf{A'}}(\mathbf{x})$, которая определяется как
\begin{equation}
	\label{eqn:fuz-problem-3}
	\mu_{\mathbf{A'}}(\mathbf{x}) = \underset{i=\overline{1,n}}{T_3} \mu_{A'_i}(x_i).
\end{equation}

Несинглтонный фаззификатор отображает измеренное $x_i=x'_i, i=\overline{1,n}$ в нечеткое число, для которого $\mu_{A'_i}(x'_i) = 1$ и $\mu_{A'_i}(x_i)$ уменьшается от единицы по мере удаления от  $x'_i$.
В соответствии с обобщенным нечетким правилом modus ponens, нечеткое множество $B'_k$ определяется композицией нечеткого множества $\mathbf{A'}$ и отношения $\mathbf{R_k}$, т.е.
\[
B'_k = \mathbf{A'} \circ (\mathbf{A_k} \to B_k),
\]
или, на уровне функций принадлежности
\begin{equation}
	\label{eqn:fuz-problem-4}
	\mu_{B'_k}(y|\mathbf{x'}) = \sup_{\mathbf{x}\in \mathbf{X}}\left\{\mu_{\mathbf{A'}}(\mathbf{x'})\overset{T_2}{\star} I(\mu_{\mathbf{A_k}}(\mathbf{x}), \mu_{B_k}(y))\right\}.
\end{equation}

В (\ref{eqn:fuz-problem-4}) применена условная нотация, так как ввод в нечеткую систему происходит при определенном значении $\mathbf{x}$, а именно $\mathbf{x'}$. Обозначение $\mu_{B'_k}(y | \mathbf{x'})$ показывает, что $\mu_{B'_k}$ изменяется с каждым значением $\mathbf{x'}$. Вычислительная сложность выражения (\ref{eqn:fuz-problem-4}) составляет $O(|X_1|\cdot |X_2|\cdot \dots \cdot |X_n|\cdot |Y|)$, т.е. экспоненциальная. 

\section{Применение предложенного метода вывода в нечеткой модели для задачи прогнозирования временных рядов}

\subsection{Нечеткая модель прогнозирования}

Пусть задан временной ряд $\left\{y_t\right\}_{t=1}^T = \left\{y_1, \dots, y_T\right\}$, где $y_t \in \mathbb{R}$ --- измеренное значение наблюдаемой переменной в момент времени $t$, а $T$ --- длина доступной выборки. При моделировании временных последовательностей с использованием нейро-нечетких систем каждое значение $y_t\in \mathbb{Y}\subseteq \mathbb{R}$ фаззифицируется в нечеткое множество $A'_t$. Тогда для прогнозирования значения $\hat{y}_{t+h}$ с горизонтом $h$ на основании среза наблюдений $y_{t-p+1}, \dots, y_t$ можно использовать нечеткую систему с базой из $N$ правил вида:
\begin{align*}
	R_k: \text{Если } \bigwedge_{i=1}^p \left(y_{t-i+1}\text{ есть }A_{ki}\right)\text{, то }y_{t+1}\textrm{ есть }A_{k\,p+1}, k=\overline{1,N},
\end{align*}
где $p$ --- размер окна запаздывания (порядок модели, количество входов нечеткой системы).

\subsection{Несинглтонная фаззификация}

Большинство имеющихся реализаций нечеткого вывода предоставляют возможность анализа лишь четких --- числовых данных.

\begin{figure}[ht]
	\centerfloat{\includegraphics[scale=0.6]{singleton-vs-nonsingleton-fs}}
	\caption{Схема системы нечеткого вывода на основе правил с использованием синглтонной и несинглтонной фаззификации}
	\label{fig:singleton-vs-nonsingleton}
\end{figure}

Нечеткая система представляется в виде композиции фаззификатора, базы правил, модуля вывода и дефаззификатора, как показано на рисунке \cref{fig:singleton-vs-nonsingleton}. Фаззификация --- отображение входных данных из исходного пространства в пространство нечетких множеств. Наиболее распространенным является метод синглтонной фаззификации. Основная причина его широкого использования является существенное упрощение реализации систем нечеткого вывода. При использовании синглтонной фаззификации поданное на вход значение $x'$ интерпретируется как истинное значение измеренной величины, что эквивалентно использованию функции принадлежности входного нечеткого множества $\mu_{A'}(x) = \left[x = x'\right]$. При задании значений входных нечетких множеств нечеткой системы с использованием синглтонного типа фаззификации теряется информация о неопределенности измеренных значений $x'$ и они рассматриваются как достоверные. Поэтому в работе используется \textit{несинглтонный способ фаззификации}.

Альтернативный подход с использованием несинглтонной фаззификации предусматривает формализацию входного значения нечетким множеством, содержащим информацию о неопределенности значения точки входных данных. Эти неопределенности могут возникать как результат несовершенства процедуры измерений (например, шумом измерительного оборудования, дефектами или деградацией качества датчиков), или когда входные данные описываются качественными понятиями естественного языка.

В случае с анализом числовых данных, неопределенность измерений формализуется функцией принадлежности, которая $\mu_{A'}(x') = 1$ и $\mu_{A'}(x)$ уменьшается по мере удаления от $x'$. При таком способе моделирования измеренное значение $x'$ рассматривается как истинное, а значения в его окрестности --- как возможные.

\begin{figure}[ht]
	\begin{minipage}[b][][b]{0.3\linewidth}
		\centering
		\includegraphics[width=\linewidth, page=1]{ns-fuz-mendel-comparizon-first-order-partition} \\ а) $\sigma_{A'} = 0\%$
	\end{minipage}
	\hfill
	\begin{minipage}[b][][b]{0.3\linewidth}
		\centering
		\includegraphics[width=\linewidth, page=2]{ns-fuz-mendel-comparizon-first-order-partition} \\ б) $\sigma_{A'} = 4\%$
	\end{minipage}
	\hfill
	\begin{minipage}[b][][b]{0.3\linewidth}
		\centering
		\includegraphics[width=\linewidth, page=3]{ns-fuz-mendel-comparizon-first-order-partition} \\ в)  $\sigma_{A'} = 12\%$
	\end{minipage}
	
	\caption{Сравнение областей активации правил при переходе от синглтонной фаззификации к несинглтонной и при увеличении ширины области неопределенности $\sigma_{A'}$.}
\end{figure}

Влияние перехода от синглтонной фаззификации к несинглтонной и величины окрестности погрешности на внутренее поведение и итоговое качество нечеткой системы продемонстрировал Мендель в \cite{Mendel2021, Mendel2017}.

Для анализа влияния от использования того или иного способа фаззификации на корректность получаемого результата в статье \cite{Mendel2020} рассматривается карта разбиений первого и второго порядка на декартовом произведении базовых множеств входных лингвистических переменных. Мендель в своей книге проводил такое сравнение для систем типа Мамдани. Поскольку в системах типа Мамдани в качестве функции импликации выступает $t$-норма, то разница от применения двух способов фаззификации проявляется в различных максимальных уровнях линии пересечения ф. п. входной посылки и антецедента правила. %В зависимости от минимум или произведение. 

Применение композиционного правила вывода $\sup$ здесь дает эффект \textit{префильтрации} (или \textit{корректировки}) входного значения. То есть ядра функций принадлежности антецедентов правил выполняют функцию эталонов, а уровень пересечения функции принадлежности для входного значения и ф. п. антецедента правила позволяет интерпретировать входное значение как суперпозицию эталонных значений антецедентов с долями равными этому уровню.

%Разбиение первого порядка 

\pgfmathdeclarefunction{gauss}{3}{%
	\pgfmathparse{exp(-((#1-#2)/(2*#3))^2)} % Correct syntax for Gaussian function
}
\pgfmathdeclarefunction{implL}{4}{%
	\pgfmathparse{min(1, 1-#4+gauss(#1, #2, #3))} % Lukasievich
}
\pgfmathdeclarefunction{implR}{4}{%
	\pgfmathparse{min(1, gauss(#1, #2, #3)/#4)} % R
}

\pgfplotsset{
	membership axes/.style={
		width=0.20\textwidth,
		height=0.20\textwidth,
		scale only axis,
		domain=0:1,
		samples=100,
		every axis plot/.append style={smooth},
		axis lines=middle,
		xmin=0, xmax=1,
		ymin=0, ymax=1,
		xticklabel style={font=\tiny, inner sep=1pt, outer sep=0pt},
		yticklabel style={font=\tiny, inner sep=1pt, outer sep=0pt},
	}
}

\begin{figure}[ht]
	\centering
	\begin{subfigure}[b]{\textwidth}
		\begin{tikzpicture}
			
			%\begin{scope}[xshift=0cm, yshift=0cm]
			\pgfmathsetmacro{\aZeroMu}{0.55}
			\pgfmathsetmacro{\aFstMu}{0.28}
			\pgfmathsetmacro{\aFstSigma}{0.09}
			\pgfmathsetmacro{\aFstCrossX}{0.55}
			\pgfmathsetmacro{\aFstCrossY}{0.11}
			\pgfmathsetmacro{\aSndMu}{0.5}
			\pgfmathsetmacro{\aSndSigma}{0.08}
			\pgfmathsetmacro{\aSndCrossX}{0.55}
			\pgfmathsetmacro{\aSndCrossY}{0.91}
			\pgfmathsetmacro{\aTrdMu}{0.89}
			\pgfmathsetmacro{\aTrdSigma}{0.2}
			\pgfmathsetmacro{\aTrdCrossX}{0.55}
			\pgfmathsetmacro{\aTrdCrossY}{0.49}
			\pgfmathsetmacro{\bFstMu}{0.2}
			\pgfmathsetmacro{\bFstSigma}{0.12}
			\pgfmathsetmacro{\bSndMu}{0.5}
			\pgfmathsetmacro{\bSndSigma}{0.1}
			\pgfmathsetmacro{\bTrdMu}{0.8}
			\pgfmathsetmacro{\bTrdSigma}{0.1}
			
			% First function
			\begin{axis}[
				membership axes,
				name=plot1,
				at={(0,0)},
				xlabel={$x_1$},
				ylabel={$\mu$},
				xlabel style={font=\footnotesize, at=(current axis.south), anchor=north, yshift=-10pt},
				ylabel style={font=\footnotesize, at=(current axis.west), anchor=east, xshift=-10pt},
				]
				\addplot[blue, name path=A0] coordinates {(\aZeroMu, 0) (\aZeroMu, 1)};
				\addplot[black, name path=A1] {gauss(\x, \aFstMu, \aFstSigma)};
				\addplot[only marks, mark=*, mark size=2pt, purple!50] coordinates {(\aFstCrossX,\aFstCrossY)};
				\coordinate (fire11) at (axis cs:\aFstCrossX,\aFstCrossY); % USE axis cs
			\end{axis}
			% Second function
			\begin{axis}[
				membership axes,
				name=plot2,
				at={(0.25\textwidth,0)},
				xlabel={$x_2$},
				ylabel={$\mu$},
				xlabel style={font=\footnotesize, at=(current axis.south), anchor=north, yshift=-10pt},
				ylabel style={font=\footnotesize, at=(current axis.west), anchor=east, xshift=-10pt},
				]
				\addplot[blue, name path=A0] coordinates {(\aZeroMu, 0) (\aZeroMu, 1)};
				\addplot[black, name path=A1] {gauss(\x, \aSndMu, \aSndSigma)};
				\addplot[only marks, mark=*, mark size=2pt, lime] coordinates {(\aSndCrossX,\aSndCrossY)};
				\coordinate (fire12) at (axis cs:\aSndCrossX,\aSndCrossY); % Use axis cs
			\end{axis}
			% Third function
			\begin{axis}[
				membership axes,
				name=plot3,
				at={(0.5\textwidth,0)},
				xlabel={$x_3$},
				ylabel={$\mu$},
				xlabel style={font=\footnotesize, at=(current axis.south), anchor=north, yshift=-10pt},
				ylabel style={font=\footnotesize, at=(current axis.west), anchor=east, xshift=-10pt},
				]
				\addplot[blue, name path=A0] coordinates {(\aZeroMu, 0) (\aZeroMu, 1)};
				\addplot[black, name path=A1] {gauss(\x, \aTrdMu, \aTrdSigma)};
				\addplot[only marks, mark=*, mark size=2pt, cyan!50] coordinates {(\aTrdCrossX,\aTrdCrossY)};
				\coordinate (fire13) at (axis cs:\aTrdCrossX,\aTrdCrossY); % Use axis cs
			\end{axis}
			% Fourth function
			\begin{axis}[
				membership axes,
				name=plot4,
				at={(0.75\textwidth,0)},
				xlabel={$y$},
				ylabel={$\mu_{B'}(y)$},
				xlabel style={font=\footnotesize, at=(current axis.south), anchor=north, yshift=-10pt},
				ylabel style={font=\footnotesize, at=(current axis.above origin), anchor=south},
				]
				\addplot [purple!50] {min(\aFstCrossY, gauss(\x, \bFstMu, \bFstSigma))};
				\addplot [lime] {min(\aSndCrossY, gauss(\x, \bSndMu, \bSndSigma))};
				\addplot [cyan!50] {min(\aTrdCrossY, gauss(\x, \bTrdMu, \bTrdSigma))};
				\addplot[fill=orange!30, orange] {max(min(\aFstCrossY, gauss(\x, \bFstMu, \bFstSigma)), min(\aSndCrossY, gauss(\x, \bSndMu, \bSndSigma)), min(\aTrdCrossY, gauss(\x, \bTrdMu, \bTrdSigma)))} \closedcycle;
				\coordinate (out11) at (axis cs:0.0,\aFstCrossY); % Use axis cs
				\coordinate (out12) at (axis cs:0.0,\aSndCrossY); % Use axis cs
				\coordinate (out13) at (axis cs:0.0,\aTrdCrossY); % Use axis cs
			\end{axis}
			
			% Connect coordinates using defined coordinate names
			\draw [purple!50, dashed, thick] (fire11) -- (out11);
			\draw [lime, dashed, thick] (fire12) -- (out12);
			\draw [cyan!50, dashed, thick] (fire13) -- (out13);
			
			% \node {\sigma=\aZeroSigma};
			\node[left, xshift=-1.2cm, font=\footnotesize, rotate=90, anchor=center] at (plot1.west) {$\sigma_{A'}=0$};
		\end{tikzpicture}
	\end{subfigure}
	
	%%
	\vspace{0.1cm}
	%%
	
	\begin{subfigure}[b]{\textwidth}
		\begin{tikzpicture}
			
			%\begin{scope}[xshift=0cm, yshift=0cm]
			\pgfmathsetmacro{\aZeroMu}{0.55}
			\pgfmathsetmacro{\aZeroSigma}{0.01}
			\pgfmathsetmacro{\aFstMu}{0.28}
			\pgfmathsetmacro{\aFstSigma}{0.09}
			\pgfmathsetmacro{\aFstCrossX}{0.515}
			\pgfmathsetmacro{\aFstCrossY}{0.185}
			\pgfmathsetmacro{\aSndMu}{0.5}
			\pgfmathsetmacro{\aSndSigma}{0.08}
			\pgfmathsetmacro{\aSndCrossX}{0.56}
			\pgfmathsetmacro{\aSndCrossY}{0.88}
			\pgfmathsetmacro{\aTrdMu}{0.89}
			\pgfmathsetmacro{\aTrdSigma}{0.2}
			\pgfmathsetmacro{\aTrdCrossX}{0.57}
			\pgfmathsetmacro{\aTrdCrossY}{0.53}
			\pgfmathsetmacro{\bFstMu}{0.2}
			\pgfmathsetmacro{\bFstSigma}{0.12}
			\pgfmathsetmacro{\bSndMu}{0.5}
			\pgfmathsetmacro{\bSndSigma}{0.1}
			\pgfmathsetmacro{\bTrdMu}{0.8}
			\pgfmathsetmacro{\bTrdSigma}{0.1}
			
			% First function
			\begin{axis}[
				membership axes,
				name=plot1,
				at={(0,0)},
				xlabel={$x_1$},
				ylabel={$\mu$},
				xlabel style={},
				xlabel style={font=\footnotesize, at=(current axis.south), anchor=north, yshift=-10pt},
				ylabel style={font=\footnotesize, at=(current axis.west), anchor=east, xshift=-10pt},
				]
				\addplot[blue, name path=A0] {gauss(\x, \aZeroMu, \aZeroSigma)};
				\addplot[black, name path=A1] {gauss(\x, \aFstMu, \aFstSigma)};
				\addplot[fill=purple!50, draw=none] {min(gauss(\x, \aZeroMu, \aZeroSigma), gauss(\x, \aFstMu, \aFstSigma))} \closedcycle;
				\coordinate (fire11) at (axis cs:\aFstCrossX,\aFstCrossY); % USE axis cs
			\end{axis}
			% Second function
			\begin{axis}[
				membership axes,
				name=plot2,
				at={(0.25\textwidth,0)},
				xlabel={$x_2$},
				ylabel={$\mu$},
				xlabel style={font=\footnotesize, at=(current axis.south), anchor=north, yshift=-10pt},
				ylabel style={font=\footnotesize, at=(current axis.west), anchor=east, xshift=-10pt},
				]
				\addplot[blue, name path=A0] {gauss(\x, \aZeroMu, \aZeroSigma)};
				\addplot[black, name path=A1] {gauss(\x, \aSndMu, \aSndSigma)};
				\addplot[fill=lime, draw=none] {min(gauss(\x, \aZeroMu, \aZeroSigma), gauss(\x, \aSndMu, \aSndSigma))} \closedcycle;
				\coordinate (fire12) at (axis cs:\aSndCrossX,\aSndCrossY); % Use axis cs
			\end{axis}
			% Third function
			\begin{axis}[
				membership axes,
				name=plot3,
				at={(0.5\textwidth,0)},
				xlabel={$x_3$},
				ylabel={$\mu$},
				xlabel style={font=\footnotesize, at=(current axis.south), anchor=north, yshift=-10pt},
				ylabel style={font=\footnotesize, at=(current axis.west), anchor=east, xshift=-10pt},
				]
				\addplot[blue, name path=A0] {gauss(\x, \aZeroMu, \aZeroSigma)};
				\addplot[black, name path=A1] {gauss(\x, \aTrdMu, \aTrdSigma)};
				\addplot[fill=cyan!50, draw=none] {min(gauss(\x, \aZeroMu, \aZeroSigma), gauss(\x, \aTrdMu, \aTrdSigma))} \closedcycle;
				\coordinate (fire13) at (axis cs:\aTrdCrossX,\aTrdCrossY); % Use axis cs
			\end{axis}
			% Fourth function
			\begin{axis}[
				membership axes,
				name=plot4,
				at={(0.75\textwidth,0)},
				xlabel={$y$},
				ylabel={$\mu_{B'}(y)$},
				xlabel style={font=\footnotesize, at=(current axis.south), anchor=north, yshift=-10pt},
				ylabel style={font=\footnotesize, at=(current axis.above origin), anchor=south},
				]
				\addplot [purple!50] {min(\aFstCrossY, gauss(\x, \bFstMu, \bFstSigma))};
				\addplot [lime] {min(\aSndCrossY, gauss(\x, \bSndMu, \bSndSigma))};
				\addplot [cyan!50] {min(\aTrdCrossY, gauss(\x, \bTrdMu, \bTrdSigma))};
				\addplot[fill=orange!30, orange] {max(min(\aFstCrossY, gauss(\x, \bFstMu, \bFstSigma)), min(\aSndCrossY, gauss(\x, \bSndMu, \bSndSigma)), min(\aTrdCrossY, gauss(\x, \bTrdMu, \bTrdSigma)))} \closedcycle;
				\coordinate (out11) at (axis cs:0.0,\aFstCrossY); % Use axis cs
				\coordinate (out12) at (axis cs:0.0,\aSndCrossY); % Use axis cs
				\coordinate (out13) at (axis cs:0.0,\aTrdCrossY); % Use axis cs
			\end{axis}
			
			% Connect coordinates using defined coordinate names
			\draw [purple!50, dashed, thick] (fire11) -- (out11);
			\draw [lime, dashed, thick] (fire12) -- (out12);
			\draw [cyan!50, dashed, thick] (fire13) -- (out13);
			
			% \node {\sigma=\aZeroSigma};
			\node[left, xshift=-1.2cm, font=\footnotesize, rotate=90, anchor=center] at (plot1.west) {$\sigma_{A'}=\aZeroSigma$};
		\end{tikzpicture}
	\end{subfigure}
	
	%%
	\vspace{0.1cm}
	%%
	
	\begin{subfigure}[b]{\textwidth}
		\begin{tikzpicture}
			\pgfmathsetmacro{\aZeroMu}{0.55}
			\pgfmathsetmacro{\aZeroSigma}{0.05}
			\pgfmathsetmacro{\aFstMu}{0.28}
			\pgfmathsetmacro{\aFstSigma}{0.09}
			\pgfmathsetmacro{\aFstCrossX}{0.49}
			\pgfmathsetmacro{\aFstCrossY}{0.4}
			\pgfmathsetmacro{\aSndMu}{0.5}
			\pgfmathsetmacro{\aSndSigma}{0.08}
			\pgfmathsetmacro{\aSndCrossX}{0.53}
			\pgfmathsetmacro{\aSndCrossY}{0.97}
			\pgfmathsetmacro{\aTrdMu}{0.89}
			\pgfmathsetmacro{\aTrdSigma}{0.2}
			\pgfmathsetmacro{\aTrdCrossX}{0.6}
			\pgfmathsetmacro{\aTrdCrossY}{0.61}
			\pgfmathsetmacro{\bFstMu}{0.2}
			\pgfmathsetmacro{\bFstSigma}{0.12}
			\pgfmathsetmacro{\bSndMu}{0.5}
			\pgfmathsetmacro{\bSndSigma}{0.1}
			\pgfmathsetmacro{\bTrdMu}{0.8}
			\pgfmathsetmacro{\bTrdSigma}{0.1}
			
			% First function
			\begin{axis}[
				membership axes,
				name=plot1,
				at={(0,0)},
				xlabel={$x_1$},
				ylabel={$\mu$},
				xlabel style={font=\footnotesize, at=(current axis.south), anchor=north, yshift=-10pt},
				ylabel style={font=\footnotesize, at=(current axis.west), anchor=east, xshift=-10pt},
				]
				\addplot[blue, name path=A0] {gauss(\x, \aZeroMu, \aZeroSigma)};
				\addplot[black, name path=A1] {gauss(\x, \aFstMu, \aFstSigma)};
				\addplot[fill=purple!50, draw=none] {min(gauss(\x, \aZeroMu, \aZeroSigma), gauss(\x, \aFstMu, \aFstSigma))} \closedcycle;
				\coordinate (fire11) at (axis cs:\aFstCrossX,\aFstCrossY); % USE axis cs
			\end{axis}
			% Second function
			\begin{axis}[
				membership axes,
				name=plot2,
				at={(0.25\textwidth,0)},
				xlabel={$x_2$},
				ylabel={$\mu$},
				xlabel style={font=\footnotesize, at=(current axis.south), anchor=north, yshift=-10pt},
				ylabel style={font=\footnotesize, at=(current axis.west), anchor=east, xshift=-10pt},
				]
				\addplot[blue, name path=A0] {gauss(\x, \aZeroMu, \aZeroSigma)};
				\addplot[black, name path=A1] {gauss(\x, \aSndMu, \aSndSigma)};
				\addplot[fill=lime, draw=none] {min(gauss(\x, \aZeroMu, \aZeroSigma), gauss(\x, \aSndMu, \aSndSigma))} \closedcycle;
				\coordinate (fire12) at (axis cs:\aSndCrossX,\aSndCrossY); % Use axis cs
			\end{axis}
			% Third function
			\begin{axis}[
				membership axes,
				name=plot2,
				at={(0.5\textwidth,0)},
				xlabel={$x_3$},
				ylabel={$\mu$},
				xlabel style={font=\footnotesize, at=(current axis.south), anchor=north, yshift=-10pt},
				ylabel style={font=\footnotesize, at=(current axis.west), anchor=east, xshift=-10pt},
				]
				\addplot[blue, name path=A0] {gauss(\x, \aZeroMu, \aZeroSigma)};
				\addplot[black, name path=A1] {gauss(\x, \aTrdMu, \aTrdSigma)};
				\addplot[fill=cyan!50, draw=none] {min(gauss(\x, \aZeroMu, \aZeroSigma), gauss(\x, \aTrdMu, \aTrdSigma))} \closedcycle;
				\coordinate (fire13) at (axis cs:\aTrdCrossX,\aTrdCrossY); % Use axis cs
			\end{axis}
			% Fourth function
			\begin{axis}[
				membership axes,
				name=plot4,
				at={(0.75\textwidth,0)},
				xlabel={$y$},
				ylabel={$\mu_{B'}(y)$},
				xlabel style={font=\footnotesize, at=(current axis.south), anchor=north, yshift=-10pt},
				ylabel style={font=\footnotesize, at=(current axis.above origin), anchor=south},
				]
				\addplot [purple!50] {min(\aFstCrossY, gauss(\x, \bFstMu, \bFstSigma))};
				\addplot [lime] {min(\aSndCrossY, gauss(\x, \bSndMu, \bSndSigma))};
				\addplot [cyan!50] {min(\aTrdCrossY, gauss(\x, \bTrdMu, \bTrdSigma))};
				\addplot[fill=orange!30, orange] {max(min(\aFstCrossY, gauss(\x, \bFstMu, \bFstSigma)), min(\aSndCrossY, gauss(\x, \bSndMu, \bSndSigma)), min(\aTrdCrossY, gauss(\x, \bTrdMu, \bTrdSigma)))} \closedcycle;
				\coordinate (out11) at (axis cs:0.0,\aFstCrossY); % Use axis cs
				\coordinate (out12) at (axis cs:0.0,\aSndCrossY); % Use axis cs
				\coordinate (out13) at (axis cs:0.0,\aTrdCrossY); % Use axis cs
			\end{axis}
			
			% Connect coordinates using defined coordinate names
			\draw [purple!50, dashed, thick] (fire11) -- (out11);
			\draw [lime, dashed, thick] (fire12) -- (out12);
			\draw [cyan!50, dashed, thick] (fire13) -- (out13);
			
			% \node {\sigma=\aZeroSigma};
			\node[left, xshift=-1.2cm, font=\footnotesize, rotate=90, anchor=center] at (plot1.west) {$\sigma_{A'}=\aZeroSigma$};
		\end{tikzpicture}
	\end{subfigure}
	\caption{Сравнение формы функций принадлежности выходных нечетких множеств для подхода Мамдани}
	\label{fig:ns-width-influence-to-out-mamdani}
\end{figure}

Показанная на этих схемах динамика более ясно раскрыта на рисунке \cref{fig:ns-width-influence-to-out-mamdani} для различных значениях среднеквадратичного отклонения в гауссовой ф. п. входного значения на примере агрегации выходной ф. п. нечеткой системы с тремя правилами в базе правил. Видно, что при переходе от синглтонной фаззификации к несинглтонной и при дальнейшем увеличении ширины среднеквадратичного отклонения ф. п. входного нечеткого множества, повышается уровень срабатывания первого правила, и, как следствие использования импликации Мамдани, уровень задействования ф. п. выходного нечеткого множества этого правила в результирующей агрегации. Кроме того, можно пронаблюдать, упомянутый ранее, эффект корректировки входного значения антецедентом третьего правила.

В случае с нечеткой системой на основе вывода логического типа, разница от использования различных способов фаззификации будет проявляться в различных формах выходного нечеткого множества.

\begin{figure}[ht]
	\centering
	\begin{subfigure}[b]{\textwidth}
		\begin{tikzpicture}
			
			%\begin{scope}[xshift=0cm, yshift=0cm]
			\pgfmathsetmacro{\aZeroMu}{0.55}
			\pgfmathsetmacro{\aFstMu}{0.28}
			\pgfmathsetmacro{\aFstSigma}{0.09}
			\pgfmathsetmacro{\aFstCrossX}{0.55}
			\pgfmathsetmacro{\aFstCrossY}{0.11}
			\pgfmathsetmacro{\aSndMu}{0.5}
			\pgfmathsetmacro{\aSndSigma}{0.08}
			\pgfmathsetmacro{\aSndCrossX}{0.55}
			\pgfmathsetmacro{\aSndCrossY}{0.91}
			\pgfmathsetmacro{\aTrdMu}{0.89}
			\pgfmathsetmacro{\aTrdSigma}{0.2}
			\pgfmathsetmacro{\aTrdCrossX}{0.55}
			\pgfmathsetmacro{\aTrdCrossY}{0.49}
			\pgfmathsetmacro{\bFstMu}{0.2}
			\pgfmathsetmacro{\bFstSigma}{0.12}
			\pgfmathsetmacro{\bSndMu}{0.5}
			\pgfmathsetmacro{\bSndSigma}{0.1}
			\pgfmathsetmacro{\bTrdMu}{0.8}
			\pgfmathsetmacro{\bTrdSigma}{0.1}
			
			% First function
			\begin{axis}[
				membership axes,
				name=plot1,
				at={(0,0)},
				xlabel={$x_1$},
				ylabel={$\mu$},
				xlabel style={font=\footnotesize, at=(current axis.south), anchor=north},
				ylabel style={font=\footnotesize, at=(current axis.west), anchor=east, xshift=-10pt},
				y dir=reverse,
				% axis x line=top,
				xticklabel pos=upper,
				xticklabel style={yshift=13pt}
				]
				\addplot[blue, name path=A0] coordinates {(\aZeroMu, 0) (\aZeroMu, 1)};
				\addplot[black, name path=A1] {gauss(\x, \aFstMu, \aFstSigma)};
				\addplot[only marks, mark=*, mark size=2pt, purple!50] coordinates {(\aFstCrossX,\aFstCrossY)};
				\coordinate (fire11) at (axis cs:\aFstCrossX,\aFstCrossY); % USE axis cs
			\end{axis}
			% Second function
			\begin{axis}[
				membership axes,
				name=plot2,
				at={(0.25\textwidth,0)},
				xlabel={$x_2$},
				ylabel={$\mu$},
				xlabel style={font=\footnotesize, at=(current axis.south), anchor=north},
				ylabel style={font=\footnotesize, at=(current axis.west), anchor=east, xshift=-10pt},
				y dir=reverse,
				% axis x line=top,
				xticklabel pos=upper,
				xticklabel style={yshift=13pt}
				]
				\addplot[blue, name path=A0] coordinates {(\aZeroMu, 0) (\aZeroMu, 1)};
				\addplot[black, name path=A1] {gauss(\x, \aSndMu, \aSndSigma)};
				\addplot[only marks, mark=*, mark size=2pt, lime] coordinates {(\aSndCrossX,\aSndCrossY)};
				\coordinate (fire12) at (axis cs:\aSndCrossX,\aSndCrossY); % Use axis cs
			\end{axis}
			% Third function
			\begin{axis}[
				membership axes,
				name=plot3,
				at={(0.5\textwidth,0)},
				xlabel={$x_3$},
				ylabel={$\mu$},
				xlabel style={font=\footnotesize, at=(current axis.south), anchor=north},
				ylabel style={font=\footnotesize, at=(current axis.west), anchor=east, xshift=-10pt},
				y dir=reverse,
				% axis x line=top,
				xticklabel pos=upper,
				xticklabel style={yshift=13pt}
				]
				\addplot[blue, name path=A0] coordinates {(\aZeroMu, 0) (\aZeroMu, 1)};
				\addplot[black, name path=A1] {gauss(\x, \aTrdMu, \aTrdSigma)};
				\addplot[only marks, mark=*, mark size=2pt, cyan!50] coordinates {(\aTrdCrossX,\aTrdCrossY)};
				\coordinate (fire13) at (axis cs:\aTrdCrossX,\aTrdCrossY); % Use axis cs
			\end{axis}
			% Fourth function
			\begin{axis}[
				membership axes,
				name=plot4,
				at={(0.75\textwidth,0)},
				xlabel={$y$},
				ylabel={$\mu_{B'}(y)$},
				xlabel style={font=\footnotesize, at=(current axis.south), anchor=north, yshift=-10pt},
				ylabel style={font=\footnotesize, at=(current axis.above origin), anchor=south},
				]
				\addplot [purple!50] {implL(\x, \bFstMu, \bFstSigma, \aFstCrossY)};
				\addplot [lime] {implL(\x, \bSndMu, \bSndSigma, \aSndCrossY)};
				\addplot [cyan!50] {implL(\x, \bTrdMu, \bTrdSigma, \aTrdCrossY)};
				\addplot[fill=orange!30, orange] {min(implL(\x, \bFstMu, \bFstSigma, \aFstCrossY), implL(\x, \bSndMu, \bSndSigma, \aSndCrossY), implL(\x, \bTrdMu, \bTrdSigma, \aTrdCrossY))} \closedcycle;
				\coordinate (out11) at (axis cs:0.0,1-\aFstCrossY); % Use axis cs
				\coordinate (out12) at (axis cs:0.0,1-\aSndCrossY); % Use axis cs
				\coordinate (out13) at (axis cs:0.0,1-\aTrdCrossY); % Use axis cs
			\end{axis}
			
			% Connect coordinates using defined coordinate names
			\draw [purple!50, dashed, thick] (fire11) -- (out11);
			\draw [lime, dashed, thick] (fire12) -- (out12);
			\draw [cyan!50, dashed, thick] (fire13) -- (out13);
			
			% \node {\sigma=\aZeroSigma};
			\node[left, xshift=-1.2cm, font=\footnotesize, rotate=90, anchor=center] at (plot1.west) {$\sigma_{A'}=0$};
		\end{tikzpicture}
	\end{subfigure}
	
	%%
	\vspace{0.1cm}
	%%
	
	\begin{subfigure}[b]{\textwidth}
		\begin{tikzpicture}
			
			%\begin{scope}[xshift=0cm, yshift=0cm]
			\pgfmathsetmacro{\aZeroMu}{0.55}
			\pgfmathsetmacro{\aZeroSigma}{0.01}
			\pgfmathsetmacro{\aFstMu}{0.28}
			\pgfmathsetmacro{\aFstSigma}{0.09}
			\pgfmathsetmacro{\aFstCrossX}{0.515}
			\pgfmathsetmacro{\aFstCrossY}{0.185}
			\pgfmathsetmacro{\aSndMu}{0.5}
			\pgfmathsetmacro{\aSndSigma}{0.08}
			\pgfmathsetmacro{\aSndCrossX}{0.56}
			\pgfmathsetmacro{\aSndCrossY}{0.88}
			\pgfmathsetmacro{\aTrdMu}{0.89}
			\pgfmathsetmacro{\aTrdSigma}{0.2}
			\pgfmathsetmacro{\aTrdCrossX}{0.57}
			\pgfmathsetmacro{\aTrdCrossY}{0.53}
			\pgfmathsetmacro{\bFstMu}{0.2}
			\pgfmathsetmacro{\bFstSigma}{0.12}
			\pgfmathsetmacro{\bSndMu}{0.5}
			\pgfmathsetmacro{\bSndSigma}{0.1}
			\pgfmathsetmacro{\bTrdMu}{0.8}
			\pgfmathsetmacro{\bTrdSigma}{0.1}
			
			% First function
			\begin{axis}[
				membership axes,
				name=plot1,
				at={(0,0)},
				xlabel={$x_1$},
				ylabel={$\mu$},
				xlabel style={font=\footnotesize, at=(current axis.south), anchor=north},
				ylabel style={font=\footnotesize, at=(current axis.west), anchor=east, xshift=-10pt},
				y dir=reverse,
				% axis x line=top,
				xticklabel pos=upper,
				xticklabel style={yshift=13pt}
				]
				\addplot[blue, name path=A0] {gauss(\x, \aZeroMu, \aZeroSigma)};
				\addplot[black, name path=A1] {gauss(\x, \aFstMu, \aFstSigma)};
				\addplot[fill=purple!50, draw=none] {min(gauss(\x, \aZeroMu, \aZeroSigma), gauss(\x, \aFstMu, \aFstSigma))} \closedcycle;
				\coordinate (fire11) at (axis cs:\aFstCrossX,\aFstCrossY); % USE axis cs
			\end{axis}
			% Second function
			\begin{axis}[
				membership axes,
				name=plot2,
				at={(0.25\textwidth,0)},
				xlabel={$x_2$},
				ylabel={$\mu$},
				xlabel style={font=\footnotesize, at=(current axis.south), anchor=north},
				ylabel style={font=\footnotesize, at=(current axis.west), anchor=east, xshift=-10pt},
				y dir=reverse,
				% axis x line=top,
				xticklabel pos=upper,
				xticklabel style={yshift=13pt}
				]
				\addplot[blue, name path=A0] {gauss(\x, \aZeroMu, \aZeroSigma)};
				\addplot[black, name path=A1] {gauss(\x, \aSndMu, \aSndSigma)};
				\addplot[fill=lime, draw=none] {min(gauss(\x, \aZeroMu, \aZeroSigma), gauss(\x, \aSndMu, \aSndSigma))} \closedcycle;
				\coordinate (fire12) at (axis cs:\aSndCrossX,\aSndCrossY); % Use axis cs
			\end{axis}
			% Third function
			\begin{axis}[
				membership axes,
				name=plot3,
				at={(0.5\textwidth,0)},
				xlabel={$x_3$},
				ylabel={$\mu$},
				xlabel style={font=\footnotesize, at=(current axis.south), anchor=north},
				ylabel style={font=\footnotesize, at=(current axis.west), anchor=east, xshift=-10pt},
				y dir=reverse,
				% axis x line=top,
				xticklabel pos=upper,
				xticklabel style={yshift=13pt}
				]
				\addplot[blue, name path=A0] {gauss(\x, \aZeroMu, \aZeroSigma)};
				\addplot[black, name path=A1] {gauss(\x, \aTrdMu, \aTrdSigma)};
				\addplot[fill=cyan!50, draw=none] {min(gauss(\x, \aZeroMu, \aZeroSigma), gauss(\x, \aTrdMu, \aTrdSigma))} \closedcycle;
				\coordinate (fire13) at (axis cs:\aTrdCrossX,\aTrdCrossY); % Use axis cs
			\end{axis}
			% Fourth function
			\begin{axis}[
				membership axes,
				name=plot4,
				at={(0.75\textwidth,0)},
				xlabel={$y$},
				ylabel={$\mu_{B'}(y)$},
				xlabel style={font=\footnotesize, at=(current axis.south), anchor=north, yshift=-10pt},
				ylabel style={font=\footnotesize, at=(current axis.above origin), anchor=south},
				]
				\addplot [purple!50] {implL(\x, \bFstMu, \bFstSigma, \aFstCrossY)};
				\addplot [lime] {implL(\x, \bSndMu, \bSndSigma, \aSndCrossY)};
				\addplot [cyan!50] {implL(\x, \bTrdMu, \bTrdSigma, \aTrdCrossY)};
				\addplot[fill=orange!30, orange] {min(implL(\x, \bFstMu, \bFstSigma, \aFstCrossY), implL(\x, \bSndMu, \bSndSigma, \aSndCrossY), implL(\x, \bTrdMu, \bTrdSigma, \aTrdCrossY))} \closedcycle;
				\coordinate (out11) at (axis cs:0.0,1-\aFstCrossY); % Use axis cs
				\coordinate (out12) at (axis cs:0.0,1-\aSndCrossY); % Use axis cs
				\coordinate (out13) at (axis cs:0.0,1-\aTrdCrossY); % Use axis cs
			\end{axis}
			
			% Connect coordinates using defined coordinate names
			\draw [purple!50, dashed, thick] (fire11) -- (out11);
			\draw [lime, dashed, thick] (fire12) -- (out12);
			\draw [cyan!50, dashed, thick] (fire13) -- (out13);
			
			% \node {\sigma=\aZeroSigma};
			\node[left, xshift=-1.2cm, font=\footnotesize, rotate=90, anchor=center] at (plot1.west) {$\sigma_{A'}=\aZeroSigma$};
		\end{tikzpicture}
	\end{subfigure}
	
	%%
	\vspace{0.1cm}
	%%
	
	\begin{subfigure}[b]{\textwidth}
		\begin{tikzpicture}
			\pgfmathsetmacro{\aZeroMu}{0.55}
			\pgfmathsetmacro{\aZeroSigma}{0.05}
			\pgfmathsetmacro{\aFstMu}{0.28}
			\pgfmathsetmacro{\aFstSigma}{0.09}
			\pgfmathsetmacro{\aFstCrossX}{0.49}
			\pgfmathsetmacro{\aFstCrossY}{0.4}
			\pgfmathsetmacro{\aSndMu}{0.5}
			\pgfmathsetmacro{\aSndSigma}{0.08}
			\pgfmathsetmacro{\aSndCrossX}{0.53}
			\pgfmathsetmacro{\aSndCrossY}{0.97}
			\pgfmathsetmacro{\aTrdMu}{0.89}
			\pgfmathsetmacro{\aTrdSigma}{0.2}
			\pgfmathsetmacro{\aTrdCrossX}{0.6}
			\pgfmathsetmacro{\aTrdCrossY}{0.61}
			\pgfmathsetmacro{\bFstMu}{0.2}
			\pgfmathsetmacro{\bFstSigma}{0.12}
			\pgfmathsetmacro{\bSndMu}{0.5}
			\pgfmathsetmacro{\bSndSigma}{0.1}
			\pgfmathsetmacro{\bTrdMu}{0.8}
			\pgfmathsetmacro{\bTrdSigma}{0.1}
			
			% First function
			\begin{axis}[
				membership axes,
				name=plot1,
				at={(0,0)},
				xlabel={$x_1$},
				ylabel={$\mu$},
				xlabel style={font=\footnotesize, at=(current axis.south), anchor=north},
				ylabel style={font=\footnotesize, at=(current axis.west), anchor=east, xshift=-10pt},
				y dir=reverse,
				% axis x line=top,
				xticklabel pos=upper,
				xticklabel style={yshift=13pt}
				]
				\addplot[blue, name path=A0] {gauss(\x, \aZeroMu, \aZeroSigma)};
				\addplot[black, name path=A1] {gauss(\x, \aFstMu, \aFstSigma)};
				\addplot[fill=purple!50, draw=none] {min(gauss(\x, \aZeroMu, \aZeroSigma), gauss(\x, \aFstMu, \aFstSigma))} \closedcycle;
				\coordinate (fire11) at (axis cs:\aFstCrossX,\aFstCrossY); % USE axis cs
			\end{axis}
			% Second function
			\begin{axis}[
				membership axes,
				name=plot2,
				at={(0.25\textwidth,0)},
				xlabel={$x_2$},
				ylabel={$\mu$},
				xlabel style={font=\footnotesize, at=(current axis.south), anchor=north},
				ylabel style={font=\footnotesize, at=(current axis.west), anchor=east, xshift=-10pt},
				y dir=reverse,
				% axis x line=top,
				xticklabel pos=upper,
				xticklabel style={yshift=13pt}
				]
				\addplot[blue, name path=A0] {gauss(\x, \aZeroMu, \aZeroSigma)};
				\addplot[black, name path=A1] {gauss(\x, \aSndMu, \aSndSigma)};
				\addplot[fill=lime, draw=none] {min(gauss(\x, \aZeroMu, \aZeroSigma), gauss(\x, \aSndMu, \aSndSigma))} \closedcycle;
				\coordinate (fire12) at (axis cs:\aSndCrossX,\aSndCrossY); % Use axis cs
			\end{axis}
			% Third function
			\begin{axis}[
				membership axes,
				name=plot2,
				at={(0.5\textwidth,0)},
				xlabel={$x_3$},
				ylabel={$\mu$},
				xlabel style={font=\footnotesize, at=(current axis.south), anchor=north},
				ylabel style={font=\footnotesize, at=(current axis.west), anchor=east, xshift=-10pt},
				y dir=reverse,
				% axis x line=top,
				xticklabel pos=upper,
				xticklabel style={yshift=13pt}
				]
				\addplot[blue, name path=A0] {gauss(\x, \aZeroMu, \aZeroSigma)};
				\addplot[black, name path=A1] {gauss(\x, \aTrdMu, \aTrdSigma)};
				\addplot[fill=cyan!50, draw=none] {min(gauss(\x, \aZeroMu, \aZeroSigma), gauss(\x, \aTrdMu, \aTrdSigma))} \closedcycle;
				\coordinate (fire13) at (axis cs:\aTrdCrossX,\aTrdCrossY); % Use axis cs
			\end{axis}
			% Fourth function
			\begin{axis}[
				membership axes,
				name=plot4,
				at={(0.75\textwidth,0)},
				xlabel={$y$},
				ylabel={$\mu_{B'}(y)$},
				xlabel style={font=\footnotesize, at=(current axis.south), anchor=north, yshift=-10pt},
				ylabel style={font=\footnotesize, at=(current axis.above origin), anchor=south},
				]
				\addplot [purple!50] {implL(\x, \bFstMu, \bFstSigma, \aFstCrossY)};
				\addplot [lime] {implL(\x, \bSndMu, \bSndSigma, \aSndCrossY)};
				\addplot [cyan!50] {implL(\x, \bTrdMu, \bTrdSigma, \aTrdCrossY)};
				\addplot[fill=orange!30, orange] {min(implL(\x, \bFstMu, \bFstSigma, \aFstCrossY), implL(\x, \bSndMu, \bSndSigma, \aSndCrossY), implL(\x, \bTrdMu, \bTrdSigma, \aTrdCrossY))} \closedcycle;
				\coordinate (out11) at (axis cs:0.0,1-\aFstCrossY); % Use axis cs
				\coordinate (out12) at (axis cs:0.0,1-\aSndCrossY); % Use axis cs
				\coordinate (out13) at (axis cs:0.0,1-\aTrdCrossY); % Use axis cs
			\end{axis}
			
			% Connect coordinates using defined coordinate names
			\draw [purple!50, dashed, thick] (fire11) -- (out11);
			\draw [lime, dashed, thick] (fire12) -- (out12);
			\draw [cyan!50, dashed, thick] (fire13) -- (out13);
			
			% \node {\sigma=\aZeroSigma};
			\node[left, xshift=-1.2cm, font=\footnotesize, rotate=90, anchor=center] at (plot1.west) {$\sigma_{A'}=\aZeroSigma$};
		\end{tikzpicture}
	\end{subfigure}
	\caption{Сравнение формы функций принадлежности выходных нечетких множеств для логического подхода. Выходные нечеткие множества получены в результате применения импликации Лукасевича $I(a, b)=1-a+b$, поэтому для легкости восприятия антецеденты правил и входные нечеткие множества показаны в виде выражений $1-\mu_{A_k}(x)$ и $1-\mu_{A'}(x)$.}
	\label{fig:ns-width-influence-to-out-logical}
\end{figure}


Можно проследить за влиянием увеличения ширины окна для измеренного значения на область выходного нечеткого множества нечеткой системы при использовании логического подхода к нечеткому выводу. При логическом подходе функция принадлежности выходного нечеткого множества формируется как результат пересечения (в данном случае операцией \textit{min}), что можно представить как постепенное вырезание области функции принадлежности выходного нечеткого множества. Из рисунка \cref{fig:ns-width-influence-to-out-logical} видно, что при увеличение ширины в пересечении проекций импликации на пространство выходной переменной формируется область пересечения более сложной формы.

\subsection{Адаптивная процедура несинглтонной фаззификации временных последовательностей}

\begin{figure}[tb!]
	\centering
	\begin{minipage}[b]{0.48\textwidth}
		\centering
		\includegraphics[width=\textwidth]{noisy-ts-demo-stable-noise}
		\caption*{а}
	\end{minipage}
	\hfill
	\begin{minipage}[b]{0.48\textwidth}
		\centering
		\includegraphics[width=\textwidth]{noisy-ts-demo-mixed-stable-noise}
		\caption*{б}
	\end{minipage}
	\vspace{0.5em}
	
	\begin{minipage}[b]{0.48\textwidth}
		\centering
		\includegraphics[width=\textwidth]{noisy-ts-demo-variable-noise}
		\caption*{в}
	\end{minipage}
	\caption{Примеры временных рядов с различными типами шума: (а) стабильный шум, (б) смешанный стабильный шум, (в) переменный шум. $SNR(dB) = 10 \log_{10} \frac{\sigma^2_s}{\sigma^2_n}$ --- отношение среднеквадратичного значения сигнала $\sigma^2_s$ к среднеквадратичному значению шума $\sigma^2_n$ (ОСШ, \textit{signal-to-noise ratio}) в децибелах. Большее значение $SNR(dB)$ соответствует меньшему уровню шума.}
	\label{fig:noisy-ts-demo}
\end{figure}

В задаче прогнозирования временных рядов измеренные значения могут содержать в себе неопределенность, зачастую возникающую из-за шумового искажения измерений. Такая неопределенность может быть выражена при сопоставлении каждому измеренному значению несинглтонной функции принадлежности нечеткого множества. \textbf{\textit{Концептуально, несинглтонный фаззификатор подразумевает, что входное значение $x_t$ является \underline{наиболее возможным} значением, среди всех значений в её окрестности $\sigma^{est}$ однако, поскольку входные данные неопределенные, соседние точки также, возможны, но в меньшей степени.
По мере удаления от входного значения $x$ возможность того, что оно будет правильным уменьшается. Таким образом, несинглтонные нечеткие множества могут эффективно улавливать входную неопределенность}}, не требуя изменений в других частях нечеткой системы, таких как антецеденты или консеквенты правил. Например, если несинглтонная входная ф. п. нечеткого множества, связанная с данным четким значением $x_0$, выражается гауссовой функцией:
\[
\mu(x) = \exp \left(x; x_0, b\right),
\]
где $b$ --- ширина или стандартное отклонение гауссовой ф. п., которая может быть задана для отражения уровня неопределенности. На практике, когда предполагается, что входные данные системы подвержены низким уровням неопределенности, ширина или значения $b$ соответствующих гауссовских ф. п. интуитивно мала, в то время как большие значения используются для моделирования более высоких уровней неопределенности.

%Pekaslan2020
Специфика оценки шумовой неопределенности временной последовательности измерений возникает поскольку, отражающей какую-то характеристику некоторого процесса, которому характерны различные виды переменчивости силы шумового воздействия. Примеры различных видов изменчивости уровня шума изображены на рис. \cref{fig:noisy-ts-demo}. В \cite{Pekaslan2020} предложена обобщенная процедура \textit{адаптивной оценки неопределенности} для сопоставления значений исходной последовательности с несинглтонными ф. п. входных нечетких множества. Данная процедура состоит из четырех шагов, записанных в блоках блок-схемы на рисунке \cref{fig:ns-procedure}:

\begin{figure}[h]
	\centering
	\includegraphics[width=0.7\textwidth]{ns-procedure}
	\caption{Схема обобщенной процедуры адаптивной несинглтонной фаззификации временной последовательности.}
	\label{fig:ns-procedure}
\end{figure}

\begin{figure}[thb]
	\centering
	\includegraphics[width=0.9\textwidth]{ns-demo-low-noise}
	\caption{Иллюстрация процедуры несинглтонной фаззификации временного ряда с низким уровнем шума.}
	\label{fig:ns-demo-low-noise}
\end{figure}

\begin{enumerate}
	\item \textit{Выбрать размер окна сглаживания:} Определяется размер окрестности точки во временном ряде, по которому выполняется оценка степени неопределенности в точке. Размер окна может быть фиксированным или динамическим (например, при использовании экспоненциально взвешенного скользящего среднего). Например, при использовании датчиков (например, в робототехнике)	размер окна может быть выбран в зависимости от частоты съема датчиков или на основе фиксированного временного интервала. Другой подход, например, в контексте	временных рядов, может заключаться в использовании алгоритма подбора для определения оптимального размера окна.
	\item \textit{Оценить степень неопределенности внутри окна:} Теперь для оценки степени неопределенности в точке используется сформированный набор наблюдений временной последовательности. В зависимости от условий эксперимента, могут использоваться различные техники оценки неопределенности: статистическая (дисперсия), максимальное правдоподобие, глубокие нейросетевые автоэнкодеры, метод главных компонент.% \todo{добавить ссылки}. 
	\item \textit{Задать функцию принадлежности входного нечеткого множества:} Используя оценку степени неопределенности из предыдущего шага, следует задать параметры функции принадлежности входного нечеткого множества, то есть, выполнить непосредственно фаззификацию.
	\item \textit{Сдвинуть окно сглаживания:} Затем следует выполнить оценку степени неопределенности и фаззификацию для точки, следующей за текущей.
\end{enumerate}

На рисунке \cref{fig:ns-demo-low-noise} изображено как каждое несинглтонное входное нечеткое множество задается на основе оценки уровня шума, определенного по описанной только что процедуре с применением скользящего окна сглаживания. На этом рисунке показан участок наблюдений с низким уровнем шума приводящий к узкой ширине (среднеквадратичному отклонению) ф. п. входного нечеткого множества. На рисунке \cref{fig:ns-demo-large-noise} уровень шума заметно больше и, как следствие, ф. п. входного нечеткого множества имеет большую ширину.

\begin{figure}[bth]
	\centering
	\includegraphics[width=0.9\textwidth]{ns-demo-large-noise}
	\caption{Иллюстрация процедуры несинглтонной фаззификации временного ряда с высоким уровнем шума.}
	\label{fig:ns-demo-large-noise}
\end{figure}

\subsection{Выбор схемы дефаззификации.}

Рассмотрим некоторые распространенные методы дефаззификации, описанные в \cite{rutkovskiy2010}:
\begin{enumerate}
	\item \ul{Дефаззификация по методу среднего центра} вычисляется в центрах $\bar{y}_k$ правил, где функция принадлежности $k$-го консеквента $\mu_{B_k}(y_k)$ принимает максимальное значение:
	\begin{equation*}
		\label{eqn:defuz-ca-1}
		\hat{y}_{CA} = \frac{\sum_{k=1}^{N} \bar{y}_k \mu_{B'_k}(\bar{y}_k)}{\sum_{k=1}^{N} \mu_{B'_k}(\bar{y}_k)}.
	\end{equation*}
	\item \ul{Дефаззификация по методу центра тяжести} определяется отношением:
	\begin{equation*}
		\label{eqn:defuz-cog-1}
		\hat{y}_{CoG} = \frac{\int_Y y \mu_{B'}(y) dy}{\int_Y \mu_{B'}(y) dy}.
	\end{equation*}
	Значение $\hat{y}_{CoG}$ в данном способе дефаззификации может быть вычислено с применением численных методов. Однако существует упрощенная схема нахождения выходного значения в данном методе с помощью дискретной формулы центра тяжести в точках центров функций принадлежности термов выходной лингвистической переменной или ф. п. консеквентов правил в базе правил \cite{rutkovskiy2010}. Эта схема выражается формулой ниже:
	\begin{equation}
		\label{eqn:defuz-cog-3}
		\hat{y}_{CoG} = \frac{\sum_{k=1}^{N} \overline{y}_k \mu_{B'}(\overline{y}_k)}{\sum_{k=1}^{N} \mu_{B'}(\overline{y}_k)},
	\end{equation}
	где $\overline{y}_k$ --- центр ф. п. нечеткого множества $B_k$, то есть такое значение $y$, в котором $\max_y \mu_{B_k}(y) = 1$.
	\item \ul{Дефаззификация по методу среднего максимума}
	\begin{equation*}
		\hat{y}_{MeOM} = \frac{\sum_{x \in core(B')} x}{|core(B')|},
	\end{equation*}
	где $core(B') = \left\{y | y \in Y \textrm{ and } \mu_{B'}(y) = \sup_{y' \in Y} \mu_{B'}(y')\right\}$. 
	В случае унимодального вида функции принадлежности $\mu_{B'}(y)$ данный способ дефаззификации можно упростить до метода максимума функции принадлежности:
	\[
	\hat{y}_{MeOM} = \mathrm{arg\,max}_{y \in Y} \mu_{B'}(y).
	\]
\end{enumerate}

В задаче регрессии дискретная формула дефаззификации по центру тяжести или другие более простые схемы (например, метод среднего центра) не показывают достаточной точности, а непрерывная формулировка первой имеет большую вычислительную сложность. Поэтому в работе для регрессии используется дефаззификация по среднему максимуму.

Сетевая структура нечеткой системы для данного способа дефаззификации изображена на рис. \cref{fig:nfs-defuz-meom-with-defuz-demo}.

\begin{figure}[tbh!]
\centering
\includegraphics[width=\linewidth]{nfs-defuz-meom-with-defuz-demo}
\caption{Нейросетевая структура нечеткой модели с несинглтонной фаззификацией и дефаззификацией по среднему максимуму.}
\label{fig:nfs-defuz-meom-with-defuz-demo}
\end{figure}

\section{Альтернативный метод нечеткого вывода с полиномиальной вычислительной сложностью}

\textbf{Вычислительная сложность выражения композиционного правила (\ref{eqn:fuz-problem-4}) для вывода логического типа при использовании non-singleton фаззификации составляет ${O(|X_1|\cdot |X_2|\cdot \dots \cdot |X_n|\cdot |Y|)}$ т.е. экспоненциальная.} Это является актуальным препятствием использования нечеткого вывода логического типа совместно с фаззификацией non-singleton.

\subsection{Нечеткое значение истинности (НЗИ)}

\begin{figure}[ht]
	\centering
	\begin{tikzpicture}[
		]
		\tikzset{
			every pin/.style={
				%fill=yellow!50!white,rectangle,rounded corners=3pt,
				font=\footnotesize, distance=2
			},
			every pin edge/.style={
				line width=0.5pt
			},
			small dot/.style={fill=black,circle,scale=0.5},
		}
		\begin{axis}[
			width=10cm, height=10cm,
			axis lines=middle,
			xmin=-0.0, xmax=1.1, ymin=-0.0, ymax=1.1,
			xlabel={$v$}, xlabel style = {at=(current axis.right of origin), anchor=west},
			ylabel={$\mu(v)$}, ylabel style = {at=(current axis.above origin), anchor=south},
			grid = major,
			clip mode=individual,
			]
			\begin{scope}
				\clip (0,0) rectangle (1,1);
				
				\draw [rotate around={45:(1,0)},line width=2pt] (1,0) ellipse (1.4142135623730951 and 0.816496580927726);
				\draw [rotate around={-45:(0,0)},line width=2pt] (0,0) ellipse (1.4142135623730951 and 0.816496580927726);
				\draw [rotate around={45:(0,1)},line width=2pt] (0,1) ellipse (1.4142135623730951 and 0.816496580927726);
				\draw [rotate around={-45:(1,1)},line width=2pt] (1,1) ellipse (1.4142135623730951 and 0.816496580927726);
				\draw [line width=2pt] plot(\x,{(-0--1*\x)/1});
				\draw [line width=2pt] plot(\x,{(--1-1*\x)/1});
				\draw [line width=2pt] plot(\x,{(--1-0*\x)/1});
				\draw [line width=2pt] plot(\x,{(-0-0*\x)/1});
				\draw [line width=2pt] (1,0) -- (1,1);
				\draw [line width=2pt] (0,0) -- (0,1);
				
				\coordinate (smalllie) at (0.207750076598, 0.8798067848812);
				\coordinate (lie) at (0.2,0.8);
				\coordinate (largelie) at (0.1776022108245, 0.7092399162497);
				
				\coordinate (smalltruth) at (0.7627014759205, 0.8600064781734);
				\coordinate (truth) at (0.775, 0.775);
				\coordinate (largetruth) at (0.8049494415761, 0.6855072896032);
				
				\coordinate (quazilie) at (0.25,0);
				\coordinate (quazitruth) at (0.25,1);
				\coordinate (absolutelie) at (0,0.5);
				\coordinate (absolutetruth) at (1,0.45);
			\end{scope}
			
			
			\node [small dot,pin={[pin distance=3cm]185:Слегка ложно}] at (smalllie) {};
			\node [small dot,pin={[pin distance=3cm]185:Ложно}] at (lie) {};
			\node [small dot,pin={[pin distance=3cm]185:Очень ложно}] at (largelie) {};
			
			\node [small dot,pin={[pin distance=3cm]-10:Слегка истинно}] at (smalltruth) {};
			\node [small dot,pin={[pin distance=3cm]-10:Истинно}] at (truth) {};
			\node [small dot,pin={[pin distance=3cm]-10:Очень истинно}] at (largetruth) {};
			
			\node [small dot,pin={[pin distance=0.8cm]40:Квазиистинно}] at (quazitruth) {};
			\node [small dot,pin={[pin distance=0.8cm]-60:Квазиложно}] at (quazilie) {};
			\node [small dot,pin={[pin distance=1cm]-10:Абсолютно истинно}] at (absolutetruth) {};
			\node [small dot,pin={[pin distance=1cm]185:Абсолютно ложно}] at (absolutelie) {};
			
			
			% Draw the object
			%\node[draw, circle, fill=red!20] (B) at (0.2,0.2) {B};
			
			% Define the position for the description
			% \node[anchor=east] (description) at (0.2,0.5) {This is the description of object B};
			
			% Draw a polyline with arrows and dashed style
			%\draw[->, dashed, thick] (B) -- ++(-0.1,0.0) -- ++(0,0.1) -- (description);
			
			
			% Draw the object
			%\node[draw, ellipse, fill=green!20] (C) at (0.9,0.9) {C};
			
			% Define the position for the description
			%\node[anchor=north] (description) at (0.5,0.6) {This is the description of object C};
			
			% Draw a polyline using edge
			%\path (C) edge[->, out=270, in=90] (description);
		\end{axis}
	\end{tikzpicture}
	\caption{Значения лингвистической переменной «истинность»}
	\label{fig:ftv-all-cases}
\end{figure}

В рамках нечеткой логики лингвистическая переменная <<истинность>> расширяет классическую бинарную модель истинности, позволяя выражать градуированные оценки. Возможные значения ее термов включают значения <<истинно>>, <<ложно>>, <<очень истинно>>, <<очень ложно>>, <<слегка истинно>>, <<слегка ложно>>, <<абсолютно истинно>>, <<абсолютно ложно>> и др. Данная концепция позволяет строить связи, содержащие качественные зависимости, в цепочках знаний в экспертных системах и системах поддержки принятия решений.

На рисунке \cref{fig:ftv-all-cases} изображены функции принадлежности термов лингвистической переменной <<истинность>>. Этим функциям принадлежности соответствуют следующие аналитические способы задания:

\begin{align*}
	M[\flqq\text{истинно}\frqq]         &= \int_0^1 v/v;                & M[\flqq\text{ложно}\frqq]         &= \int_0^1 1-v/v; \\
	M[\flqq\text{слегка истинно}\frqq]   &= \int_0^1 \sqrt{v}/v;           & M[\flqq\text{слегка ложно}\frqq]   &= \int_0^1 \sqrt{1-v}/v; \\
	M[\flqq\text{очень истинно}\frqq]      &= \int_0^1 v^2/v;               & M[\flqq\text{очень ложно}\frqq]    &= \int_0^1 \dfrac{(1-v)^2}{v}; \\
	M[\flqq\text{абсолютно истинно}\frqq] & = \dfrac{1}{1} + \int_0^1 \dfrac{0}{v}; & M[\flqq\text{абсолютно ложно}\frqq] & = \dfrac{1}{0} + \int_0^1 \dfrac{0}{v}v; \\
	M[\flqq\text{квазиистинно}\frqq]     &= \int_0^1 1/v;                & M[\flqq\text{квазиложно}\frqq]     &= \int_0^1 0/v.
\end{align*}

В данной работе лингвистическая переменная <<истинность>> предлагается использовать для нечеткой оценки истинности одних нечетких высказываний относительно других. Значения этих высказываний формализуются нечеткими множествами $A$ и $A'$, определенными на базовом множестве $X$. Здесь высказывание соответствующее нечеткому множеству $A'$ рассматривается как достоверная, относительно которого оценивается истинность высказывания заданного нечетким множеством $A$.

\textbf{Определение.} Нечеткой истинностью множества $A$ относительно нечеткого множества $A'$ называется нечеткое множество $CP(A,A')$ такое, что:

\begin{equation}
	\label{eqn:ftv-definition}
	\mu_{CP(A, A')}(v) = \sup_{\substack{\mu_{A}(x) = v \\ x \in X}}\left\{\mu_{A'}(x)\right\}.
\end{equation}

\begin{figure}[ht]
	\centering
	\begin{tikzpicture}[
		scale=5,
		%width=15cm, height=15cm,
		]
		% Draw axes
		\draw[->] (0, 0) -- (0, 1) node[above] {$\mu_{CP(A, A')}(v)$};
		\draw[->] (0, 0) -- (1, 0) node[right] {$v = \mu_{A}(x)$};
		\draw[->] (0, 0) -- (0,-1) node[below] {$x$};
		\draw[->] (0, 0) -- (-1, 0) node[left] {$\mu_{A'}(x)$};
		
		\draw[dashed] (0.3, 0.01) -- (0.3, -0.18054861091112354) -| (-0.0779857041069743, -0.18054861091112354);
		\draw[dashed] (0.3, 0.01) -- (0.3, -0.6194513890888765) -| (-0.6999713601259281, -0.6194513890888765) -| (-0.6999713601259281, 0.6999713601259281) -| (0.3, 0.6999713601259281);
		
		\node [lime] at (-0.0779857041069743, -0.18054861091112354) {$\mathbf{\times}$};
		\node [red] at (-0.6999713601259281, -0.6194513890888765) {$\mathbf{\times}$};
		\node [red] at (0.3, 0.6999713601259281) {$\mathbf{\times}$};
		
		% Draw Gaussian function
		%\draw[blue, thick, domain=0.01:1, samples=100, smooth] plot (\x, {exp(-((((0.4-0.5) - 0.2*sqrt(-ln(\x)))/0.2)^2)});
		\draw[blue, thick, domain=0.01:1, samples=100, smooth] plot (\x, {exp(-((((0.4-0.5) + 0.2*sqrt(-ln(\x)))/0.2)^2)});
		\draw[blue, thick, domain=0:1, samples=100, smooth] plot ({exp(-((\x-0.4)/0.2)^2)}, -\x);
		\draw[blue, thick, domain=0:1, samples=100, smooth] plot ({-exp(-((\x-0.5)/0.2)^2)}, -\x);
		\draw[blue, thin, dotted] (0, 0) -- (-1, 1);
		
	\end{tikzpicture}
	\caption{Пример вычисления нечеткого значения истинности}
	\label{fig:ftv-computation}
\end{figure}

Процедура вычисления истинности одного нечеткого множества относительно другого, согласно формуле (\ref{eqn:ftv-definition}), отражена на рисунке \cref{fig:ftv-computation}.

Понятие \textit{относительной истинности} достоверного высказывания $A'$ относительно оцениваемого высказывания $A$ опирается на несколько аксиом:
\begin{itemize}
	\item \textit{Аксиома истинности.} Нечеткое значение истинности ИСТИННО задается нечетким множеством:
	\begin{equation*} 
		CP(A,A') = \left\{\langle\mu_{CP(A,A')}(v), v\rangle\right\} = \left\{v/v\right\}, v \in [0; 1],
	\end{equation*}
	что выполняется тогда и только тогда, когда $A$ относительно соответствует $A'$, т. е. функции принадлежности нечетких множеств $A'$ и $A$ совпадают.
	
	\pgfplotsset{
		membership axes/.style={
			%width=0.45\textwidth,
			%height=0.45\textwidth,
			scale only axis,
			domain=0:1,
			samples=100,
			every axis plot/.append style={smooth},
			axis lines=middle,
			xmin=0, xmax=1,
			ymin=0, ymax=1,
			xticklabel style={font=\tiny, inner sep=1pt, outer sep=0pt},
			yticklabel style={font=\tiny, inner sep=1pt, outer sep=0pt},
			legend style={font=\small},
		}
	}
	
	\begin{figure}[ht]
		\newcommand{\aOne}{0.5}
		\newcommand{\bOne}{0.05}
		\newcommand{\aTwo}{0.5}
		\newcommand{\bTwo}{0.05}
		\begin{subfigure}[t]{0.5\textwidth}
			\begin{tikzpicture}
				\begin{axis}[
					membership axes,
					]
					\addplot [black, thick] {gauss(x, \aOne, \bOne)};
					\addplot [blue, thick] {gauss(x, \aTwo, \bTwo)};
				\end{axis}
			\end{tikzpicture}
			\caption{$\mu_A(x; a_1, b_2)$ = $\mu_{A'}(x; a_2, b_2)$ при $a_1 = a_2, b_1 = b_2$}
		\end{subfigure}
		\hfill
		\begin{subfigure}[t]{0.5\textwidth}
			\begin{tikzpicture}
				\begin{axis}[
					membership axes,
					]
					\addplot [blue] {x};
					%{max(gauss(\aOne-2*\bOne*sqrt(-ln(x)), \aTwo, \bTwo), gauss(\aOne+2*\bOne*sqrt(-ln(x)), \aTwo, \bTwo))};
				\end{axis}
			\end{tikzpicture}
			\caption{$\mu_{CP(A,A')}(v)$ при $a_1 = a_2$ и $b_1 = b_2$}
		\end{subfigure}
		\caption{Иллюстрация случая истинного отношения высказываний}
		\label{fig:ftv-gauss-true}
	\end{figure}
	
	На рис. \cref{fig:ftv-gauss-true} представлены графики совпадающих функций принадлежности высказываний и построенной функции принадлежности нечеткого значения истинности.
	
	\item \textit{Аксиома ложности.} Нечеткое значение истинности ЛОЖНО задается нечетким множеством:
	\begin{equation*} 
		CP(A,A') = \left\{\langle\mu_{CP(A,A')}(v), v\rangle\right\} = \left\{(1-v)/v\right\} = \left\{v/(1-v)\right\}, v \in [0; 1],
	\end{equation*}
	что выполняется тогда и только тогда, когда утверждаемое высказывание $A$ противоположно утверждаемому в $A'$, т. е. функции принадлежности высказываний $A'$ и $A$ удовлетворяют одному из условий:
	\[
	\mu_{A'}(x) = 1 - \mu_{A}(x)
	\]
	или
	\[
	\mu_{A'}(x) = \left\{
	\begin{alignedat}{2}
		1 - \mu_{A}(x), &\quad x \le x_{max} \\
		0, &\quad x > x_{max}
	\end{alignedat}
	\right.
	\]
	или
	\[
	\mu_{A'}(x) = \left\{
	\begin{alignedat}{2}
		0, &\quad x < x_{max} \\
		1 - \mu_{A}(x), &\quad x \ge x_{max},
	\end{alignedat}
	\right.
	\]
	где $x_{max} = \textrm{arg\,max}_x \mu_{A}(x)$.
	
	\begin{figure}[ht]
		\newcommand{\aOne}{0.5}
		\newcommand{\bOne}{0.05}
		\newcommand{\aTwo}{0.5}
		\newcommand{\bTwo}{0.05}
		\begin{subfigure}[t]{0.5\textwidth}
			\begin{tikzpicture}
				\begin{axis}[
					membership axes,
					]
					\addplot [black, thick] {gauss(x, \aOne, \bOne)};
					\addlegendentry{$\mu_A(x; a_1, b_1)$};
					\addplot [blue, thick] {1-gauss(x, \aTwo, \bTwo)};
					\addlegendentry{$\mu_{A'}(x; a_2, b_2)$};
				\end{axis}
			\end{tikzpicture}
			\caption{$\mu_{A'}(x; a_2, b_2) = 1 - \mu_A(x; a_1, b_1)$ при $a_1 = a_2, b_1 = b_2$}
		\end{subfigure}
		\begin{subfigure}[t]{0.5\textwidth}
			\begin{tikzpicture}
				\begin{axis}[
					membership axes,
					]
					\draw [blue] (0,1) -- (1,0);
					\addlegendimage{blue, no marks, line legend}
					\addlegendentry{$\mu_{CP(A,A')}$}
					%\addplot [blue, thick] {max(gauss(\aOne-\bOne*sqrt(-ln(x)), \aTwo, \bTwo), gauss(\aOne+\bOne*sqrt(-ln(x)), \aTwo, \bTwo))};
				\end{axis}
			\end{tikzpicture}
			\caption{$\mu_{CP(A,A')}(v)$ при $\mu_{A'}(x) = 1 - \mu_A(x)$}
		\end{subfigure}
		\caption{Иллюстрация случая ложного отношения высказываний}
		\label{fig:ftv-gauss-false}
	\end{figure}
	
	На рис. \cref{fig:ftv-gauss-false} представлены графики противоположных по значению функций принадлежности высказываний $A'$ и $A$ и построенной функции принадлежности нечеткого значения истинности.
	\item \textit{Аксиома абсолютной истинности.}  Значение истинности АБСОЛЮТНО ИСТИННО задается нечетким множеством:
	\begin{equation*}
		CP(A, A') = \left\{\langle\mu_{CP(A, A')}(v), v\rangle\right\} = \left\{v/1\right\} = \left\{1/1\right\}, v \in [0, 1],
	\end{equation*}
	что выполняется тогда и только тогда, когда $A'$ абсолютно соответствует $A$, то есть в случае когда оценка данная в высказываниях $A'$ и $A$ является четкой или нечеткой, когда носитель высказывания $A'$ включен в носитель высказывания $A$.
	
	\begin{figure}[ht]
		\newcommand{\aOne}{0.5}
		\newcommand{\bOne}{0.05}
		\newcommand{\aTwo}{0.5}
		\newcommand{\bTwo}{0.05}
		\begin{subfigure}[t]{0.5\textwidth}
			\begin{tikzpicture}
				\begin{axis}[
					membership axes,
					]
					\addplot [black, thick] {gauss(x, \aOne, \bOne)};
					\addlegendentry{$\mu_A(x)$}
					%\addplot [blue, thick] {gauss(x, \aTwo, \bTwo)};
					\draw [blue, thick] (0,0) -- (\aTwo,0);
					\draw [blue, thick] (\aTwo,0) -- (1,0);
					\draw [blue, thick] (\aTwo,0) -- (\aTwo,1);
					\addlegendimage{blue, thick, no marks, line legend}
					\addlegendentry{$\mu_{A'}(x)$}
				\end{axis}
			\end{tikzpicture}
			\caption{$\mu_A(x; a_1, b_1)$ и $\mu_{A'}(x; a_2, b_2)$ при $a_1 = a_2, b_1 \gg b_2$}
		\end{subfigure}
		\begin{subfigure}[t]{0.5\textwidth}
			\begin{tikzpicture}
				\begin{axis}[
					membership axes,
					]
					\draw [blue] (0,0) -- (1,0);
					\draw [blue] (1,0) -- (1,1);
					\addlegendimage{blue, thick, no marks, line legend}
					\addlegendentry{$\mu_{CP(A,A')}(v)$}
					%{max(gauss(\aOne-\bOne*sqrt(-ln(x)), \aTwo, \bTwo), gauss(\aOne+\bOne*sqrt(-ln(x)), \aTwo, \bTwo))};
				\end{axis}
			\end{tikzpicture}
			\caption{$\mu_{CP(A,A')}(v)$ при $a_1 = a_2, b_1 \gg b_2$}
		\end{subfigure}
		\caption{Иллюстрация случая абсолютно истинного отношения высказываний}
		\label{fig:ftv-gauss-absolute-true}
	\end{figure}
	
	На рис. \ref{fig:ftv-gauss-absolute-true} представлены графики функций принадлежности высказывания $A'$, включенного в $A$ и функции принадлежности нечеткого значения истинности, соответствующие данной ситуации. Для моделирования четкого значения функции принадлежности (синглтона) взята гауссова функция кривая с дисперсией, стремящейся к нулю.
	
	\item \textit{Аксиома абсолютной ложности.}  Значение истинности АБСОЛЮТНО ЛОЖНО задается нечетким множеством:
	\begin{equation*}
		CP(A, A') = \left\{\langle\mu_{CP(A, A')}(v), v\rangle\right\} = \left\{v/0\right\} = \left\{1/0\right\}, v \in [0, 1],
	\end{equation*}
	что выполняется тогда и только тогда, когда $A'$ абсолютно не соответствует $A$, то есть в случае когда оценки данные в высказываниях $A'$ и $A$ имеют несовпадающие носители.
	
	\begin{figure}[ht]
		\newcommand{\aOne}{0.3}
		\newcommand{\bOne}{0.05}
		\newcommand{\aTwo}{0.67}
		\newcommand{\bTwo}{0.05}
		\begin{subfigure}[t]{0.5\textwidth}
			\begin{tikzpicture}
				\begin{axis}[
					membership axes,
					]
					\addplot [black, thick] {gauss(x, \aOne, \bOne)};
					\addlegendentry{$\mu_A(x)$}
					\addplot [blue, thick] {gauss(x, \aTwo, \bTwo)};
					\addlegendentry{$\mu_{A'}(x)$}
				\end{axis}
			\end{tikzpicture}
			\caption{$\mu_A(x; a_1, b_1)$ и $\mu_{A'}(x; a_2, b_2)$ при $|a_1 - a_2| \gg 0, b_1 \approx b_2$}
		\end{subfigure}
		\begin{subfigure}[t]{0.5\textwidth}
			\begin{tikzpicture}
				\begin{axis}[
					membership axes,
					samples=1000,
					]
					\addplot [blue, thick] {gauss(\aOne+2*\bOne*sqrt(-ln(x)), \aTwo, \bTwo)};
					\addlegendentry{$\mu_{CP(A,A')}(x)$}
				\end{axis}
			\end{tikzpicture}
			\caption{$\mu_{CP(A,A')}(v)$ при $|a_1 - a_2| \gg 0, b_1 \approx b_2$}
		\end{subfigure}
		\caption{Иллюстрация случая абсолютно ложного отношения высказываний}
		\label{fig:ftv-gauss-absolute-false}
	\end{figure}
	
	На рис. \ref{fig:ftv-gauss-absolute-false} представлены графики непересекающихся гауссовых функций принадлежности высказываний $A'$ и $A$ с удаленными центрами и построенная для этого случая функция принадлежности нечеткого значения истинности.
	
	\item \textit{Аксиома квазиистинности.} Нечеткое значение истинности КВАЗИИСТИННО задается нечетким множеством:
	\begin{equation*} 
		CP(A,A') = \left\{\langle\mu_{CP(A,A')}(v), v\rangle\right\} = \left\{1/v\right\}, v \in [0; 1],
	\end{equation*}
	что выполняется тогда и только тогда, когда утверждение в высказывании $A'$ является частным по отношению к $A$, то есть оценка данная в высказывании $A$ интервальная и совпадает с носителем высказывания $A$.
	
	\begin{figure}[ht]
		\newcommand{\aOne}{0.5}
		\newcommand{\bOne}{0.05}
		\newcommand{\aTwo}{0.5}
		\newcommand{\bTwo}{0.05}
		\begin{subfigure}[t]{0.5\textwidth}
			\begin{tikzpicture}
				\begin{axis}[
					membership axes,
					]
					\addplot [black, thick] {gauss(x, \aOne, \bOne)};
					\addlegendentry{$\mu_A(x)$}
					\draw [blue, thick] (0,1) -- (1,1);
					\addlegendimage{blue, thick, no marks, line legend}
					\addlegendentry{$\mu_{A'}(x)$}
				\end{axis}
			\end{tikzpicture}
			\caption{$\mu_A(x; a_1, b_1)$ и $\mu_{A'}(x; a_2, b_2)$ при $a_1 = a_2, b_1 \ll b_2$}
		\end{subfigure}
		\begin{subfigure}[t]{0.5\textwidth}
			\begin{tikzpicture}
				\begin{axis}[
					membership axes,
					]
					\draw [blue] (0,1) -- (1,1);
					\addlegendimage{blue, no marks, line legend}
					\addlegendentry{$\mu_{CP(A,A')}(v)$}
					%\addplot [blue, thick] {max(gauss(\aOne-\bOne*sqrt(-ln(x)), \aTwo, \bTwo), gauss(\aOne+\bOne*sqrt(-ln(x)), \aTwo, \bTwo))};
				\end{axis}
			\end{tikzpicture}
			\caption{$\mu_{CP(A,A')}(v)$ при $a_1 = a_2, b_1 \ll b_2$}
		\end{subfigure}
		\caption{Иллюстрация случая кваиистинного отношения высказываний}
		\label{fig:ftv-gauss-quazi-true}
	\end{figure}
	
	На рис. \cref{fig:ftv-gauss-quazi-true} представлены графики функций принадлежности нечеткого множества $A$, полностью содержащегося в нечетком множестве $A'$, и рассчитанной функции принадлежности нечеткого значения истинности.
	
	\item \textit{Аксиома квазиложности.} Нечеткое значение истинности КВАЗИЛОЖНО задается нечетким множеством:
	\begin{equation*} 
		CP(A,A') = \left\{\langle\mu_{CP(A,A')}(v), v\rangle\right\} = \left\{0/v\right\}, v \in [0; 1],
	\end{equation*}
	что справедливо тогда и только тогда, когда утверждаемое в $A'$ не имеет реального подтверждения в действительности. Иными словами, отсутствует возможность установления истинности высказывания $A'$, так как не определено, существует ли в действительности то, что утверждается в $A'$.
	
	Данную ситуацию нельзя выразить математически и изобразить, поскольку истинность функция принадлежности высказывания $A'$ не может быть оценена.
\end{itemize}

\subsection{Вычисление нечеткого значения истинности, когда функции принадлежности высказываний задаются гауссовыми функциями}

Пусть функции принадлежности нечетких множеств высказываний $A$ и $A'$ заданы разновидностью гауссовых функций:
\begin{equation*}
	\mu_{A}(x; a, b) = e^{-\frac{(x-a)^2}{2b^2}} \quad \mu_{A'}(x; c, d) = e^{-\frac{(x-c)^2}{2d^2}}.
\end{equation*}

Тогда, согласно формуле нечеткого значения истинности (\ref{eqn:ftv-definition}), для вычисления НЗИ в точке $v_0$ необходимо сперва найти все точки из области определения функции принадлежности факта, в которых он принимает значение $v_0$. В случае с гауссовой функцией это можно сделать, с помощью обратной гауссовой функции:
\begin{equation*}
	x(v) = a \pm b\sqrt{-2\ln{v}},
\end{equation*}
тогда
\begin{align}
	\mu_{CP(A, A')}(v) &= \max\left\{e^{-\frac{((a - b\sqrt{-2\ln v})-c)^2}{2 d^2}},e^{-\frac{((a + b\sqrt{-2\ln v})-c)^2}{2 d^2}}\right\} \nonumber \\
	&= \max\left\{e^{-\frac{((a-c) - b\sqrt{-2\ln v})^2}{2 d^2}},e^{-\frac{((a-c) + b\sqrt{-2\ln v})^2}{2 d^2}}\right\} \label{eqn:ftv-gauss-expanded}
\end{align}

\subsection{Метод вывода на основе НЗИ}

Используя правило истинностной модификации [1] можно выразить:
\[
\mu_{A'}(\mathbf{x}) = \tau_{A|A'}(\mu_A(x))
\]
где $\tau_{A|A'}$ "--- нечеткое значение истинности (НЗИ) нечеткого множества $A$ относительно $A'$, представляющее собой функцию принадлежности совместимости $CP(A_k, A')$ $A_k$ по отношению к $A'$, причем $A'$ рассматривается как достоверное: %[Дюбуа и др., 1990]
\begin{equation}
	\label{eqn:ftv-compute-12}
	\tau_{A_k|A'}(v) = \mu_{CP(A_k, A')}(v) = \sup_{\substack{\mu_{A_k}(x) = v \\ x \in X}} \left\{ \mu_{A'}(x)\right\}.
\end{equation}
Таким образом НЗИ отражает совместимость факта с посылкой в нечеткой форме.

Перейдем от переменной $x$ к переменной $v$ в выражении композиционного правила вывода (\ref{eqn:fuz-problem-4}), обозначив
\[
\mu_{A_k}(x) = v \textrm{ и } \mu_{A'}(x) = \tau_{A_k|A'}(v).
\]

Тогда для нечеткой системы с одним входом истинностное преобразование позволяет выполнить переход к новому виду выражения (\ref{eqn:fuz-problem-4}):
\begin{equation}
	\label{eqn:ftv-compute-5}
	\mu_{B'_k}(y|\mathbf{x'}) = \sup_{v \in [0,1]}\left\{\tau_{A_k|A'}(v) \overset{T_2}{\star} I(v, \mu_{B_k}(y))\right\}.
\end{equation}

Это соответствует новой структуре правил в базе правил, отличную от канонических структур Заде и Такаги-Сугено:
\begin{equation}
	\text{Если } \textit{нзи} \text{ есть } \text{ИСТИННО}, \text{ то }\ y\ \text{есть}\ B'_k.
	\label{eqn:ftv-compute-13}
\end{equation}

При переходе к нечеткому выводу по $n$ входам формула вычисления НЗИ для нечетких отношений посылки и факта имеет вид:
\begin{equation*}
	\tau_{\mathbf{A_k}|\mathbf{A'}}(v) = \sup_{\substack{\mu_{\mathbf{A_k}}(x_1, \dots, x_n) = v \\ (x_1, \dots, x_n) \in \mathbf{x}}} \left\{\mu_{\mathbf{A'}}(x_1, \dots, x_n)\right\} .
\end{equation*}

Или в выражении через операции сверток $t$-норм $T_1$ (\ref{eqn:fuz-problem-2}) и $T_3$ (\ref{eqn:fuz-problem-3}):
\begin{equation}
	\label{eqn:ftv-compute-6}
	\tau_{\mathbf{A_k}|\mathbf{A'}}(v) = \sup_{\substack{\underset{i=\overline{1,n}}{\mathrm{T_1}}\mu_{A_{ki}}(x_i)=v \\ (x_1, \dots, x_n) \in \mathbf{x}}} \left\{ \underset{i=\overline{1,n}}{\mathrm{T_3}} \mu_{A'_i}(x_i) \right\}.
\end{equation}

Вместо выражения (\ref{eqn:ftv-compute-6}), НЗИ для $n$ входов может быть вычислено как свертка НЗИ по каждому отдельному входу:
\begin{equation}
	\label{eqn:ftv-compute-7}
	\tau_{\mathbf{A_k}|\mathbf{A'}}(v) = \underset{i=\overline{1,n}}{\mathrm{\tilde{T}}} \tau_{A_{ki}|A'_i}, v \in [0, 1],
\end{equation}
где $\mathrm{\tilde{T}}$ - расширенная по принципу обобщения $n$-местная $t$-норма \cite{kutsenko2015methods}, которая определяется как
\begin{equation}
	\label{eqn:ftv-compute-8}
	\underset{i=\overline{1,n}}{\mathrm{\tilde{T}}} \tau_{A_{ki}|A'_i}(v) = \sup_{\substack{\underset{i=\overline{1,n}}{\mathrm{T_1}}v_i = v \\ (v_1, \dots, v_n) \in [0, 1]^n}} \left\{\underset{i=\overline{1,n}}{\mathrm{T_3}}\tau_{A_{ki}|A'_i}(v_i)\right\}.
\end{equation}

Для $n=2$ $\mathrm{\tilde{T}}$ записывается как:
\begin{equation}
\underset{i=\overline{1,2}}{\mathrm{\tilde{T}}} \tau_{A_{ki}|A'_i}(v) = \sup_{\substack{v_1 \mathrm{ T_1 } v_2 = v \\ v_1, v_2 \in [0, 1]}} \left\{ \tau_{A_{k1}|A'_1}(v_1) \mathrm{ T_3 } \tau_{A_{k2}|A'_2}(v_2) \right\}, v \in [0,1].
\label{eqn:ftv-compute-11}
\end{equation}

Рекурсивная схема вычисления свертки НЗИ по формуле (\ref{eqn:ftv-compute-8}) иллюстрируется выражением:
\begin{align}
	\label{eqn:ftv-compute-10}
	\tau_{\mathbf{A_k}|\mathbf{A'}}(v) & = \underset{i=\overline{1,n}}{\mathrm{\tilde{T}_1}}\tau_{A_{ki}|A'_i}(v_i) \\
	& = \left(\dots\left(\left(\mu_{CP(A_{k1}, A'_1)}(v_1)\ \mathrm{\tilde{T}_1}\ \mu_{CP(A_{k2}, A'_2)}(v_2)\right)\ \mathrm{\tilde{T}_1}\ \dots \right) \mathrm{\tilde{T}_1}\ \mu_{CP(A_{kn}, A'_n)}\right).
\end{align}

Тогда для системы с $n$ входами выражения нечеткого вывода на основе НЗИ (\ref{eqn:ftv-compute-5}) примет вид:
\begin{equation}
	\label{eqn:ftv-compute-11}
	\mu_{B'_k}(y|\mathbf{x'}) = \sup_{v \in [0, 1]} \left\{\tau_{\mathbf{A_k}|\mathbf{A'}}(v) \overset{\mathrm{T_2}}{\star} I(v, \mu_{B_k}(y))\right\}.
\end{equation}

\begin{figure}[tbh!]
	\centering
	\includegraphics[width=\linewidth]{ftv-schema-comparizon}
	\caption{Сравнение классической схемы нечеткого вывода и схемы нечеткого вывода на основе НЗИ}
	\label{fig:ftv-schema-comparizon}
\end{figure}

В формуле (\ref{eqn:ftv-compute-11}) данный подход позволяет переместить процесс вывода в единое пространство НЗИ, где функции истинности, в отличии от различных пространств в подходе Заде, могут быть объединены в более эффективный вычислительный процесс.

Тогда, поскольку применение нового вида правила не зависит от количества входов в нечеткой системе, порядок функции временной сложности вычисления $B'_k$ на основе выражения (\ref{eqn:ftv-compute-11}) составляет $O\left(n|V|^2+|V|\cdot |Y|\right)$, где $V=[0;1]$. Сравнение схем нечетких выводов в соответствии с соотношениями (\ref{eqn:fuz-problem-4}) и (\ref{eqn:ftv-compute-11}) представлены на рис. \cref{fig:ftv-schema-comparizon}.

\subsection{Вывод логического типа на основе нечеткого значения истинности}

Для использованной в работе дефаззификации по среднему максимуму вывод на основе нечеткого значения истинности выражается формулами:

\begin{align}
	\tau_{A_k|A'}(v) = \underset{i=\overline{1,n}}{\mathrm{\tilde{T}}} \left\{\mu_{CP(A_{ki}, A'_{t-p+i})}(v_i)\right\}, k=\overline{1,N},\\
	\hat{y}_{t+h} = \argmax_{y\in\mathbb{Y}} \left\{\Tnorm_{k=1}^N \left\{\sup_{v\in [0, 1]} \left\{\tau_{\mathbf{A_k}|\mathbf{A'}}(v)\overset{\mathrm{T}_2}{\star} I\left(v, \mu_{A_{k\,p+1}}(y)\right)\right\}\right\}\right\}.
\end{align}

Данным формулам соответствует сетевая структура нейро-нечеткой системы, изображенная на рисунке \cref{fig:nfs-ftv-and-defuz-meom-with-defuz-demo}.

\begin{figure}[thb]
	\centering
	\includegraphics[width=\linewidth]{nfs-ftv-and-defuz-meom-with-defuz-demo}
	\caption{Схема нейро-нечеткой системы с вычислением НЗИ и дефаззификацией по среднему максимуму, а также пример работы процедуры дефаззификации.}
	\label{fig:nfs-ftv-and-defuz-meom-with-defuz-demo}
\end{figure}

\begin{figure}[ht]
	\centering
	\includegraphics[]{out-mf-with-low-crossing}
	\caption{Пример нечетких множеств, удовлетворяющих условию $\mu_{B_k}(y_r) = 0$ для $y \ne r$}
	\label{fig:out-mf-with-low-crossing}
\end{figure}

Как показано в \cite{Karatach2024}, одна из возможностей упрощения процедуры вывода возникает, когда функции принадлежностей термов выходной лингвистической переменной достаточно удалены друг от друга и имеют низкую степень взаимного пересечения, то есть выполняется соотношение $\mu_{B_k}(y_r) \approx 0$ при $k \ne r$, что проиллюстрировано на рисунке \cref{fig:out-mf-with-low-crossing}.

Рассмотрим вычисление $\tau_{k\,r}$ для различных категорий импликаций:
\begin{itemize}
	\item для \textit{S}-импликации
	\begin{equation*}
		\hat{y}_{CoG} = \frac{\sum_{k=1}^{N} \overline{y}_k \Tnorm_{r=1}^N \left\{\sup_{v\in [0, 1]} \left\{\tau_{A_r|A'}(v)\overset{\mathrm{T}_2}{\star}(1-v)\right\}\right\}}{\sum_{k=1}^{N} \Tnorm_{r=1}^N \left\{\sup_{v\in [0, 1]} \left\{\tau_{A_r|A'}(v)\overset{\mathrm{T}_2}{\star}(1-v)\right\}\right\}},
	\end{equation*}
	\item для \textit{R}-импликации
	\begin{equation*}
		\hat{y}_{CoG} = \frac{\sum_{k=1}^{N} \overline{y}_k \Tnorm_{r=1}^N \left\{\tau_{A_r|A'}(0)\right\}}{\sum_{k=1}^{N} \Tnorm_{r=1}^N \left\{\tau_{A_r|A'}(0)\right\}},
	\end{equation*}
	\item для \textit{Q}-импликации
	\begin{equation*}
		\hat{y}_{CoG} = \frac{
			% Numerator
			\begin{aligned}[t]
				\sum_{k=1}^{N} \overline{y}_k \mathrm{T}_2 \Bigg\{
				& \sup_{v\in [0, 1]} \left\{\tau_{A_k|A'}(v)\overset{\mathrm{T}_2}{\star}\max(1-v, v)\right\}\times \\
				& \times\Tnorm_{\substack{r=1\\r\ne k}}^N \left\{
				\sup_{v\in [0, 1]} \left\{\tau_{A_r|A'}(v)\overset{\mathrm{T}_2}{\star}(1-v)\right\}
				\right\}
				\Bigg\}
			\end{aligned}
		}{
			% Denominator (structure mirrors the numerator)
			\begin{aligned}[t]
				\sum_{k=1}^{N} \mathrm{T}_2 \Bigg\{
				& \sup_{v\in [0, 1]} \left\{\tau_{A_k|A'}(v)\overset{\mathrm{T}_2}{\star}\max(1-v, v)\right\}\times \\
				& \times\Tnorm_{\substack{r=1\\r\ne k}}^N \left\{
				\sup_{v\in [0, 1]} \left\{\tau_{A_r|A'}(v)\overset{\mathrm{T}_2}{\star}(1-v)\right\}
				\right\}
				\Bigg\}
			\end{aligned}
		}.
	\end{equation*}
\end{itemize}

%\todo{Можно организовать вычисление всех значений $b_{k\,r}$ $\tau_{k\,r}$, но использовать разреженные матрицы в качестве структуры данных для хранения значений где $b_{k\,r} > 0$.}

%Идея сокращения вычислений за счет фильтрации неактивированных функций принадлежности справедливы не только для ф. п. консеквентов. Возможно исключить из отдельных термов, а и для набора кластеризованных в небольшие группы функций принадлжености со значительной степенью пересечения. При такой конфигурации выходного нечеткого пространства нет необходимости включать в процесс вывода правила, в которых функции принадлжености консеквента имеют низкий уровень пересечения с ф. п. правил, имеющих высокий уровень срабатывания для данного входа нечеткой системы.}

\subsection{Свойства вывода типа Мамдани на основе нечеткого значения истинности}

В \cite{Sinuk2023} ослаблено ограничение на использование одной и той же $T$-нормы в формуле композиционного правила вывода в работах Менделя для случая вывода типа Мамдани, а также показано, что вывод по отдельному правилу в случае $T_2 = T_4 = T$ может быть записан через \textit{меру возможности}:
\begin{align*}
	\mu_{B'_k}(y) = \sup_{v\in[0;1]}\left\{\tau_{A_{k}|A'}(v) \overset{\mathrm{T_2}}{\star} (v \overset{\mathrm{T_4}}{\star} \mu_{B_k}(y)) \right\} = \textstyle\prod_{\mathbf{A_k}|\mathbf{A'}} \overset{\mathrm{T}}{\star} \mu_{B_k}(y),
\end{align*}
где $\prod_{\mathbf{A_k}|\mathbf{A'}}=\sup_{v\in[0;1]}\left\{\tau_{A_{k}|A'}(v) \overset{\mathrm{T}}{\star} v\right\}$.

Если при этом используется дефаззификация \textit{по центру сумм (CoS)} и $T$-норма Ларсена, то, как доказано в \cite{Sinuk2023}, результат дефаззификации зависит от ширины гауссовой или треугольной функции принадлежности консеквента, тогда как дефаззификация \textit{по среднему центру} учитывает только параметр центра. Например, при использовании в качестве выходной ф. п. гауссовой функции $\mu_{B_k}(y) = exp(-((y-\bar{y}_k)/\sigma_k)^2)$ формула дефаззификации имеет вид: 

\begin{equation}
	\hat{y}_{CoS} = \frac{\int_{\mathbb{Y}} y \sum_{k=1}^{N} \prod_{\mathbf{A_k}|\mathbf{A'}} \overset{\mathrm{T}_2}{\star} \mu_{B_k}(y)}{\int_{\mathbb{Y}} \sum_{k=1}^{N} \prod_{\mathbf{A_k}|\mathbf{A'}} \overset{\mathrm{T}_2}{\star} \mu_{B_k}(y)} = \frac{\sum_{k=1}^{N} \prod_{\mathbf{A_k}|\mathbf{A'}} \bar{y}_k \sigma_k}{\sum_{k=1}^{N} \prod_{\mathbf{A_k}|\mathbf{A'}} \sigma_k},
\end{equation}
поскольку $	\int_{-\inf}^{\inf}\mu_{B_k}(y) dy = \sigma_k \sqrt{\pi}$ и $\int_{-\inf}^{\inf} y \mu_{B_k}(y) dy = \bar{y}_k \sigma_k \sqrt{\pi}$.




\section{Выводы по главе}

\begin{enumerate}
	\item Приведенное описание нечеткого вывода логического типа для нечеткой системы на основе правил в обобщенном виде может быть использована с несинглтонной фаззификацией. Нечеткий вывод при несинглтонной фаззификации позволяет учесть информацию о неопределенности входных данных, но имеет экспоненциальную вычислительную сложность.
	\item Нечеткая модель прогнозирования временных рядов, полученная из нечеткой системы на основе правил, легла в основу разработанного метода прогнозирования, позволяющего адекватно учесть неопределенность значений временных рядов уникальных объектов или процессов.
	\item При использовании несинглтонного способа фаззификации показан эффект активации большего количества правил с ростом неопределенности. Для разработанный метод прогнозирования временных рядов с учетом неопределенности предложено использовать процедуру адаптивной фаззификации и дефаззификацию по среднему максимуму.
	\item Предложенный альтернативный метод нечеткого логического вывода на основе НЗИ обеспечивает полиномиальную вычислительную сложность. В этом методе вывода в процедуре вывода выделяется предварительный этап вычисления НЗИ, которые вычисляются по каждому входу независимо с последующей сверткой в единое истинностное пространство. Метод определяет новую структуру правил вида: <<Если $нзи$ есть \textit{ИСТИННО}, то $y$ есть $B_k$>>.
\end{enumerate}

\FloatBarrier
