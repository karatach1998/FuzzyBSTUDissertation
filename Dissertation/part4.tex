\chapter{Применение разработанной нечеткой модели для прогнозирования временных рядов в задачах экономики и финансов}

Прогнозирование временных рядов в экономике и финансах играет ключевую роль как инструмент анализа и предсказания динамики последовательно изменяющихся данных. Оно служит основой для принятия обоснованных решений в условиях неопределенности, обеспечивая оценку будущих тенденций, рисков и возможностей. Это позволяет субъектам экономической и финансовой деятельности — от индивидуальных инвесторов до государственных структур — оптимизировать управление ресурсами, минимизировать потери и формировать долгосрочные стратегии.

Стратегическое планирование и бюджетирование (обоснование финансовых планов, оптимизация ресурсного распределения), Управление рисками и инвестиционные решения

В государственных и корпоративных бюджетах прогнозы временных рядов используются для оценки объёмов доходов и расходов, планирования денежных потоков и определения ключевых показателей эффективности. У компаний прогнозирование сезонных и трендовых колебаний продаж помогает корректировать производственные мощности, склады и персонал, тем самым минимизируя издержки и снижая риск дефицита или перепроизводства. Прогнозы временных рядов служат основой для разработки моделей оптимального распределения активов и стратегий ребалансировки портфеля в зависимости от ожидаемых рыночных движений.

Одномерные модели оправданы, когда целевой ряд обладает высокой автокорреляцией, слабо зависит или отсутствует информация о значениях внешних факторов, ограниченны вычислительные и временные ресурсы для оценки взаимосвязей измерений. Примеры: краткосрочное прогнозирование инфляции, прогнозирование индекса S\&P 500. Многомерные методы необходимы, когда влияние других временных рядов или внешних предикторов существенно для точности прогноза, доступна качественная информация об этих переменных или когда доступна информация о их будущих значениях, а также когда необходимо исследовать влияние <<шока>> в одной переменной на остальные. Примеры: оценка влияния нефтяных шоков на экономику Нигерии.

Область и ситуации в которой применима разработанная модель (авторегрессионная)

\section{Задача прогнозирования стоимости ценных бумаг}
\section{Задача прогнозирования динамики использования банковского продукта / CLTV по продукту}

\section{Описание набора данных}

\section{Решение задачи с использованием разработанной нечеткой модели}


\section{Выводы}