\chapter{Анализ методов прогнозирования временных рядов в задачах управления с позиции адекватности учета уникальности свойств генерирующих их объектов}\label{ch:ch1}

\section{Роль прогнозирования в задачах управления и принятия решений}

Прогнозирование является фундаментальным элементом информационного обеспечения управленческой деятельности, выступая критически важным инструментом поддержки принятия решений в различных сферах. Его основная функция заключается в научно обоснованном предвидении возможных тенденций развития управляемого объекта или процесса.

Принципиальное значение прогнозирования в управлении состоит в формировании предпосылок для принятия обоснованных решений. На этапе, предшествующем непосредственному планированию, прогнозирование позволяет определить реальность и целесообразность потенциальных стратегических направлений.

Ключевая роль прогнозирования проявляется в нескольких ключевых аспектах:
\begin{itemize}
	\item Научное предвидение: Прогнозирование обеспечивает выявление тенденций и закономерностей развития исследуемых систем, что создает фундамент для стратегического планирования.
	\item Информационная поддержка: Процесс прогнозирования направлен на получение научно обоснованных вариантов изменения управляемого объекта во времени и пространстве.
	\item Антиципация рисков: Через прогнозирование возможно заблаговременное определение потенциальных негативных последствий и рисков при выборе управленческих решений Многовариантность сценариев: Прогнозирование позволяет разрабатывать множественные сценарии развития событий, охватывающие различные потенциальные траектории.
\end{itemize}


Особую значимость прогнозирование приобретает в условиях высокой неопределенности и динамической изменчивости внешней среды. Оно трансформируется из простого инструмента предсказания в сложный механизм поддержки адаптивного управления.

В контексте принятия решений прогнозирование выступает не просто методом экстраполяции существующих тенденций, а сложным аналитическим инструментом, интегрирующим статистические методы, экспертные оценки и современные технологии обработки данных.

Таким образом, прогнозирование является не только вспомогательным, но и системообразующим элементом процесса управления, обеспечивающим научную обоснованность, стратегическую дальновидность и риск-ориентированность принимаемых решений.

\section{Прогнозирование по временным рядам для уникальных объектов}

\ul{Информационная система (ИС)} --- это взаимосвязанная совокупность средств, методов и персонала, используемых для хранения, обработки и выдачи информации для достижения цели управления. Она автоматизирует и поддерживает бизнес-процессы, а также решает конкретные информационные задачи в рамках заданной предметной области.

Информационная система поддержки принятия решений часто использует \textit{информационную технологию} прогнозирования для формирования материала для принятия обоснованных решений.

\begin{figure}[htb]
	\centering
	\includegraphics[width=\linewidth]{information-technology-diagram}
	\caption{Схема системы управления на основе информационной технологии прогнозирования.}
	\label{fig:information-technology-diagram}
\end{figure}

\ul{Информационная технология (ИТ)} --- это совокупность методов, процессов, программного обеспечения, оборудования и сетей, предназначенных для сбора, обработки, хранения, передачи и защиты данных и информации.

Относительно информационной системы, информационные технологии представляют собой конкретные инструменты (аппаратное обеспечение, программное обеспечение, средства коммуникации), с помощью которых строится и работает информационная система, как показано на рисунке \cref{fig:information-technology-diagram}. \textit{Информационная технология прогнозирования} тогда является инструментом прогнозирования в информационной системе на основе поступающей на вход информации.

Информационная технология прогнозирования, как правило, представляет собой программную реализацию некоторого метода прогнозирования, опирающуюся на возможности аппаратной платформы (например, использующую технологии массивно-параллельных вычислений для распараллеливания процедуры прогноза).

\ul{Метод} в машинном обучении --- это алгоритм или способ построения модели, то есть процедура, набор правил или подход, по которым из обучающих данных формируется модель. Метод определяет, как будут анализироваться данные и как будет происходить обучение. Метод прогнозирования использует некоторую \textit{модель} прогнозирования.

\ul{Модель} --- описание явления или процесса, достаточное для достижения цели решаемой задачи. \ul{Модель прогнозирования} --- описание протекающего во времени процесса, достаточное для получения прогноза требуемой точности.

Входные данные такой информационной системы включают \textit{временной ряд}, то есть последовательность значений, составленную посредством фиксации измеренных показателей в определенном временном промежутке, отражающую развитие некоторого процесса или состояния объекта. Объект управления при этом может отличаться от источника генерации временных данных.

Часто производится снятие временных рядов показателей сразу с некоторой совокупности объектов или группы протекающих параллельно процессов. Но в отдельных случаях такие объекты или процессы являются \textit{уникальными}, то есть не имеют ни одного полностью идентичного аналога в изучаемой совокупности. Такие объекты или процессы являются единственными в своем роде или настолько индивидуализированными, что невозможно создать репрезентативную выборку из аналогичных объектов для классического статистического анализа. В контексте прогнозирования временных рядов это означает, что для этих объектов или процессов есть только один доступный временной ряд наблюдений, и невозможно опереться на <<типичные>> свойства популяции.

%Уникальная система — это объект или процесс, для которого отсутствует или практически недостижима репрезентативная статистическая выборка аналогов, и который изучается исключительно на основе своей собственной динамики и свойств.

%На практике нередко встречаются уникальные объекты исследования, для которых невозможно построить ансамбль однотипных реализаций. В таких случаях традиционные ансамблевые методы оказываются неприменимыми.

Далее приведены несколько примеров систем, являющихся уникальными объектами.

\textit{Пример 1. Гидроэлектростанция с уникальным географическим положением.}

Рассмотрим пример временного ряда, описывающего дневной объём воды, проходящей через плотину конкретной гидроэлектростанции (ГЭС).

Поскольку каждая ГЭС имеет уникальное географическое расположение, климатические и гидрологические условия, невозможно найти другое предприятие с идентичными параметрами. 

\textit{Пример 2. Пациент с уникальными физиологическими показателями.}

Ещё более сложный случай — оценка воздействия лекарственного препарата на физиологические показатели конкретного человека.

Если усреднить эффект препарата по множеству пациентов, можно получить усреднённое значение, однако оно не отражает индивидуальную реакцию конкретного организма, поскольку каждый человек имеет свои особенности — генетику, метаболизм, историю болезни, образ жизни.

Более корректным подходом является измерение индивидуального отклика — наблюдение за изменениями показателей при многократном приёме одинаковой дозы лекарства одним пациентом в разные моменты времени. В этом случае формируется индивидуальный временной ряд реакции на препарат.

\textit{Пример 3. Корпоративные клиенты банка с уникальной бизнес-моделью в уникальной экономической среде.}

Корпоративных клиентов банка можно рассматривать как уникальные системы, потому что каждый из них представляет собой сложную, многокомпонентную организацию со своими собственными параметрами, которые формируют уникальный <<профиль>> взаимодействия с банком.

Каждая компания имеет свою структуру управления, юридическую форму, подразделения, процессы принятия решений --- это влияет на то, как она взаимодействует с банком, кто уполномочен проводить операции, как организованы финансовые потоки.

В отличие от физических лиц, компании отличаются видами деятельности, отраслевой спецификой, логистикой и цепочками поставок, сезонностью бизнеса.

\section{Условия адекватности учета неопределенности временных рядов статистическими методами}

В области технических наук понятие \textit{<<адекватность>>} часто используется в контексте моделирования и описывает степень соответствия модели реальному объекту или процессу. \ul{Адекватность модели} машинного обучения (или, в более общем смысле, любой модели) — это степень соответствия модели реальной системе или объекту, который она описывает, с учетом цели моделирования. Это понятие подразумевает, что модель должна совпадать с моделируемой системой в отношении поставленных задач, обеспечивая приемлемую точность и полезность для принятия решений.

%Согласно научному определению, адекватность модели --- это совпадение свойств (функций, параметров, характеристик и т. п.) модели и соответствующих свойств моделируемого объекта. Это подразумевает, что модель адекватна оригиналу, если она верно (в достаточной степени) отражает его свойства и может быть использована для анализа или прогнозирования поведения реального объекта.

В данной работе анализируется адекватность моделей прогнозирования \textit{с позиции адекватности учета неопределенности} временных рядов посредством использования статистических характеристик согласно предусловиям применения вероятностных оценок. Можно выделить следующие условия корректного вычисления статистических характеристик для ансамбля объектов:
\begin{enumerate}
\item \textit{Однородность совокупности наблюдений.} Все элементы ансамбля должны принадлежать одной и той же статистической совокупности, то есть представлять собой реализации одной случайной величины или одного стохастического процесса с неизменными параметрами распределения. Нарушение однородности приводит к несопоставимости выборок и некорректности оценок момента первого и второго порядков.
\item \textit{Статистическая независимость или контролируемая структура зависимости.}
Для корректного применения классических оценок математического ожидания и дисперсии, а также для справедливости стандартных асимптотических результатов (закон больших чисел, центральная предельная теорема), требуется независимость наблюдений. В случаях, когда зависимости неизбежны (например, присутствует автокорреляция), должна быть известна или оценима структура корреляций с последующим учётом её влияния на дисперсию оценок и эффективный размер выборки.
\item \textit{Асимптотическая нормальность оценок.}
Согласно центральной предельной теореме, при достаточно большом числе независимых и одинаково распределённых наблюдений выборочное среднее стремится к нормальному распределению независимо от вида исходного закона. Минимально необходимое значение $n$ определяется свойствами распределения; часто используемое практическое правило $n\ge30$ является лишь приближённой эвристикой.
Для получения оценки математического ожидания с заданной абсолютной погрешностью $\epsilon$ и доверительной вероятностью $1 - \alpha$ используется оценка размера ансамбля
\[
n\approx \left(\frac{z_{1-\alpha/2}\sigma}{\epsilon}\right)^2,
\]
где $\sigma$ --- стандартное отклонение, а $z_{1-\alpha/2}$ --- квантиль стандартного нормального распределения.
\end{enumerate}

При ограниченном объёме ансамбля и отсутствии уверенности в нормальности распределения оценок иногда возможно использовать t-распределение (для приблизительно нормальных данных) либо непараметрические методы, такие как бутстрэп, позволяющие получать доверительные интервалы и характеристики точности без строгих предположений о распределении.

Если условия вычисления статистических характеристик по ансамблю не выполняются, ансамблевое усреднение заменяется временным (как часто делается в статистической физике, климатологии и теории сигналов).

\ul{Эргодичность временного ряда.} Чтобы такая замена была корректной, необходимо, чтобы усреднение по времени было эквивалентно усреднению по ансамблю. Этим свойством обладают \textit{эргодические процессы}.

Процесс называется \textit{эргодическим}, если его временные средние совпадают со статистическими (математическими) средними:
\[
\lim_{T \to \infty} \frac{1}{T} \sum_{t=1}^{T} x_t = M(x_t)
\]
с вероятностью 1.

Иными словами, эргодичность --- это свойство, позволяющее оценивать характеристики всего стохастического процесса по одной достаточно длинной реализации. Иногда это свойство называют \textit{свойством хорошего перемешивания}, подразумевая, что значения ряда, взятые в разные моменты времени, образуют выборку, эквивалентную множеству значений, полученных при повторных экспериментах в один и тот же момент.

Для того чтобы стационарный процесс был эргодичным, достаточно, чтобы автоковариация быстро убывала с увеличением лага, то есть выполнялось условие:
\[
\lim_{n \to \infty} \left( \frac{1}{n} \sum_{k=1}^{n} \gamma(k) \right) = 0,
\]
где $\gamma(k)$ — функция автоковариации.

Формально эргодичность это свойство всего временного процесса. На практике часто говорят о \textit{эргодичности реализации} временного ряда.

\ul{Свойство стационарности временных рядов.} В анализе временных рядов важное значение имеют стационарные процессы, вероятностные свойства которых не изменяются во времени. Именно такие ряды чаще всего применяются для описания случайных (стохастических) составляющих наблюдаемых данных, поскольку позволяют делать устойчивые статистические выводы и строить предсказательные модели.

Временной ряд $x_t, t = 1, 2, \ldots, n$, называется \textit{строго стационарным} (или \textit{стационарным в узком смысле}), если его совместное распределение вероятностей для любых $n$ последовательных наблюдений $x_1, x_2, \ldots, x_n$ совпадает с распределением наблюдений $x_{1+k}, x_{2+k}, \ldots, x_{n+k}$ при любом целом сдвиге $k$. Иными словами, закон распределения и его характеристики не зависят от конкретного момента времени. Следовательно, \textit{математическое ожидание} и \textit{среднее квадратическое отклонение} стационарного ряда постоянны:
\[
M(x_t) = a = \text{const}, \quad \sigma(x_t) = \sigma = \text{const}.
\]

Эти параметры можно оценить по наблюдениям $x_t$ следующим образом:
\[
\bar{x}*t = \frac{1}{n} \sum*{t=1}^{n} x_t,
\]
\[
\sigma_t^2 = \frac{1}{n} \sum_{t=1}^{n} (x_t - \bar{x}_t)^2.
\]

Одним из простейших примеров стационарного временного ряда является **белый шум** — последовательность некоррелированных случайных величин с нулевым математическим ожиданием и постоянной дисперсией. В частности, ошибки $e_t$ в классической линейной регрессионной модели считаются белым шумом, а при нормальном распределении — нормальным (гауссовским) белым шумом.

Стохастический процесс можно рассматривать как функцию двух аргументов: времени и случайности.
Если случайность зафиксирована, мы получаем одну реализацию процесса — конкретную наблюдаемую последовательность.

Однако теоретические характеристики, такие как математическое ожидание $M(x_t)$, определяются как \textit{усреднение по ансамблю} --- то есть по множеству всех возможных реализаций процесса в данный момент времени. На практике же в нашем распоряжении обычно \textit{только одна реализация}, и повторные наблюдения в один и тот же момент времени невозможны.

Поэтому при анализе временных рядов мы вынуждены заменять усреднение по ансамблю **усреднением по времени**, рассчитывая средние значения и другие характеристики на основе одной реализации.

%Стационарные процессы типа $ARMA(p, q)$ удовлетворяют этому условию и, следовательно, являются эргодичными.

Стоит подчеркнуть, что не всякий стационарный процесс является эргодическим, хотя для практических целей наличие стационарности часто подразумевает эргодичность.

Стационарность гарантирует неизменность статистических характеристик во времени. Эргодичность позволяет использовать эти характеристики, вычисленные по одной реализации, как оценки теоретических (ансамблевых) значений.

Таким образом, одной стационарности недостаточно для корректной статистической обработки временных рядов — необходима также эргодичность, обеспечивающая достоверность оценок, полученных на основе наблюдаемых данных.


\section{Анализ адекватности учета неопределенности временных рядов распространенными моделями прогнозирования}

\underline{Линейный предсказатель.} Каноническая форма линейного предсказателя  вычисляет оценку будущего значения как линейную комбинацию прошлых наблюдений. Модель представляет формулой свертки:
\[
\hat x_k = \sum_{i=1}^{p} \beta_i\,x_{k-i},
\]
где \(\beta_1,\dots,\beta_p\) --- коэффициенты предсказателя (параметры фильтра), $x_{k-i}$ --- прошлые значения ряда, а \(\hat x_k\) --- оценка (прогноз) значения \(x_k\) по прошлым наблюдениям.

Прогноз этой модели будет точным только тогда, когда удовлетворяют коэффициентам разностного уравнения, коэффициенты которого известны. К таким временным рядам относится, например, аддитивные комбинации синусоид с различными частотами. Однако, такие модели являются идеализацией, а значения их параметров (амплитуды, частоты и начальные фазы) заранее неизвестны и их надо оценивать, что снижает точность прогноза.

\underline{Модель <<Бокса-Дженкинса>>}\text{ представляет} собой классический и один из наиболее распространённых подходов к прогнозированию временных рядов на основе линейных стохастических моделей. Его основная идея заключается в том, что динамику наблюдаемых данных можно описать комбинацией авторегрессионных и скользящих средних компонент, параметры которых оцениваются по прошлым значениям ряда, а затем используются для построения прогноза. Метод был разработан Джорджем Боксом и Гвильямом Дженкинсом в 1970-е годы и с тех пор стал стандартом при анализе и моделировании стационарных временных последовательностей.

В основе подхода лежит представление о временном ряде как о реализации стационарного линейного стохастического процесса. Наиболее общая форма модели, применяемой в методе Бокса–Дженкинса, записывается как
\[
\phi(B)(1-B)^d x_t = \theta(B)\varepsilon_t,
\]
где ($x_t$) — значение временного ряда в момент времени $t$, $B$ — оператор запаздывания ($Bx_t = x_{t-1}$), ($\varepsilon_t$) — белый шум с нулевым средним и постоянной дисперсией ($\sigma^2$), ($\phi(B)$) и ($\theta(B)$) — многочлены операторов авторегрессии и скользящего среднего соответственно, а $d$ — порядок дифференцирования, обеспечивающий стационарность. В зависимости от выбора параметров получаются частные случаи: $AR(p)$ — авторегрессионная модель, $MA(q)$ — модель скользящего среднего, $ARMA(p,q)$ — их комбинация и $ARIMA(p,d,q)$ — модель для нестационарных рядов, приведённых к стационарности посредством дифференцирования.

Построение прогностической модели по Боксу–Дженкинсу включает несколько последовательных этапов.
\begin{enumerate}
	\item \textit{Предобработка данных и обеспечение стационарности.} На этом шаге устраняются тренды и сезонные компоненты, так как метод применим только к стационарным рядам. Используются дифференцирование, логарифмирование или вычитание сезонных средних.
	\item \textit{Идентификация модели.} Здесь подбираются порядки p, d и q, описывающие структуру зависимости. Основные инструменты идентификации — автокорреляционная функция (ACF) и частичная автокорреляционная функция (PACF). Для модели AR(p) характерно затухание ACF и обрыв PACF после лага p, тогда как для MA(q) — наоборот. Эти статистические характеристики строятся на основе ковариационной структуры ряда и позволяют предположить тип модели.
	\item \textit{Оценка параметров.} После выбора структуры модели параметры ($\phi_i$) и ($\theta_j$) оцениваются методами максимального правдоподобия или наименьших квадратов. Полученные оценки минимизируют среднеквадратическую ошибку прогноза.
	\item \textit{Диагностика модели.} Проверяется адекватность полученной модели и предпосылки метода. Остатки ($\hat\varepsilon_t$) должны представлять собой белый шум — без автокорреляции и со стабильной дисперсией. Для этого анализируется ACF остатков и проводится тест Льюнга–Бокса, основанный на статистике
	\[
	Q = n(n+2)\sum_{k=1}^m \frac{\hat\rho_k^2}{n-k},
	\]
	которая при справедливости гипотезы о белом шуме имеет приближённо распределение ($\chi^2_{m-p-q}$). Если остатки не удовлетворяют этим условиям, модель уточняется, и процедура повторяется.
	\item \textit{Выбор и интерпретация окончательной модели.} Из нескольких кандидатных моделей выбирается та, которая обеспечивает наилучшее качество описания данных при минимальном числе параметров. Для этого используются информационные критерии Акаике (AIC) и Шварца (BIC).
	\item \textit{Прогнозирование.} После окончательной калибровки модели строятся точечные и интервальные прогнозы. Одношаговый прогноз для AR(p)-модели определяется формулой
	\[
	\hat{x}*{t+1} = \hat{\mu} + \sum*{i=1}^p \hat{\phi}*i x*{t+1-i},
	\]
	а многошаговые прогнозы получаются итеративным применением этой зависимости, заменяя будущие значения их прогнозами. Дисперсия ошибки прогноза увеличивается с горизонтом, что учитывается при построении доверительных интервалов.
\end{enumerate}

Метод Бокса–Дженкинса базируется на анализе вторых моментов распределения, то есть на автокорреляционных и ковариационных характеристиках. Это делает его особенно эффективным для линейных стационарных процессов, где вся зависимость между наблюдениями описывается именно через ковариационную структуру. Основные статистические предпосылки применения метода — линейность модели, стационарность процесса, независимость и одинаковое распределение ошибок (белый шум), а также устойчивость и инвертируемость характеристических многочленов (корни должны лежать вне единичного круга).

С практической точки зрения метод требует достаточной длины временного ряда для устойчивой оценки параметров и чувствителен к структурным сдвигам, гетероскедастичности и шумам измерений. При нарушении этих условий возможны неточности в прогнозах, поэтому в современной практике часто применяются гибридные или модифицированные подходы (например, ARIMA–GARCH, регрессионные расширения или субполосные методы).

Таким образом, метод Бокса–Дженкинса представляет собой последовательную процедуру построения, проверки и использования линейной стохастической модели временного ряда, основанную на анализе автокорреляционных свойств данных. Он сочетает строгие статистические принципы с практической удобством и до сих пор остаётся базовым инструментом анализа и прогнозирования временных рядов в экономике, инженерии и естественных науках.

Для рассмотренных ранее примеров можно проанализировать 

В \textit{первом примере} для случая единственной ГЭС в выборке объектов нельзя оценить статистические характеристики, например дисперсию или среднее значение за конкретный день года, путём усреднения по множеству ГЭС. Однако если предположить, что поступление воды на плотину является стационарным и эргодичным процессом, то можно использовать многолетние наблюдения для оценки: среднего объёма воды для определённого сезона, дисперсии колебаний, вероятности экстремальных притоков.

Таким образом, свойства стационарности и эргодичности позволяют корректно анализировать динамику даже уникального гидротехнического объекта, используя исключительно данные его собственных наблюдений во времени.

Во \textit{втором примере} появляется принципиальная трудность: организм человека нестационарен. Со временем происходят возрастные изменения, колебания гормонального фона, метаболическая адаптация к лекарству, изменения образа жизни и состояния здоровья. Всё это нарушает условие стационарности временного ряда.

Даже если предположить наличие эргодичности, нарушение стационарности делает временные оценки неустойчивыми и потенциально искаженными. Для корректного анализа в таких случаях требуются специализированные методы обработки нестационарных временных рядов, включая фильтрацию, выделение трендов, нормализацию по сопутствующим показателям и использование адаптивных моделей.

В \textit{третьем примере} усреднение по другим клиентам нельзя применять из-за различий в бизнес-моделях, однако можно использовать локально стационарные участки ряда или модели, адаптирующиеся к изменениям во времени. В целом, если объект уникален и ансамбль недоступен, статистические характеристики временного ряда приходится оценивать только по его собственным значениям, и корректность таких оценок напрямую зависит от степени стационарности и эргодичности процесса, либо от возможностей применяемых методов компенсировать нестационарность.

\section{Оценка адекватности учета неопределенности временных рядов в подходах на основе теории нечетких множеств}

Тогда как вероятность (стандартная теория Колмогорова) моделирует стохастическую неопределённость --- случайность событий при повторимых экспериментах, нечёткая теория (Zadeh, Dubois–Prade и др.) моделирует нечеткость или эпистемическую неопределенность (высказывания о степени принадлежности, степени совместимости состояния с наблюдением).

Теория нечетких множеств позволяет выразить неопределенность даже по единичной реализации временного ряда, используя функцию принадлежности
\[
\mu: \mathrm{X} \rightarrow [0,1].
\]
Такая возможность полезна для уникальных объектов, где неопределенность может быть субъективной или основанной на экспертных оценках. Более того, методы нечетких множеств проще в применении, поскольку понятие функции принадлежности соответствует более простому аналогу вероятностной меры в теории вероятностей, что делает их удобными даже для случаев, когда неопределенность могла бы быть описана вероятностно. Это позволяет интегрировать экспертные знания напрямую, без строгих предположений о стационарности, и формализовать лингвистические переменные.


При прогнозировании нестационарных рядов нечеткие системы (например, на основе правил типа "если-то") не накладывают жестких ограничений на модель порождающего процесса, что хорошо подходит для нестационарных сценариев. Вместо этого они сопоставляют текущие данные с <<образцами>> в базе правил, фокусируясь на сходстве, а не на статистических предположениях о стационарности. Степень сходства отражается в результате композиции входного нечеткого множества с нечетким правилом:
\[
\mathbf{A'} \circ \left(\mathbf{A} \rightarrow B\right).
\]

Такие правила позволяют адекватно описывать слабо структурированные данные временных рядов через нечеткие множества, интерпретируя неопределенность, связанную не только со случайностью (как в вероятностных моделях), но и с недостаточной точностью датчиков, структурными сдвигами или экспертными допущениями. Это обеспечивает робастность: алгоритмы становятся нечувствительными к малым отклонениям от предположений, включая нестационарность процесса, благодаря минимаксным операциям Заде.

%Кроме того, нечеткие системы оперируют с нечеткими ограничениями и целями, используя лингвистические переменные для отражения сущности процесса в неопределенных условиях, что особенно актуально для многоуровневых систем с единичными реализациями. В итоге, для уникального объекта с нестационарным рядом это дает более гибкую оценку неопределенности, не требующую ансамбля данных или стационарности, как в классических моделях.

%Последний метод прогнозирования временных рядов предполагает, что ряд является стационарным, то есть обладает постоянством статистических характеристик. Другим часто используемым свойством временных рядов является свойство \textit{эргодичности}, когда среднее значение по ряду эквивалентно среднему значению, рассчитанным по всем набору реализаций этого ряда в некоторый момент времени.
%
%В некоторых случаях, например, при прогнозировании объема воды для некоторой ГЭС с уникальным географическим положением или при оценке воздействия лекарства на некоторый показатель организма человека с уникальными физиологическими свойствами, оценка статистических характеристик по ансамблю объектов невозможна или некорректна. Если при этом временной ряд по своей природе не обладает свойством эргодичности, то классические методы его моделирования оказываются неприменимы.
%
%Основанные на статистических характеристиках методы прогнозирования оперируют распределением вероятности, которое моделирует стохастическую неопределенность, возникающую из-за случайности событий. Нечеткая теория описывает распределение возможности посредством нечетких множеств, которое моделирует эпистемическую неопределенность, возникающую из-за неполноты или неточности собранной информации. Тогда для измеренных значений нечеткие множества могут быть построены по единичной реализации ряда, например, с использованием экспертной оценки или другого способа оценки их нечеткости. Тогда при использовании этих нечетких множеств в системе нечеткого вывода будет оцениваться степень их соответствия имеющимся в системе знаниям.


%В отличии от распределения вероятности, которое выражает степень соответствия какому-то конкретному закону распределения, распределение возможности выражает насколько значение $\mathbf{x}$ совместимо с имеющейся в нечеткой системе информацией\todo{, которая в системе на основе правил при логическом выводе является наиболее логичной аппроксимацией}. Это позволяет, помимо случайности ошибки измеренных значений, оценивать неопределенность, возникающую из-за неполноты знаний, меняющегося характера или постоянно нестабильного процесса.


\section{Постановка задачи}

В результате проведения диссертационного исследования требуется разработать информационную технологию прогнозирования временных рядов, предназначенную для применения в системах управления, где решения принимаются на основании анализа динамики параметров уникальных объектов или процессов.

Для этого нужно сперва разработать нечеткую модель, использующую нечеткую систему на основе правил с несинглтонной фаззификацией. Затем сформировать метод использования разработанной нечеткой модели для прогнозирования временных рядов.

Составить алгоритмическое обеспечение для эффективной реализации этапов разработанного метода прогнозирования. Для этого требуется подготовить эффективный алгоритм нечеткого логического вывода с использованием несинглтонной фаззификацией. Также в рамках разработки метода прогнозирования требуется адаптировать алгоритм построения базы правил на основе данных.

Затем следует в рамках реализации информационной технологии необходимо разработать программное обеспечение на основе предложенного метода прогнозирования с эффективной эксплуатацией возможностей аппаратной платформы.

Для разработанной информационной технологии прогнозирования следует провести оценку работоспособности. При проведении такой оценки имеет смысл продемонстрировать улучшение качества прогнозирования при применении данной информационной технологии по сравнению с другими подходами, менее адекватно учитывающими неопределенность значений временного ряда уникальных объектов. Также стоит экспериментально показать эффективность выбранной конфигурации нечеткой модели и разработанных алгоритмов.

\section{Выводы по главе}

\begin{enumerate}
	\item При построении информационных систем поддержки принятия решений часто используется информационная технология прогнозирования на основе временных рядов. Временные ряды собираются с объектов или процессов и поступают на вход информационной системе. В некоторых случаях эти объекты или процессы являются уникальными, то есть не имеющими аналогов в изучаемой совокупности.
	\item Для временных рядов, собранных с уникальных объектов или процессов, использование статистического подхода оценки неопределенности измерений не является вполне адекватным. Неадекватность вызвана тем, что для получения статистических оценок требуется выполнить усреднение по ансамблю достаточного количества объектов, либо ожидается эргодичность (или хотя бы стационарность) временного ряда.
	\item Рассмотренные распространенные модели прогнозирования --- линейный предсказатель и модель Бокса-Дженкинса --- либо вообще не предусматривают учет неопределенности во временном ряде, либо используют для этого статистические оценки.
	\item Нечеткие множества позволяют выразить неопределенность значений единичной реализации временного ряда без использования статистических оценок. Нечеткая система вывода на основе правил не накладывает ограничений на модель временного процесса и позволяет строить аппроксимирующую функцию даже для нестационарного временного ряда.
\end{enumerate}

\FloatBarrier





\begin{comment}

\chapter{Системы нечеткого вывода как метод анализа зашумленных или неопределенных входных данных}


Высокопроизводительный анализ данных (HPDM) позволяет обрабатывать и анализировать огромные массивы данных с использованием современных вычислительных технологий – как специализированных аппаратных средств (высокопроизводительных вычислительных кластеров, GPU ускорителей), так и оптимизированного программного обеспечения. Этот подход объединяет имеющиеся модели анализа данных с возможностями масштабируемых технологий Big Data и высокопроизводительных вычислений (HPC), что обеспечивает получение знаний на основе данных практически в реальном времени и значительно сокращает время от сбора данных до принятия обоснованных решений. Широкое промышленное применение получили программные пакеты (например, Apache Spark, NVIDIA RAPIDS), включающие традиционные модели анализа данных, а также инструменты JIT-компиляции готового программного кода (например, Numba).

Для определенных задач традиционные <<жёсткие>> вычисления (основанные на точных математических моделях) оказываются неэффективными или неприменимыми. Модели мягких вычислений принимают неопределённость, шум в данных и частичную истинность, что позволяет находить решения в условиях реального мира, где информация часто неполна или противоречива. Вместо строгих математических моделей используются эвристические и адаптивные методы, имитирующие человеческое мышление (например, лингвистические переменные вида <<высокая температура>>). Мягкие вычисления часто комбинируются с нейросетями (для обучения) и эволюционными алгоритмами (для оптимизации), усиливая их способность к обработке сложных систем.

Во многих случаях решение таких задач является вычислительно сложным, из-за чего технологии HPC оказываются востребованными в методах мягких вычислений. Наиболее ярко эта потребность выражается в задачах: планирование <<умных>> городов с использованием эволюционных алгоритмов для многокритериальной оптимизации \cite{Toutouh2019, Gora2015}, использование систем на основе нечеткой логике для управления и мониторинга производственных процессов в режиме реального времени \cite{Zhang2023, Vivas2022}, анализ сложных структур с большим количеством связей в задачах фармацевтики и генетики с применением генетических и роевых алгоритмов \cite{Liu2016, Easwaran2024}. Из этих примеров видно, что широкий спектр методов мягких вычислений хорошо поддаются распараллеливанию их алгоритмов.

Некоторые высокопроизводительные реализации нейро-нечетких систем были оформлены в самостоятельные программные модули. Частой практикой достижения высокой производительности нечетких систем является эффективная утилизация аппаратных ресурсов \cite{FuzzyLite}, в том числе, встраиваемых систем и плат ПЛИС \cite{Aldair2018}. В \cite{Lopez2015} и \cite{Elkano2017} представлена реализация схемы \textit{MapReduce} для классификации несбалансированных больших данных с использованием подхода Chi \cite{Chi1996}. В ситуации, когда большой набор данных целиком или фрагментарно может быть проанализирован с использованием памяти только одного вычислительного узла, но требуется провести много итераций анализа, для быстрого получения промежуточного результата нечеткого анализа целессообразно использовать (графический) ускоритель \cite{}.

% Мягкие вычисления (soft computing) --- это методология, объединяющая неточные и приближённые подходы для решения сложных задач, где традиционные «жёсткие» вычисления (основанные на точных математических моделях) оказываются неэффективными или неприменимыми. Понятие ввёл Лотфи Заде в 1994 году, и оно охватывает такие направления, как нечёткая логика, нейронные сети, эволюционные алгоритмы и вероятностные методы.

% Модели мягких вычислений принимают неопределённость, шум в данных и частичную истинность, что позволяет находить решения в условиях реального мира, где информация часто неполна или противоречива. Вместо строгих математических моделей используются эвристические и адаптивные методы, имитирующие человеческое мышление (например, лингвистические переменные вида «высокая температура»). Мягкие вычисления часто комбинируются с нейросетями (для обучения) и эволюционными алгоритмами (для оптимизации), усиливая их способность к обработке сложных систем.

% В отличие от жёстких вычислений, требующих точных моделей, мягкие вычисления фокусируются на адаптации к реальным условиям, где преобладает нечёткость. Например, они эффективны в анализе зашумлённых сигналов (как в ЭЭГ) или в системах управления, где решения принимаются на основе приближённых оценок. Также методы мягких вычислений используются для оценки параметров, где традиционные измерения затруднены (например, субъективные критерии качества).

% \todo{Ключевым элементом мягких вычислений являются нечёткие системы, которые преобразуют субъективные описания в структурированные решения через гранулирование информации — разбиение данных на семантические блоки. Это особенно важно в условиях неполных или противоречивых данных, где традиционные методы терпят неудачу. Таким образом, мягкие вычисления служат мостом между формальными моделями и реальным миром, предлагая гибкие и устойчивые к шумам решения.}

\section{Нейро-нечеткие системы}

Текущее развитие адаптивных нейро-нечетких систем (ANFIS) направлено на повышение применимости таких моделей за счет повышения точности моделирования и увеличения скорости получения результата моделирования. Прогресс по этим двум направлениям подкреплен экспериментами по использованию различных способов фаззификации \cite{Symmetry2021, Qian2023, Pekaslan2020}, импликации \cite{Shi2013, FernandezPeralta2025, Zhang2022, fern2021}, дефаззификации \cite{VanLeekwijck1999, Nasiboglu2022}, выборе $t$-норм, в том числе, $t$-норм в композиционном правиле вывода \cite{Pourabdollah2015}.

Например, в \cite{Wagner2016} предлагается использовать меру Жаккара между нечеткими множествами на входе и н. м. антецедента для определения уровня срабатывания правила в нейро-нечетких системах при несинглтонной фаззификации (NSFLS).

Существует также комбинированный подход с использованием так называемых гибких нейро-нечетких систем \cite{rutkovskiy2010, Cingolani2012}, сочетающих в себе два метода нечеткого вывода. Использование параметрических $t$-норм дает возможность осуществлять плавный переход от одного метода вывода к другому. Это позволяет совершать подбор оптимального метода вывода для конкретной задачи посредством оптимизации данного дополнительного гиперпараметра.

Чаще всего нейро-нечеткие системы используются для анализа четких входных данных взятых из четких наборов данных. Однако существуют примеры использования нечетких систем для моделирования нечетких данных. Например, прогнозирования временных рядов \cite{Pekaslan2020, Pourabdollah2017}.

Исследуются также и подходы для обучения нейро-нечетких систем. В работах широко применяется метод \cite{Wang1992} из-за своей простоты. Также появляются методы с использованием прогрессивных методов кластеризации правил \cite{Svetlakov2021}, эволюционного подхода \cite{}, градиентного спуска \cite{} и непрерывного обучения на потоковых данных \cite{Lima2010, Alves2021}.


\section{Методы нечеткого вывода}\label{sec:ch1-fuzzy-logical-inference-problem}

Эффективная организация нечеткого вывода является одной из главных точек повышения производительности при использовании нечетких систем. \todo{Проблема разрабатывается.} Предложенная в 1965 году Заде теория нечетких множеств \cite{zadeh1965} позже была использована для построения схемы нечеткого логического вывода \cite{Zadeh1996} и нечетких логических систем \cite{rutkovskiy2010}. Использование этой схемы нечеткого вывода для составных посылок оказывается затруднено из-за экспоненциальной временной сложности операций над нечеткими отношениями.

В ответ на проблему сложности использования классического нечеткого вывода спустя время появились методы Мамдани, Такаги-Сугено, Ларсена и Цукамото, вносящие в формулы нечеткого вывода упрощения для облегчения реализации. Упрощения прежде всего выражаются в использовании $t$-нормы вместо функции импликации, удовлетворяющей свойствам нечеткой импликации \cite{rutkovskiy2010}. Недостатком такого упрощения является искажение законов классической логики. Кроме того возникает расхождение при использовании лингвистической модели для анализа данных при многих правилах. Из-за этих недостатков данные подходы подвергаются критике.

В статьях \cite{Dubois2012, Izquierdo2018} авторы отмечают несколько наблюдений о различиях вывода типа Мамдани и логического типа при четких входах:
\begin{itemize}
	\item При полном соответствии входного нечеткого множестваа $A'$ одному из антецедентов $A' = A_i$ в подходе Мамдани нельзя получить на выходе $B' = B_i$ при срабатывании еще одного правила $A_j \ne A_i, j \ne i$, в отличии от конъюнкции правил $B' = B_i \wedge B_j$ при логическом подходе.
	\item В подходе Мамдани при срабатывании в базе правил $R^M$ нескольких противоречивых правил выходное нечеткое множество всегда $A \circ R^M \ne \emptyset$, а выходное дефаззифицированное значений является усреднением центров противоречивых консеквентов $B_1$ и $B_2$. При этом точка $y$ может оказаться за пределами носителей нечетких множеств  $B_1$ и $B_2$, из-за чего для $\mu_{A'}(x_0)=1$ в полученной выходной точка $\mu_{B'}(y)=0$. В случае логического вывода такие правила взаимоисключаются, а на выходе будет получено пустое нечеткое множество $\emptyset$ при достаточном удалении друг от друга $B_1$ и $B_2$.
\end{itemize}

Изучение вопроса возможности использования несинглтонной фаззификации возродил и развил американский ученный Джерри Мендель \cite{Mendel2017}. Мендель привел аналитическое сравнение, показывающее, что NSFLS превосходят синглтонные нейро-нечетки системы в задаче прогнозирования зашумленных хаотических временных рядов благодаря своей способности учитывать неопределенность входных данных непосредственно при выводе. Он также сформировал доказательство того, что NSFLS является универсальным аппроксиматором любой непрерывной функции в некоторой области.

Однако исследования Менделя ограничены проработкой несинглтонной фаззификации в системах типа Мамдани и Такаги-Сугено, которые наследуют описанные выше расхождения с каноническим логическим выводом. Также формулы нечеткого вывода в его работах легко переписываются в удобный для вычисления вид, что достигается в упрощении в виде использования одной и той же $T$-нормы в выражении вывода, что сужает возможную вариативность нечеткой модели.

Таким образом на момент проведения исследования не существовало подходов обеспечивающих эффективный канонический нечеткий логический вывод Заде с возможностью полноценно обрабатывать фаззифицированные методом non-singleton входные значения. Актуальность решения описанной проблемы подтверждается:
\begin{enumerate}
	\item увеличением количества публикаций с использованием non-singleton фаззификации для вывода типа Мамдани за последние 10 лет, в которых достигается значимый прирост в качестве нечеткого моделирования;
	\item низкой проработанностью логического типа нечеткого вывода при non-singleton фаззификации, тогда как в определенных задачах синглтонный логический вывод превосходил по качеству моделирования метод Мамдани.
\end{enumerate}


%Предложенная в 1965 году Заде теория нечетких множеств \cite{zadeh1965} позже была использована для построения нечетких систем, которые нашли свое практическое применение в системах автоматизированного управления \cite{mamdani1975}, \cite{Lee1990}, прогнозирования временных последовательностей на зашумленных временных данных \cite{} и распознавания образов \cite{}.
%
%При нечетком выводе с использованием синглтонной фаззификации, входное значение рассматривается как точное значение измеряемой величины, исключающее какую-либо погрешность измерений. Несинглтонная фаззификация позволяет учесть эту погрешность и передать ее вместе с измеренным значением на вход нечеткого вывода. Такой тип систем был подробно изучен в \cite{mendel}.
%
%В отличии от синглтонной фаззификации, использование несинглтонной фаззификации в процедуре вывода имеет экспоненциальную вычислительную сложность при вычислении композиционного правила вывода. Некоторые работы пытаются направлены на обеспечение практической применимости нечетких систем на основе несинглтонной фаззификации \cite{liang}.
%
%Другая трудность в использовании несинглтонной фаззификации состоит в точном моделировании неопределенной во входных значениях. В большинстве случаев применения несинглтонной фаззификации предполагается, что неопределенность во входах одно из общеизвестных распределений, как правило, гауссовое [6] или треугольное. Для корректного определения характера шума во входных значениях может потребоваться проведение отдельного анализа. Также можно внедрить механизм адаптивного моделирования неопределенности в блок несинглтонной фаззификации нечеткой системы, который бы оценивал уровень шума на основе потока входных данных [2].


%\section{Сравнение нечетких логических систем с нечеткими системами типа Мамдани и Такаги-Сугено}
%
%Нечеткая логическая система использует нечеткую логическую импликацию для связывания антецедента и консеквента в нечетком правиле, а также связку И для агрегации правил.
%
%Нечеткая система типа Мамдани использует t-норму для связывания антецедента и консеквента в нечеком правиле, а также связку ИЛИ для агрегации правил.
%
%Как заключено в [https://ssrn.com/abstract=2900827] 

\end{comment}

