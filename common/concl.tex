%% Согласно ГОСТ Р 7.0.11-2011:
%% 5.3.3 В заключении диссертации излагают итоги выполненного исследования, рекомендации, перспективы дальнейшей разработки темы.
%% 9.2.3 В заключении автореферата диссертации излагают итоги данного исследования, рекомендации и перспективы дальнейшей разработки темы.
\begin{enumerate}
  \item Проведен анализ актуальных проблем методов нечеткого вывода с использованием несинглтонной фаззификации. Приведено описание понятия нечеткого значения истинности. Обозначена актуальность оценки качества нечеткого моделирования при использовании нечеткого вывода логического типа с использованием несинглтонной фаззификации на задаче прогнозирования временных рядов, описана адаптивная процедура оценки неопределенности и приведен адаптированный алгоритм метода роя частиц для подбора параметров базы правил.
  \item Разработан и теоретически обоснован новый способ нечеткого логического вывода на основе нечеткого значения истинности, позволяющий оценить нечеткую степень совместимости входных нечетких множеств и н. м. антецедента правила по каждом входу независимо. Получен новый вид нечетких правил <<Если \textit{истинно}, то $B_k$>>. Теоретически и экспериментально было показано, что сложность вычисления свертки НЗИ $O(|V|\cdot n)$ и сложность нечеткого вывода на основе НЗИ  $O(|V|\cdot |Y|)$ полиномиально зависят от числа входов $n$.
  \item Предложен эффективный параллельный алгоритмы вычисления свертки НЗИ для нечетких систем с несколькими входами, обеспечивающий сложность свертки НЗИ $O(|V|\times log(n))$. Выполнена параллельная реализация нечеткого логического вывода с использованием технологии CUDA. Разработаны оптимизации по организации памяти и вычислений, обеспечивающие масштабируемость и эффективность при обработке больших массивов данных. 
  Дефаззификация по методу среднего максимума (MeOM) реализована на основе метода оптимизации Gradient-aware PSO.
  \item Разработанный метод нечеткого вывода на основе НЗИ адаптирован к задаче мультиклассовой классификации, где при использовании метода дефаззификации по \todo{обеспечавается упрощение}.
  \item На наборе данных Mackey-Glass для нечеткой модели на основе логического типа вывода и несинглтонной фаззификации достигнута точность прогнозирования временных рядов по метрике sMAPE --- $8\%$ при размере базе правил --- 30 правил. Это значительно превосходит точность прогнозирования по этой метрике в том же эксперименте при использовании синглтонной фаззификации ($\approx 40\%$) и имеет значительно меньшее количество правил при сопоставимой точности прогнозирования $\approx 10\%$ для вывода типа Мамдани с несинглтонной фаззификацией (184 правил).
  \item Программная реализация разработанной нечеткой модели была применена для решения задачи прогнозирования помесячных объемов транзакций безналичных платежей корпоративных клиентов банка. Достигнут прирост точности прогнозирования в $12\%/5\%$ по метрике RMAE при прогнозировании на один/три месяца в сравнении с лучшей моделью --- многослойный перцептрон. 
  %\item Практическая значимость работы подтверждена реализацией программного комплекса с удобным интерфейсом для интеграции в современные системы анализа данных, а также возможностью применения в задачах поддержки принятия решений, прогнозирования и интеллектуального анализа данных.
  %\item Перспективы дальнейших исследований связаны с развитием методов автоматического построения базы правил, адаптивной фаззификации, расширением класса функций принадлежности, а также применением разработанного подхода к задачам с многомерными и потоковыми данными.
\end{enumerate}
