%% Согласно ГОСТ Р 7.0.11-2011:
%% 5.3.3 В заключении диссертации излагают итоги выполненного исследования, рекомендации, перспективы дальнейшей разработки темы.
%% 9.2.3 В заключении автореферата диссертации излагают итоги данного исследования, рекомендации и перспективы дальнейшей разработки темы.
\begin{enumerate}
  \item Анализ состояния исследований в области анализа качественных/за­шумленных данных с использованием нечеткого моделирования показал недостатки существующих методов. В частности, было выявлено отсут­ствие таких методов для моделирования систем и процессов, имеющих множество качественных/зашумленных входов, которые работали бы с полиномиальной вычислительной сложностью от количества входов и не были бы связаны со значительными упрощениями теории нечеткого логического вывода.
  \item Для логического подходов были разработаны методы, позволяющие осуществлять нечеткий логиеский вывод с полиномиаль­
  ной сложностью для произвольного набора используемых t-норм, что обеспечивает более гибкую настройку таких систем. Для определенных частных случаев в наборах используемых норм были разработаны опти­мизированные версии этих методов с использованием меры возможности.
  \item Разработан метод классификации для объектов со многими нечеткими входах\ldots
  \item Математическое моделирование показало \ldots
  \item Для выполнения поставленных задач был создан \ldots
\end{enumerate}
