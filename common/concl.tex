%% Согласно ГОСТ Р 7.0.11-2011:
%% 5.3.3 В заключении диссертации излагают итоги выполненного исследования, рекомендации, перспективы дальнейшей разработки темы.
%% 9.2.3 В заключении автореферата диссертации излагают итоги данного исследования, рекомендации и перспективы дальнейшей разработки темы.
\begin{enumerate}
	\item Проведенный анализ показал, что существующие методы не позволяют адекватно отразить уникальность исследуемого объекта и учесть изменчивость характеристик данных входных временных рядов. 
	\item Разработан метод прогнозирования временных рядов в задачах управления на основе нечеткого вывода, который позволяет адекватно отразить уникальность и изменчивость характеристик входных данных.
	\item Разработаны вычислительные процедуры предложенного метода прогнозирования временных рядов, обеспечивающие вычислительную эффективность за счет использования предложенного метода нечеткого вывода, алгоритма свертки НЗИ и адаптированного алгоритма PSO для параметрической оптимизации базы правил и дефаззификации MeOM.
	\item Разработанный прототип программной поддержки информационной технологии прогнозирования на основе параллельной архитектуры вычислений продемонстрировал прирост точности при сравнении с другими методами прогнозирования.
\end{enumerate}


\textbf{Выводы}

\begin{itemize}
	\item Разработанная информационная технология позволяет усовершенствовать инструменты информационного обеспечения прогнозирования на основе временных рядов.
	%\item Информационное обеспечения обогащено методом прогнозирования временных рядов на основе нечеткого моделирования, адекватно отражающего уникальные свойства объекта.
	\item Реализация информационной технологии может быть выполнена на стандартных компьютерах, оснащенных графическими ускорителями с поддержкой технологии CUDA.
%	\item Разработанный метод на основе нечетких систем с несинглтонной фаззификацией обеспечивает более адекватный учет более широкого спектра неопределенностей в эмпирических временных данных.
%	\item Реализованный прототип программной системы предоставляет возможность эффективной обработки временных данных, содержащих неопределенность, за счет использования технологий параллельных вычислений, доступных на большинстве современных аппаратных платформ.
%	\item Реализованная информационная система дает возможность принимать более адекватные решения управления.
\end{itemize}

\textbf{Перспективы дальнейших исследований}

Повышение эффективности алгоритма прогнозирования временных рядов с использованием нечеткой системы за счет более эффективного способа агрегации базы правил.

%Повышение адекватности учёта неопределённости может быть достигнуто за счёт перехода от нечетких множеств первого типа к нечетким множествам второго типа (type-2), которые дополнительно описывают неопределённость самих функций принадлежности посредством так называемой области неопределённости. Это обеспечивает более точное моделирование ситуаций, характеризующихся неточными или противоречивыми исходными данными и экспертными оценками, а также повышает устойчивость нечеткого вывода при вариациях параметров модели.


\textbf{Рекомендации по использованию}

Рекомендуется организациям разрабатывающим программы для информационного обеспечения прогнозирования.
