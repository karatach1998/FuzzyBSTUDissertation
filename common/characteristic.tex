
{\actuality} Высокопроизводительный интеллектуальный анализ данных дает возможность принимать обоснованные решения на основе знаний, получаемых посредством обработки данных со скоростью близкой к реальному времни. Семейство методов мягких вычислений с применением техник высокопроизводительного анализа данных открывает возможность находить закономерности и взаимосвязи в данных, содержащих неопределенность. Одним из методов мягких вычислений предназначенным для анализа неопределенных данных являются методы нечеткого моделирования.

В описанной Л. Заде теории нечеткой логики важной проблемой остается задача нечеткого логического вывода. Распространение получили подходы опирающиеся на использованием методов нечеткого вывода выработанных Э. Мамдани, П. Ларсеном, Т. Такаги, М. Сугено и Ю. Цукамато. Эти подходы, а также основанные на них производные методы, как правило, используют четкие значения входов и $t$-норму вместо импликации, что позволяет упростить реализацию нечеткого вывода. Однако такое упрощение приводит к несоответствию с теорией Заде, что можно выявить при рассмотрении лингвистических моделей со многими нечеткими входами, то есть когда используется несинглтонный метод фаззификации.

В теории нечеткой логики нечеткий логический вывод реализуется с помощью обобщенных нечетких правил \textit{modus ponens} и \textit{modus tollens} на основе \textit{композиционного правила вывода}. При нескольких входах вычисление по данным правилом приводит к экспоненциальной зависимости вычислительной сложности от количества входов. Данное ограничение является основным препятстивем для применения нечеткого логического вывода с несколькими посылками, тогда как необходимость анализа многомерных данных \todo{является} актуальной задачей, \todo{например, \dots}.

Разработку нечеткого вывода с использованием несинглтонной фаззификации возродил Д. Мендель. Он продемонстрировал прирост качества нечеткого моделирования с использованием фаззификации типа non-singleton, например, в задаче прогнозирования временных рядов. Однако его исследования ограничены проработкой нечеткого вывода типа Мамдани и Такаги-Сугено. Кроме того Мендель строит формальные выкладки процедуры нечеткого вывода при использовании одной и той же $t$-нормы, что необоснованно сужает гибкость формул вывода.

Проблемы связанные с нечетким выводом и нечетким моделированием в России изучались и прорабатывались И. А. Ходашинстким, Н. Г. Ярушкина, А. И. Аверкин.



\ifsynopsis
%Этот абзац появляется только в~автореферате.
%Для формирования блоков, которые будут обрабатываться только в~автореферате,
%заведена проверка условия \verb!\!\verb!ifsynopsis!.
%Значение условия задаётся в~основном файле документа (\verb!synopsis.tex! для
%автореферата).
\else
%Этот абзац появляется только в~диссертации.
%Через проверку условия \verb!\!\verb!ifsynopsis!, задаваемого в~основном файле
%документа (\verb!dissertation.tex! для диссертации), можно сделать новую
%команду, обеспечивающую появление цитаты в~диссертации, но~не~в~автореферате.
\fi

% {\progress}
% Этот раздел должен быть отдельным структурным элементом по
% ГОСТ, но он, как правило, включается в описание актуальности
% темы. Нужен он отдельным структурынм элемементом или нет ---
% смотрите другие диссертации вашего совета, скорее всего не нужен.

{\researchObject} являются методы высокопроизводительного анализа данных на основе нечеткой логики.

{\researchSubject} являются методы и алгоритмы нечеткого логического вывода с несинглтонной фаззификацией, а также методы высокопроизводительного анализа данных на его основе.

{\aim} данной работы является повышение производительности анализа неопределенных данных путем разработки математического и программного обеспечения с применением нечетких систем на основе правил при несинглтонной фаззификации.

Для~достижения поставленной цели необходимо было решить следующие {\tasks}:
\begin{enumerate}[beginpenalty=10000] % https://tex.stackexchange.com/a/476052/104425
  \item Провести обзор проблем и предлагаемых подходов построения и реализации нечетких систем на основе правил для анализа данных.
  \item Разработать метод вывода на основе нечеткого значения истинности (НЗИ) для системы MISO-структуры логического типа и типа Мамдани, обеспечивающий полиномиальную вычислительную сложность.
  \item Разработать метод классификации объектов с нечетким или качественным описанием атрибутов на основе правил по предложенному методу нечеткого вывода с использованием НЗИ с возможностью комбинирования различных T-норм.
  \item Разработать нечеткую модель регрессии временных рядов с нечеткими оценками значений временной последовательности на основе метода нечеткого вывода с использованием нечеткого значения истинности.
  \item Выполнить программную реализацию выработанного метода нечеткого вывода и разработанной модели регрессии временных рядов с использованием технологии параллельных вычислений CUDA, обеспечив эффективность реализации. Реализовать алгоритм построения базы правил на основе данных.
  \item Применить разработанный модуль нечеткого логического вывода для высокопроизводительного анализа зашумленных данных в выбранной предметной области.
\end{enumerate}

{\novelty}
\begin{enumerate}[beginpenalty=10000] % https://tex.stackexchange.com/a/476052/104425
	\item Впервые применено нечеткое значение истинности и принцип обобщения для получения выходного значения при нескольких нечетких входах в соответствии с обобщенным нечетким правилом вывода \textit{modus ponens} для нечетких систем логического типа, в результате чего была получена новая структура базы правил: <<Если \textit{истинно}, то $B_k$>>.
	\item Разработан метод нечеткого вывода логического типа с использованием нечеткого значения истинности, имеющий полиномиальную вычислительную сложность при многих нечетких входах.
	\item Разработан метод классификации для объектов с нечеткими оценками признаков на основе предложенного метода нечеткого логического вывода с использованием нечеткого значения истинности.
	\item Разработан метод регрессии временных рядов с нечеткими оценками измеренных значений на основе предложенного метода нечеткого логического вывода и алгоритм построения базы правил %\dots.
	\item Разработан параллельный алгоритм, реализующий нечеткий вывод на основе нечеткого значения истинности %с применением отбора \dots
\end{enumerate}

{\theoreticalValue} заключается в расширении класса задач анализа данных, эффективно решаемых при помощи нечеткого моделирования, соответствующего теории нечеткого вывода Л. Заде, за счет использования нечеткого значения истинности и разработанного алгоритма параллельной свертки нечетких значений истинности по входам.

{\practicalValue} заключается в разработанных на основе предложенного метода нечеткого вывода нечетких логических моделях адаптированные для задач классификации и регрессии при задании измеренных характеристиках моделируемых объектов нечеткими значениями, а также в реализованном на основе разработанных моделей нечеткого вывода программного обеспечения и результатами проведенных вычислительных экспериментов по оценке производительности этой реализации.

{\methods} В работе использованы методы теории нечетких множеств, нечетких отношений, нечеткого логического вывода и мягких вычислений, а также методы оптимизации и методы анализа данных.

{\defpositions}
\begin{enumerate}[beginpenalty=10000] % https://tex.stackexchange.com/a/476052/104425
  \item Разработан метод нечеткого логического вывода при фаззификации типа non-singleton на основе нечеткого значения истинности. Разработанный метод нечеткого вывода имеет новый вид нечетких правил <<Если \textit{истинно}, то $B_k$>> и обеспечивает полиномиальную вычислительную сложность при многих нечетких входах. Выполненная параллельная реализация данного метода продемонстрировала линейный рост времени работы алгоритма с увеличением количества входов.
 
  \item Разработан метод классификации для объектов с нечеткими оценками признаков с использованием нечетких систем логического типа на основе нечеткого значения истинности.
  
  \item Разработан параллельный алгоритм свертки НЗИ, сокращающий вычислительною сложность до $O(|V|\cdot \log{n})$.

  \item Выполнена программная реализация разработанного метода нечеткого вывода с применением разработанного алгоритма свертки НЗИ на основе технологии параллельного программирования CUDA. Высокая производительность реализации обеспечена за счет эффективного использования аппаратных ресурсов графического ускорителя.
  
  \item Разработана нечеткая модель регрессии для временных рядов с нечеткими оценками неопределенности измеренных значений с использованием нечеткого вывода логического типа на основе нечеткого значения истинности. Данная нечеткая модель показала прирост качества прогнозирования временных рядов ($8\%$ по метрике sMAPE) на наборе данных Mackey-Glass по сравнению с нечетким моделированием на основе синглтонной фаззификации (с точностью $\approx 40\%$), а также требуемое количество правил при логическом типа вывода (30) оказалось значительно меньше количества правил при вывода типа Мамдани (184) для достижения сопоставимого качества ($\approx 10\%$).
  
  %\todo{Может возникнуть вопрос: что сделанное в работе/какую новизну подтверждает проведенный эксперимент, демонстрирующий хорошее качество прогнозирования?}
  %\item \dots
  %\item \dots
\end{enumerate}
%В папке Documents можно ознакомиться с решением совета из Томского~ГУ
%(в~файле \verb+Def_positions.pdf+), где обоснованно даются рекомендации
%по~формулировкам защищаемых положений.

{\specialityRelation} Содержание диссертации соответствует следующим пунктам паспорта специальности 2.3.8. Информатика и информационные процессы по следующим направлениям исследований:
\begin{itemize}
	\item п. 4 паспорта специальности: Разработка методов и технологий цифровой обработки аудиовизуальной информации с целью обнаружения закономерностей в данных, включая обработку текстовых и иных изображений, видео контента. Разработка методов и моделей распознавания, понимания и синтеза речи, принципов и методов извлечения требуемой информации из текстов.
	\item п. 13 паспорта специальности: Разработка и применение методов распознавания образов, кластерного анализа, нейро-сетевых и нечетких технологий, решающих правил, мягких вычислений при анализе разнородной информации в базах данных.
\end{itemize}

{\reliability} полученных результатов обеспечивается корректным применением математических методов, доказанностью выводов, результатами проведенных экспериментов и их сопоставлением с результатами экспериментов других научных групп, апробацией на научно-практических конференциях.

\textbf{Внедрение результатов диссертационного исследования.} Разработанная в процессе диссертационного исследования нечеткая модель прогнозирования временных рядов была зарегистрирована как программа ЭВМ в Роспатенте и внедрена в процесс составления прогнозов активности клиентов по одному из банковских продуктов в ПАО <<Сбербанк>>. Предложенные алгоритмы также использованы при выполнении научного проек­та при поддержке РФФИ №20-07-00030 «Разработка высокопроизводительных методов интеллектуального анализа данных на основе нечеткого моделиро­вания и создание компьютерной системы поддержки принятия решений для классификации и прогнозирования».

{\probation}
Основные результаты работы докладывались~на:
\begin{enumerate}
	\item Международная конференция <<Перспективные компьютерные и цифровые технологии» (ACDT 2021)>>, г. Белгород, 2021.
	\item XV Международная научная конференция <<Параллельные вычислительные технологии (ПаВТ) 2021>>, г. Волгоград, 2021.
	\item XI Международной научно-практической конференции <<Интегрированные модели и мягкие вычисления в искусственном интеллекте (ИММВ-2022)>>, г. Коломна, 2022 г.
	\item XX Национальная конференция по искусственному интеллекту с международным участием (КИИ-2022), г. Москва, 2022.
	\item XVII Международная научная конференция <<Параллельные вычислительные технологии (ПаВТ) 2023>>, г. Санкт-Петербург, 2023.
	\item XXI Национальная конференция по искусственному интеллекту с международным участием (КИИ-2023), г. Смоленск, 2023.
	\item Х Всероссийская научно-техническая конференция «Информационные технологии в науке, образовании и производстве» (ИТНОП-2025), г. Орел, 2025.
\end{enumerate}

{\contribution} Постановка цели и задач научного исследования, планирование экспериментов, подготовка публикаций по выполненной работе проводилась совместно с научным руководителем. Автором самостоятельно разработан алгоритм параллельной свертки нечетких значений истинности, а также выполнена реализация механизма нечеткой вывода с использованием нечеткого значения истинности и модели регрессии временных рядов для систем MISO-структуры при использовании фаззификации типа non-singleton, а также выполнена реализация алгоритма оптимизации для настройки параметров термов в базе правил на основе набора данных. С использованием данной реализации автором выполнены эксперименты, подтверждающие полиномиальную зависимость времени выполнения нечеткого вывода от количества входов, а также проведены эксперименты для оценки качества моделирования нечеткой логической системой с использованием фаззификации типа non-singleton в задаче прогнозирования временных рядов для набора данных Mackey-Glass и набора данных из практической задачи прогнозирования транзакционной активности клиентов банка. Проведена аппробация разработанных методов нечеткого вывода и нечеткого моделирования.

\ifnumequal{\value{bibliosel}}{0}
{%%% Встроенная реализация с загрузкой файла через движок bibtex8. (При желании, внутри можно использовать обычные ссылки, наподобие `\cite{vakbib1,vakbib2}`).
    {\publications} Основные результаты по теме диссертации изложены
    в~XX~печатных изданиях,
    X из которых изданы в журналах, рекомендованных ВАК,
    X "--- в тезисах докладов.
}%
{%%% Реализация пакетом biblatex через движок biber
    \begin{refsection}[bl-author, bl-registered]
        % Это refsection=1.
        % Процитированные здесь работы:
        %  * подсчитываются, для автоматического составления фразы "Основные результаты ..."
        %  * попадают в авторскую библиографию, при usefootcite==0 и стиле `\insertbiblioauthor` или `\insertbiblioauthorgrouped`
        %  * нумеруются там в зависимости от порядка команд `\printbibliography` в этом разделе.
        %  * при использовании `\insertbiblioauthorgrouped`, порядок команд `\printbibliography` в нём должен быть тем же (см. biblio/biblatex.tex)
        %
        % Невидимый библиографический список для подсчёта количества публикаций:
        \phantom{\printbibliography[heading=nobibheading, section=1, env=countauthorvak,          keyword=biblioauthorvak]%
        \printbibliography[heading=nobibheading, section=1, env=countauthorwos,          keyword=biblioauthorwos]%
        \printbibliography[heading=nobibheading, section=1, env=countauthorscopus,       keyword=biblioauthorscopus]%
        \printbibliography[heading=nobibheading, section=1, env=countauthorconf,         keyword=biblioauthorconf]%
        \printbibliography[heading=nobibheading, section=1, env=countauthorother,        keyword=biblioauthorother]%
        \printbibliography[heading=nobibheading, section=1, env=countregistered,         keyword=biblioregistered]%
        \printbibliography[heading=nobibheading, section=1, env=countauthorpatent,       keyword=biblioauthorpatent]%
        \printbibliography[heading=nobibheading, section=1, env=countauthorprogram,      keyword=biblioauthorprogram]%
        \printbibliography[heading=nobibheading, section=1, env=countauthor,             keyword=biblioauthor]%
        \printbibliography[heading=nobibheading, section=1, env=countauthorvakscopuswos, filter=vakscopuswos]%
        \printbibliography[heading=nobibheading, section=1, env=countauthorscopuswos,    filter=scopuswos]}%
        %
        \nocite{*}%
        %
        {\publications} Основные результаты по теме диссертации изложены в~\arabic{citeauthor}~печатных изданиях,
        \arabic{citeauthorvak} из которых изданы в журналах, рекомендованных ВАК%
        \ifnum \value{citeauthorscopuswos}>0%
            , \arabic{citeauthorscopuswos} "--- в~периодических научных журналах, индексируемых Web of~Science и Scopus%
        \fi%
        \ifnum \value{citeauthorconf}>0%
            , \arabic{citeauthorconf} "--- в~тезисах докладов.
        \else%
            .
        \fi%
        \ifnum \value{citeregistered}=1%
            \ifnum \value{citeauthorpatent}=1%
                Зарегистрирован \arabic{citeauthorpatent} патент.
            \fi%
            \ifnum \value{citeauthorprogram}=1%
                Зарегистрирована \arabic{citeauthorprogram} программа для ЭВМ.
            \fi%
        \fi%
        \ifnum \value{citeregistered}>1%
            Зарегистрированы\ %
            \ifnum \value{citeauthorpatent}>0%
            \formbytotal{citeauthorpatent}{патент}{}{а}{}%
            \ifnum \value{citeauthorprogram}=0 . \else \ и~\fi%
            \fi%
            \ifnum \value{citeauthorprogram}>0%
            \formbytotal{citeauthorprogram}{программ}{а}{ы}{} для ЭВМ.
            \fi%
        \fi%
        % К публикациям, в которых излагаются основные научные результаты диссертации на соискание учёной
        % степени, в рецензируемых изданиях приравниваются патенты на изобретения, патенты (свидетельства) на
        % полезную модель, патенты на промышленный образец, патенты на селекционные достижения, свидетельства
        % на программу для электронных вычислительных машин, базу данных, топологию интегральных микросхем,
        % зарегистрированные в установленном порядке.(в ред. Постановления Правительства РФ от 21.04.2016 N 335)
    \end{refsection}%
    \begin{refsection}[bl-author, bl-registered]
        % Это refsection=2.
        % Процитированные здесь работы:
        %  * попадают в авторскую библиографию, при usefootcite==0 и стиле `\insertbiblioauthorimportant`.
        %  * ни на что не влияют в противном случае
        \nocite{vakbib2}%vak
        \nocite{patbib1}%patent
        \nocite{progbib1}%program
        \nocite{bib1}%other
        \nocite{confbib1}%conf
    \end{refsection}%
        %
        % Всё, что вне этих двух refsection, это refsection=0,
        %  * для диссертации - это нормальные ссылки, попадающие в обычную библиографию
        %  * для автореферата:
        %     * при usefootcite==0, ссылка корректно сработает только для источника из `external.bib`. Для своих работ --- напечатает "[0]" (и даже Warning не вылезет).
        %     * при usefootcite==1, ссылка сработает нормально. В авторской библиографии будут только процитированные в refsection=0 работы.
}

% При использовании пакета \verb!biblatex! будут подсчитаны все работы, добавленные
% в файл \verb!biblio/author.bib!. Для правильного подсчёта работ в~различных
% системах цитирования требуется использовать поля:
% \begin{itemize}
%         \item \texttt{authorvak} если публикация индексирована ВАК,
%         \item \texttt{authorscopus} если публикация индексирована Scopus,
%         \item \texttt{authorwos} если публикация индексирована Web of Science,
%         \item \texttt{authorconf} для докладов конференций,
%         \item \texttt{authorpatent} для патентов,
%         \item \texttt{authorprogram} для зарегистрированных программ для ЭВМ,
%         \item \texttt{authorother} для других публикаций.
% \end{itemize}
% Для подсчёта используются счётчики:
% \begin{itemize}
%         \item \texttt{citeauthorvak} для работ, индексируемых ВАК,
%         \item \texttt{citeauthorscopus} для работ, индексируемых Scopus,
%         \item \texttt{citeauthorwos} для работ, индексируемых Web of Science,
%         \item \texttt{citeauthorvakscopuswos} для работ, индексируемых одной из трёх баз,
%         \item \texttt{citeauthorscopuswos} для работ, индексируемых Scopus или Web of~Science,
%         \item \texttt{citeauthorconf} для докладов на конференциях,
%         \item \texttt{citeauthorother} для остальных работ,
%         \item \texttt{citeauthorpatent} для патентов,
%         \item \texttt{citeauthorprogram} для зарегистрированных программ для ЭВМ,
%         \item \texttt{citeauthor} для суммарного количества работ.
% \end{itemize}
% Счётчик \texttt{citeexternal} используется для подсчёта процитированных публикаций;
% \texttt{citeregistered} "--- для подсчёта суммарного количества патентов и программ для ЭВМ.

%Для добавления в список публикаций автора работ, которые не были процитированы в
%автореферате, требуется их~перечислить с использованием команды \verb!\nocite! в
%\verb!Synopsis/content.tex!.
