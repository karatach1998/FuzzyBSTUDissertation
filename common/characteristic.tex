
{\actuality} Процесс управления сводится к принятию решений и контролю за их реализацией. Принимаемые решения должны быть обоснованными. Методы обоснования решений входят в системы информационного обеспечения управления. Одним из таких методов является использование результатов анализа возможных исходов при различном воздействии на объект управления. Для анализа выполняется прогнозирование этих исходов. Прогнозирование часто осуществляется по временным рядам, которые генерируются при регистрации значений параметров некоторого контролируемого процесса. Обоснованность принимаемых на основе прогноза решений зависит от адекватности извлечения и учета неопределенности временных данных используемой моделью прогнозирования.

Решением задачи прогнозирования временных рядов занималось большое количество исследователей. В результате среди прочих выделился набор широко используемых моделей прогнозирования, таких как линейный предсказатель и модель Бокса-Дженкинса. Такие подходы используют для получения вероятностных оценок характеристик временных рядов свойство стационарности и статистическое усреднение по ансамблю характеристик объектов. Неопределенность при этом часто рассматривается как ошибки измерений, подчиненные конкретному закону распределения, то есть происходит <<навязывание законов природе>>. Таким образом, из-за наложения этих ограничений вероятностные модели не совсем адекватны.

Альтернативным подходом может служить использование нечетких систем на основе правил с фаззификацией типа non-singleton разработанной Л. Заде теории нечетких множеств. Тогда как, вероятностные модели основаны на усреднении характеристик временных рядов. Для задания функции принадлежности при этом не требуется иметь ансамбль временных рядов или свойство стационарности отдельного временного ряда.

Распространенные подходы нечеткого вывода, выработанные Э. Мамдани, П. Ларсеном, Т. Такаги, М. Сугено и Ю. Цукамато, используют четкие значения входов, полученные в результате фаззификации singleton для простоты реализации. Из-за такого упрощения теряется информация о неопределенности входных значений временных рядов. Использование фаззификации типа non-singleton позволяет сохранить информацию о нечеткости входных данных, но приводит к экспоненциальной вычислительной сложности нечеткого логического вывода при многих входах модели. Тогда, чтобы использовать нечеткий логический вывод при фаззификации non-singleton, необходимо создать нечеткую модель с разработкой эффективных алгоритмов.
 
%В теории нечеткой логики нечеткий логический вывод реализуется с помощью обобщенных нечетких правил \textit{modus ponens} и \textit{modus tollens} на основе \textit{композиционного правила вывода}. При нескольких входах вычисление по данным правилом приводит к экспоненциальной зависимости вычислительной сложности от количества входов. Данное ограничение является основным препятствием для применения нечеткого логического вывода с несколькими посылками, тогда как необходимость анализа многомерных данных является актуальной задачей.

Таким образом, задача разработки информационной технологии прогнозирования на основе нечеткого вывода при фаззификации типа non-singleton является актуальной.

%Разработку нечеткого вывода с использованием несинглтонной фаззификации возродил Д. Мендель. Он рассмотрел задачу вывода в постановке non-singleton нечеткого моделирования с использованием фаззификации типа non-singleton, например, в задаче прогнозирования временных рядов. Однако его исследования ограничены проработкой нечеткого вывода типа Мамдани и Такаги-Сугено. Кроме того Мендель строит формальные выкладки процедуры нечеткого вывода при использовании одной и той же $t$-нормы, что необоснованно сужает гибкость формул вывода.

Проблемы, связанные с нечетким выводом и нечетким моделированием в России изучались и прорабатывались А.Н. Аверкин, В.В Борисов, И.А. Ходашинский, А.В. Язенин, Н.Г. Ярушкина. За рубежом развитием нечеткой теории занимались  P. Angelov, D. Dubois, H. Ishibuchi, J.M. Mendel, L. Rutkowski, H. Tanaka, R.R. Yager, T. Yasukawa, L. Wang и др..



\ifsynopsis
%Этот абзац появляется только в~автореферате.
%Для формирования блоков, которые будут обрабатываться только в~автореферате,
%заведена проверка условия \verb!\!\verb!ifsynopsis!.
%Значение условия задаётся в~основном файле документа (\verb!synopsis.tex! для
%автореферата).
\else
%Этот абзац появляется только в~диссертации.
%Через проверку условия \verb!\!\verb!ifsynopsis!, задаваемого в~основном файле
%документа (\verb!dissertation.tex! для диссертации), можно сделать новую
%команду, обеспечивающую появление цитаты в~диссертации, но~не~в~автореферате.
\fi

% {\progress}
% Этот раздел должен быть отдельным структурным элементом по
% ГОСТ, но он, как правило, включается в описание актуальности
% темы. Нужен он отдельным структурынм элемементом или нет ---
% смотрите другие диссертации вашего совета, скорее всего не нужен.

{\researchObject} являются информационная технология управления на основе анализа временных рядов.

{\researchSubject} являются методы и алгоритмы прогнозирования временных рядов на основе нечеткого моделирования.

{\aim} данной работы является совершенствование информационного обеспечения управления на основе разработки информационной технологии и алгоритмов прогнозирования временных рядов с использованием нечетких систем, адекватно учитывающих уникальность объектов.

Для~достижения поставленной цели необходимо было решить следующие {\tasks}:
\begin{enumerate}[beginpenalty=10000] % https://tex.stackexchange.com/a/476052/104425
  \item Анализ методов прогнозирования временных рядов в задачах управления с позиции адекватности учета уникальности свойств генерирующих их объектов.
  \item Разработка метода прогнозирования временных рядов с использованием нечетких систем на основе правил. %учета неопределенности их значений с помощью нечетких моделей, учитывающих нечеткость входов.
  \item Разработка алгоритмической реализации метода прогнозирования временных рядов на основе нечетких моделей.
  \item Разработка информационной технологии прогнозирования временных рядов и прототипа ее программной реализации.
  \item Оценка работоспособности программно-алгоритмической реализации на основе вычислительных экспериментов.
\end{enumerate}

{\novelty} обладают следующие результаты диссертационного исследования:
\begin{enumerate}[beginpenalty=10000] % https://tex.stackexchange.com/a/476052/104425
	\item Установлена неадекватность отражения свойств уникальных объектов на основе существующих методов прогнозирования временных рядов. %существующих методов показали, что они основаны на моделях, которые не учитывают уникальности объекта и как правило основываются на гипотезе стационарности временных рядов, что делает результаты прогнозирования неустойчивыми к изменчивости статистических характеристик временного ряда.
	%Сделано предложение использовать нечеткую систему, позволяющей учесть нечеткость входных данных.
	\item Модель нечеткого вывода на основе нечеткого значения истинности (НЗИ), позволяющая адекватно учесть изменчивость характеристик входных данных.% за счет использования несинглтонной фаззификации
	\item Метод вывода для систем MISO-структуры логического типа и типа Мамдани на основе нечеткого правила <<Если \textit{нзи} есть ИСТИННО, то $y$ есть $B$>>, снижающий экспоненциальную сложность до полиномиальной.
	\item Метод прогнозирования временных рядов на основе разработанной системы нечеткого вывода.%включающий этапы обучения
	\item Алгоритмы реализации метода прогнозирования на основе нечеткого вывода с применением технологии параллельных вычислений.% обеспечивающих полиномиальную вычислительную сложность. %Архитектура параллельных вычислений для разработанных алгоритмов прогнозирования временных рядов.
\end{enumerate}

% то что может быть использовано в других задачах
{\theoreticalValue} определяет принцип учета нечеткости входных данных, модель нечеткого вывода на этой основе и способы получения результата. % разработанных методом нечеткого вывода и архитектурой системы, его реализующей.

{\practicalValue} определяется возможностью построения систем прогнозирования временных рядов в задачах принятия решения с учетом неопределенности временных рядов на основе разработанных результатов. Применение разработанной информационной технологии позволяет осуществить прогнозирование временных рядов с изменчивыми характеристиками в задачах принятия решения. Теоретические результаты и разработанный прототип программного обеспечения используется в учебном процессе БГТУ им. В. Г. Шухова. Среди них две зарегистрированные программы ЭВМ:
\begin{enumerate} 
\item Свидетельство \textnumero 2025661081 <<Среда для параллельного обобщенного нечеткого вывода на основе нечеткого значения истинности>>.
\item Свидетельство \textnumero 2025660990 <<Библиотека моделирования временных рядов на основе параллельного нечеткого вывода с использованием нечеткого значения истинности>>.
\end{enumerate}

Работа выполнена при участии в проекте РФФИ \textnumero 20-07-00030 <<Разработка высокопроизводительных методов интеллектуального анализа данных на основе нечёткого моделирования и создание компьютерной системы поддержки принятия решений для классификации и прогнозирования>>.


%Справка об использовании: Приняток к использованию

{\methods} В работе использовалась методология принятия решений, методы теории временных рядов, теории нечетких множеств и машинного обучения. % основывается на анализе резульатаов 

{\defpositions}
\begin{enumerate}[beginpenalty=10000] % https://tex.stackexchange.com/a/476052/104425
	\item Применение разработанной информационной технология позволяет расширить системы управления для принятия обоснованных решений на основе временных рядов, описывающих уникальные объекты. %\todo{на такие то случаи....} в задачах управления позволяет повысить обоснованность принимаемых решений за счет адекватности учета характеристик временных рядов в задачах их прогнозирования. %нестационарньсть характеристик
	
	\item Разработанное алгоритмическое обеспечение и программная реализация, которая не предъявляет особых требований к архитектуре вычислительных сред.
	
	\item Результаты проведенных вычислительные эксперименты иллюстрируют работоспособность информационной технологии.
\end{enumerate}
%В папке Documents можно ознакомиться с решением совета из Томского~ГУ
%(в~файле \verb+Def_positions.pdf+), где обоснованно даются рекомендации
%по~формулировкам защищаемых положений.

{\specialityRelation} Содержание диссертации соответствует следующим пунктам паспорта специальности 2.3.1. <<Системный анализ, управление и обработка информации, статистика>> по следующим направлениям исследований:
\begin{itemize}
	\item п. 1 паспорта специальности: Теоретические основы и методы системного анализа, оптимизации, 	управления, принятия решений, обработки информации и искусственного интеллекта.
	\item п. 4 паспорта специальности: Разработка методов и алгоритмов решения задач системного анализа, оптимизации, управления, принятия решений, обработки информации и искусственного интеллекта.
\end{itemize}

{\reliability} полученных результатов обеспечивается корректностью математических преобразований, отсутствием противоречий с известными положениями теории и практики прогнозирования временных рядов, адекватным учетом нечеткость входных данных и иллюстрируется результатами вычислительных экспериментов о работоспособности информационной технологии, а также результатами в публикациях в рецензируемых журналах и на конференциях.

{\probation}
Основные результаты работы докладывались~на следующих конференциях:
\begin{enumerate}
	\item Международная конференция <<Перспективные компьютерные и цифровые технологии» (ACDT 2021)>>, г. Белгород, 2021.
	\item XV Международная научная конференция <<Параллельные вычислительные технологии (ПаВТ) 2021>>, г. Волгоград, 2021.
	\item XI Международной научно-практической конференции <<Интегрированные модели и мягкие вычисления в искусственном интеллекте (ИММВ-2022)>>, г. Коломна, 2022 г.
	\item XX Национальная конференция по искусственному интеллекту с международным участием (КИИ-2022), г. Москва, 2022.
	\item XVII Международная научная конференция <<Параллельные вычислительные технологии (ПаВТ) 2023>>, г. Санкт-Петербург, 2023.
	\item XXI Национальная конференция по искусственному интеллекту с международным участием (КИИ-2023), г. Смоленск, 2023.
	\item Х Всероссийская научно-техническая конференция «Информационные технологии в науке, образовании и производстве» (ИТНОП-2025), г. Орел, 2025.
\end{enumerate}

{\contribution} все результаты диссертационного исследования получены либо автором лично, либо при его непосредственном участии.

\ifnumequal{\value{bibliosel}}{0}
{%%% Встроенная реализация с загрузкой файла через движок bibtex8. (При желании, внутри можно использовать обычные ссылки, наподобие `\cite{vakbib1,vakbib2}`).
    {\publications} Основные результаты по теме диссертации изложены
    в~XX~печатных изданиях,
    X из которых изданы в журналах, рекомендованных ВАК,
    X "--- в тезисах докладов.
}%
{%%% Реализация пакетом biblatex через движок biber
    \begin{refsection}[bl-author, bl-registered]
        % Это refsection=1.
        % Процитированные здесь работы:
        %  * подсчитываются, для автоматического составления фразы "Основные результаты ..."
        %  * попадают в авторскую библиографию, при usefootcite==0 и стиле `\insertbiblioauthor` или `\insertbiblioauthorgrouped`
        %  * нумеруются там в зависимости от порядка команд `\printbibliography` в этом разделе.
        %  * при использовании `\insertbiblioauthorgrouped`, порядок команд `\printbibliography` в нём должен быть тем же (см. biblio/biblatex.tex)
        %
        % Невидимый библиографический список для подсчёта количества публикаций:
        \phantom{\printbibliography[heading=nobibheading, section=1, env=countauthorvak,          keyword=biblioauthorvak]%
        \printbibliography[heading=nobibheading, section=1, env=countauthorwos,          keyword=biblioauthorwos]%
        \printbibliography[heading=nobibheading, section=1, env=countauthorscopus,       keyword=biblioauthorscopus]%
        \printbibliography[heading=nobibheading, section=1, env=countauthorconf,         keyword=biblioauthorconf]%
        \printbibliography[heading=nobibheading, section=1, env=countauthorother,        keyword=biblioauthorother]%
        \printbibliography[heading=nobibheading, section=1, env=countregistered,         keyword=biblioregistered]%
        \printbibliography[heading=nobibheading, section=1, env=countauthorpatent,       keyword=biblioauthorpatent]%
        \printbibliography[heading=nobibheading, section=1, env=countauthorprogram,      keyword=biblioauthorprogram]%
        \printbibliography[heading=nobibheading, section=1, env=countauthor,             keyword=biblioauthor]%
        \printbibliography[heading=nobibheading, section=1, env=countauthorvakscopuswos, filter=vakscopuswos]%
        \printbibliography[heading=nobibheading, section=1, env=countauthorscopuswos,    filter=scopuswos]}%
        %
        \nocite{*}%
        %
        {\publications} Основные результаты по теме диссертации изложены в~\arabic{citeauthor}~печатных изданиях,
        \arabic{citeauthorvak} из которых изданы в журналах, рекомендованных ВАК%
        \ifnum \value{citeauthorscopuswos}>0%
            , \arabic{citeauthorscopuswos} "--- в~периодических научных журналах, индексируемых Web of~Science и Scopus%
        \fi%
        \ifnum \value{citeauthorconf}>0%
            , \arabic{citeauthorconf} "--- в~тезисах докладов.
        \else%
            .
        \fi%
        \ifnum \value{citeregistered}=1%
            \ifnum \value{citeauthorpatent}=1%
                Зарегистрирован \arabic{citeauthorpatent} патент.
            \fi%
            \ifnum \value{citeauthorprogram}=1%
                Зарегистрирована \arabic{citeauthorprogram} программа для ЭВМ.
            \fi%
        \fi%
        \ifnum \value{citeregistered}>1%
            Зарегистрированы\ %
            \ifnum \value{citeauthorpatent}>0%
            \formbytotal{citeauthorpatent}{патент}{}{а}{}%
            \ifnum \value{citeauthorprogram}=0 . \else \ и~\fi%
            \fi%
            \ifnum \value{citeauthorprogram}>0%
            \formbytotal{citeauthorprogram}{программ}{а}{ы}{} для ЭВМ.
            \fi%
        \fi%
        % К публикациям, в которых излагаются основные научные результаты диссертации на соискание учёной
        % степени, в рецензируемых изданиях приравниваются патенты на изобретения, патенты (свидетельства) на
        % полезную модель, патенты на промышленный образец, патенты на селекционные достижения, свидетельства
        % на программу для электронных вычислительных машин, базу данных, топологию интегральных микросхем,
        % зарегистрированные в установленном порядке.(в ред. Постановления Правительства РФ от 21.04.2016 N 335)
    \end{refsection}%
    \begin{refsection}[bl-author, bl-registered]
        % Это refsection=2.
        % Процитированные здесь работы:
        %  * попадают в авторскую библиографию, при usefootcite==0 и стиле `\insertbiblioauthorimportant`.
        %  * ни на что не влияют в противном случае
        \nocite{vakbib2}%vak
        \nocite{patbib1}%patent
        \nocite{progbib1}%program
        \nocite{bib1}%other
        \nocite{confbib1}%conf
    \end{refsection}%
        %
        % Всё, что вне этих двух refsection, это refsection=0,
        %  * для диссертации - это нормальные ссылки, попадающие в обычную библиографию
        %  * для автореферата:
        %     * при usefootcite==0, ссылка корректно сработает только для источника из `external.bib`. Для своих работ --- напечатает "[0]" (и даже Warning не вылезет).
        %     * при usefootcite==1, ссылка сработает нормально. В авторской библиографии будут только процитированные в refsection=0 работы.
}

% При использовании пакета \verb!biblatex! будут подсчитаны все работы, добавленные
% в файл \verb!biblio/author.bib!. Для правильного подсчёта работ в~различных
% системах цитирования требуется использовать поля:
% \begin{itemize}
%         \item \texttt{authorvak} если публикация индексирована ВАК,
%         \item \texttt{authorscopus} если публикация индексирована Scopus,
%         \item \texttt{authorwos} если публикация индексирована Web of Science,
%         \item \texttt{authorconf} для докладов конференций,
%         \item \texttt{authorpatent} для патентов,
%         \item \texttt{authorprogram} для зарегистрированных программ для ЭВМ,
%         \item \texttt{authorother} для других публикаций.
% \end{itemize}
% Для подсчёта используются счётчики:
% \begin{itemize}
%         \item \texttt{citeauthorvak} для работ, индексируемых ВАК,
%         \item \texttt{citeauthorscopus} для работ, индексируемых Scopus,
%         \item \texttt{citeauthorwos} для работ, индексируемых Web of Science,
%         \item \texttt{citeauthorvakscopuswos} для работ, индексируемых одной из трёх баз,
%         \item \texttt{citeauthorscopuswos} для работ, индексируемых Scopus или Web of~Science,
%         \item \texttt{citeauthorconf} для докладов на конференциях,
%         \item \texttt{citeauthorother} для остальных работ,
%         \item \texttt{citeauthorpatent} для патентов,
%         \item \texttt{citeauthorprogram} для зарегистрированных программ для ЭВМ,
%         \item \texttt{citeauthor} для суммарного количества работ.
% \end{itemize}
% Счётчик \texttt{citeexternal} используется для подсчёта процитированных публикаций;
% \texttt{citeregistered} "--- для подсчёта суммарного количества патентов и программ для ЭВМ.

%Для добавления в список публикаций автора работ, которые не были процитированы в
%автореферате, требуется их~перечислить с использованием команды \verb!\nocite! в
%\verb!Synopsis/content.tex!.
