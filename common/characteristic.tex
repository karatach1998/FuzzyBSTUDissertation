
{\actuality} Высокопроизводительный интеллектуальный анализ данных дает возможность принимать обоснованные решения на основе знаний, получаемых посредством обработки данных со скоростью близкой к реальному времни. Семейство методов мягких вычислений с применением техник высокопроизводительного анализа данных открывает возможность находить закономерности и взаимосвязи в данных, содержащих неоднозначность, с высокой скоростью.

Существующие реализации кейсов мягких вычислений опираются на использованием методов нечеткого вывода выработанных Мамдани, Ларсеном, Такаги и Сугено. \note[blue]{Написать про логический подход}. В этих случаях системы вывода работают с четкими входами, что позволяет упростить реализацию процедуры вывода. Однако такое упрощение порождает отличие с классическим подходом Заде.

В теории нечеткой логики нечеткий логический вывод реализуется за с помощью обобщенных нечетких правил \textit{modus ponens} и \textit{modus tollens} на основе \textit{композиционного правила вывода}. При нескольких входах вычисление по данным правилом приводит к экспоненциальной зависимости вычислительной сложности от количества входов. Данное ограничение является основным препятстивем для применения нечеткого логического вывода с несколькими посылками, тогда как необходимость анализа многомерных данных \todo{является} актуальной задачей, \todo{например, \dots}.

\note[blue]{Возможно стоит описать изученность проблемы в России?}



\ifsynopsis
Этот абзац появляется только в~автореферате.
Для формирования блоков, которые будут обрабатываться только в~автореферате,
заведена проверка условия \verb!\!\verb!ifsynopsis!.
Значение условия задаётся в~основном файле документа (\verb!synopsis.tex! для
автореферата).
\else
%Этот абзац появляется только в~диссертации.
%Через проверку условия \verb!\!\verb!ifsynopsis!, задаваемого в~основном файле
%документа (\verb!dissertation.tex! для диссертации), можно сделать новую
%команду, обеспечивающую появление цитаты в~диссертации, но~не~в~автореферате.
\fi

% {\progress}
% Этот раздел должен быть отдельным структурным элементом по
% ГОСТ, но он, как правило, включается в описание актуальности
% темы. Нужен он отдельным структурынм элемементом или нет ---
% смотрите другие диссертации вашего совета, скорее всего не нужен.

{\aim} данной работы является повышение производительности анализа неопределенных данных путем разработки математического и программного обеспечения на основе нечетких систем при несинглтонной фаззификации.

Для~достижения поставленной цели необходимо было решить следующие {\tasks}:
\begin{enumerate}[beginpenalty=10000] % https://tex.stackexchange.com/a/476052/104425
  \item Провести обзор проблем и предлагаемых подходов построения и реализации нечетких систем анализа данных с качественным описанием.
  \item Разработать метод вывода на основое нечеткого значения истинности для системы MISO-сструктуры логического типа, обеспечивающий полиномиальную вычислительную сложность.
  \item Выполнить программную реализацию выработанного метода нечеткого вывода с использованием технологии параллельных вычислений CUDA, обеспечив эффективность реализации за счет внедрения оптимизаций алгоритма вывода.
  \item Применить разработанный модуль нечеткого логического вывода для высокопроизводительного анализа зашумленных данных в выбранной предметной области.
\end{enumerate}


{\novelty}
\begin{enumerate}[beginpenalty=10000] % https://tex.stackexchange.com/a/476052/104425
  \item Впервые применено нечеткое значение истинности и принцип обобщения для получения выходного значения при нескольких нечетких входах в соответствии с обобщенным нечетким правилом вывода \textit{modus ponens} для нечетких систем логического типа, в результате чего была получена новая структура базы правил: <<Если \textit{истинно}, то $B_k$>>.
  \item Разработан метод нечеткого вывода логического типа с использованием нечеткого значения истинности, имеющий полиномиальную вычислительную сложность при многих нечетких входах.
  \item Разработан метод регрессии временных рядов с нечеткими оценками измеренных значений на основе предложенного метода нечеткого вывода логического типа и алгоритм построения базы правил \dots.
  \item Разработан параллельный алгоритм, реализующий нечеткий вывод на основе нечеткого значения истинности с применением отбора \dots
\end{enumerate}

\textbf{Теоретическая значимость} заключается в расширении класса задач анализа данных, эффективно решаемых при помощи нечеткого моделирования, соответствующего теории нечеткого вывода Л. Заде.

\textbf{Практическая значимость} \ldots

{\methods} В работе использованы методы теории нечетких множеств, нечетких отношений, нечеткого логического вывода, принятия решений и мягких вычислений.

{\defpositions}
\begin{enumerate}[beginpenalty=10000] % https://tex.stackexchange.com/a/476052/104425
  \item Метод вывода для нечетких систем логического типа на основе нечеткого значения истинности, имеющий полиномиальную вычислительную сложность при многих нечетких входах.
  \item Метод регрессии для временных рядов с нечеткими оценками измеренных значений на основе метода нечеткого вывода логического типа.
  \item Разработанный вид нечетких правил <<Если \textit{истинно}, то $B_k$>>.
  \item Разработанный параллельный алгоритм для предложенного метода вывода и его эффективная реализация на графическом процессоре с поддержкой технологии CUDA.
  \item \dots
  \item \dots
\end{enumerate}
%В папке Documents можно ознакомиться с решением совета из Томского~ГУ
%(в~файле \verb+Def_positions.pdf+), где обоснованно даются рекомендации
%по~формулировкам защищаемых положений.

\textbf{Соответствие диссертации научной специальности.} Диссертационная работа соответствует паспорту специальности 2.3.1. <<Системный анализ, управление и обработка информации, статистика>> по следующим областям исследования:
\begin{itemize}
  \item п. 10 <<Методы и алгоритмы интеллектуальной поддержки при принятии управленческих решений в технических системах>>.
\end{itemize}

\textbf{Внедрение результатов диссертационного исследования.} Результаты диссертационной работы внедрены \todo{\dots}. Предложенные алгоритмы также использованы при выполнении научного проек­та при поддержке РФФИ №20-07-00030 «Разработка высокопроизводительных методов интеллектуального анализа данных на основе нечеткого моделиро­вания и создание компьютерной системы поддержки принятия решений для классификации и прогнозирования».

{\reliability} полученных результатов обеспечивается корректным применением математического аппарата, экспериментальными исследованиями, апробацией на научно-практических конференциях, доказанностью выводов.

{\probation}
Основные результаты работы докладывались~на:
\begin{enumerate}
	\item Международная конференция <<Перспективные компьютерные и цифровые технологии» (ACDT 2021)>>,
	г. Белгород, 2021.
	\item XV Международная научная конференция <<Параллельные вычислительные технологии (ПаВТ) 2021>>, г. Волгоград, 2021.
	\item XI Международной научно-практической конференции <<Интегрированные модели и мягкие вычисления в искусственном интеллекте (ИММВ-2022)>>, г. Коломна, 2022 г.
	\item XX Национальная конференция по искусственному интеллекту с международным участием (КИИ-2022), г. Москва, 2022.
	\item XVII Международная научная конференция <<Параллельные вычислительные технологии (ПаВТ) 2021>>, г. Санкт-Петербург, 2023.
	\item XXI Национальная конференция по искусственному интеллекту с международным участием (КИИ-2023), г. Смоленск, 2023.
\end{enumerate}

{\contribution} Все изложенные в диссертации результаты исследования получены либо соискателем лично, либо при его непосредственном участии.

\ifnumequal{\value{bibliosel}}{0}
{%%% Встроенная реализация с загрузкой файла через движок bibtex8. (При желании, внутри можно использовать обычные ссылки, наподобие `\cite{vakbib1,vakbib2}`).
    {\publications} Основные результаты по теме диссертации изложены
    в~XX~печатных изданиях,
    X из которых изданы в журналах, рекомендованных ВАК,
    X "--- в тезисах докладов.
}%
{%%% Реализация пакетом biblatex через движок biber
    \begin{refsection}[bl-author, bl-registered]
        % Это refsection=1.
        % Процитированные здесь работы:
        %  * подсчитываются, для автоматического составления фразы "Основные результаты ..."
        %  * попадают в авторскую библиографию, при usefootcite==0 и стиле `\insertbiblioauthor` или `\insertbiblioauthorgrouped`
        %  * нумеруются там в зависимости от порядка команд `\printbibliography` в этом разделе.
        %  * при использовании `\insertbiblioauthorgrouped`, порядок команд `\printbibliography` в нём должен быть тем же (см. biblio/biblatex.tex)
        %
        % Невидимый библиографический список для подсчёта количества публикаций:
        \phantom{\printbibliography[heading=nobibheading, section=1, env=countauthorvak,          keyword=biblioauthorvak]%
        \printbibliography[heading=nobibheading, section=1, env=countauthorwos,          keyword=biblioauthorwos]%
        \printbibliography[heading=nobibheading, section=1, env=countauthorscopus,       keyword=biblioauthorscopus]%
        \printbibliography[heading=nobibheading, section=1, env=countauthorconf,         keyword=biblioauthorconf]%
        \printbibliography[heading=nobibheading, section=1, env=countauthorother,        keyword=biblioauthorother]%
        \printbibliography[heading=nobibheading, section=1, env=countregistered,         keyword=biblioregistered]%
        \printbibliography[heading=nobibheading, section=1, env=countauthorpatent,       keyword=biblioauthorpatent]%
        \printbibliography[heading=nobibheading, section=1, env=countauthorprogram,      keyword=biblioauthorprogram]%
        \printbibliography[heading=nobibheading, section=1, env=countauthor,             keyword=biblioauthor]%
        \printbibliography[heading=nobibheading, section=1, env=countauthorvakscopuswos, filter=vakscopuswos]%
        \printbibliography[heading=nobibheading, section=1, env=countauthorscopuswos,    filter=scopuswos]}%
        %
        \nocite{*}%
        %
        {\publications} Основные результаты по теме диссертации изложены в~\arabic{citeauthor}~печатных изданиях,
        \arabic{citeauthorvak} из которых изданы в журналах, рекомендованных ВАК%
        \ifnum \value{citeauthorscopuswos}>0%
            , \arabic{citeauthorscopuswos} "--- в~периодических научных журналах, индексируемых Web of~Science и Scopus%
        \fi%
        \ifnum \value{citeauthorconf}>0%
            , \arabic{citeauthorconf} "--- в~тезисах докладов.
        \else%
            .
        \fi%
        \ifnum \value{citeregistered}=1%
            \ifnum \value{citeauthorpatent}=1%
                Зарегистрирован \arabic{citeauthorpatent} патент.
            \fi%
            \ifnum \value{citeauthorprogram}=1%
                Зарегистрирована \arabic{citeauthorprogram} программа для ЭВМ.
            \fi%
        \fi%
        \ifnum \value{citeregistered}>1%
            Зарегистрированы\ %
            \ifnum \value{citeauthorpatent}>0%
            \formbytotal{citeauthorpatent}{патент}{}{а}{}%
            \ifnum \value{citeauthorprogram}=0 . \else \ и~\fi%
            \fi%
            \ifnum \value{citeauthorprogram}>0%
            \formbytotal{citeauthorprogram}{программ}{а}{ы}{} для ЭВМ.
            \fi%
        \fi%
        % К публикациям, в которых излагаются основные научные результаты диссертации на соискание учёной
        % степени, в рецензируемых изданиях приравниваются патенты на изобретения, патенты (свидетельства) на
        % полезную модель, патенты на промышленный образец, патенты на селекционные достижения, свидетельства
        % на программу для электронных вычислительных машин, базу данных, топологию интегральных микросхем,
        % зарегистрированные в установленном порядке.(в ред. Постановления Правительства РФ от 21.04.2016 N 335)
    \end{refsection}%
    \begin{refsection}[bl-author, bl-registered]
        % Это refsection=2.
        % Процитированные здесь работы:
        %  * попадают в авторскую библиографию, при usefootcite==0 и стиле `\insertbiblioauthorimportant`.
        %  * ни на что не влияют в противном случае
        \nocite{vakbib2}%vak
        \nocite{patbib1}%patent
        \nocite{progbib1}%program
        \nocite{bib1}%other
        \nocite{confbib1}%conf
    \end{refsection}%
        %
        % Всё, что вне этих двух refsection, это refsection=0,
        %  * для диссертации - это нормальные ссылки, попадающие в обычную библиографию
        %  * для автореферата:
        %     * при usefootcite==0, ссылка корректно сработает только для источника из `external.bib`. Для своих работ --- напечатает "[0]" (и даже Warning не вылезет).
        %     * при usefootcite==1, ссылка сработает нормально. В авторской библиографии будут только процитированные в refsection=0 работы.
}

% При использовании пакета \verb!biblatex! будут подсчитаны все работы, добавленные
% в файл \verb!biblio/author.bib!. Для правильного подсчёта работ в~различных
% системах цитирования требуется использовать поля:
% \begin{itemize}
%         \item \texttt{authorvak} если публикация индексирована ВАК,
%         \item \texttt{authorscopus} если публикация индексирована Scopus,
%         \item \texttt{authorwos} если публикация индексирована Web of Science,
%         \item \texttt{authorconf} для докладов конференций,
%         \item \texttt{authorpatent} для патентов,
%         \item \texttt{authorprogram} для зарегистрированных программ для ЭВМ,
%         \item \texttt{authorother} для других публикаций.
% \end{itemize}
% Для подсчёта используются счётчики:
% \begin{itemize}
%         \item \texttt{citeauthorvak} для работ, индексируемых ВАК,
%         \item \texttt{citeauthorscopus} для работ, индексируемых Scopus,
%         \item \texttt{citeauthorwos} для работ, индексируемых Web of Science,
%         \item \texttt{citeauthorvakscopuswos} для работ, индексируемых одной из трёх баз,
%         \item \texttt{citeauthorscopuswos} для работ, индексируемых Scopus или Web of~Science,
%         \item \texttt{citeauthorconf} для докладов на конференциях,
%         \item \texttt{citeauthorother} для остальных работ,
%         \item \texttt{citeauthorpatent} для патентов,
%         \item \texttt{citeauthorprogram} для зарегистрированных программ для ЭВМ,
%         \item \texttt{citeauthor} для суммарного количества работ.
% \end{itemize}
% Счётчик \texttt{citeexternal} используется для подсчёта процитированных публикаций;
% \texttt{citeregistered} "--- для подсчёта суммарного количества патентов и программ для ЭВМ.

%Для добавления в список публикаций автора работ, которые не были процитированы в
%автореферате, требуется их~перечислить с использованием команды \verb!\nocite! в
%\verb!Synopsis/content.tex!.
